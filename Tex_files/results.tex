% !TeX root = main.tex
\chapter{Results}\label{chap:results}


\section{Quartz crystal Microbalance and Ion current Probe}
To systematically characterize the plasma dynamics and deposition behavior, measurements were conducted using both a quartz crystal microbalance (QCM) for mass deposition and a biased ion collector probe for ion current density.

The experimental parameter space spanned three variables: distance from the macroparticle filter (10--20~cm), applied magnetic field strength (0--0.25~T), and nitrogen background pressure (0--0.3~Pa), with all permutations measured as listed in Table~\ref{tab:comprehensive_measurements}. At this stage, the raw quantities (mass flux in nanograms/cm$^2$s and ion current in milliamperes) are not directly comparable and only trends will be observed. Their relationship will be established through flux calculations in Section~\ref{sec:flux}.

\iffalse
\begin{table}[h]
	\centering
	\begin{tabularx}{0.85\textwidth}{|X|X|X|}
		\hline
		\textbf{Distance (cm)} & \textbf{Pressure (Pa)} & \textbf{Magnetic Field (T)} \\ \hline
		10 & 0 & 0 \\ \hline
		12 & 0.025 & 0.05 \\ \hline
		14 & 0.05 & 0.1 \\ \hline
		16 & 0.075 & 0.15 \\ \hline
		18 & 0.1 & 0.2 \\ \hline
		20 & 0.2 & 0.25 \\ \hline
		& 0.3 & \\ \hline
	\end{tabularx}
	\caption{Parameters used for the QCM/Ion current probe}
	\label{tab:parameters}
\end{table}
\fi

Figure~\ref{fig:3D overview} provides a three-dimensional visualization of the complete dataset, illustrating how the mass flux and ion current vary simultaneously with all three control parameters. Several global trends are immediately apparent: both quantities decrease with increasing distance due to plasma expansion, increase with applied magnetic field due to enhanced plasma confinement and ionization, and exhibit complex pressure dependence that warrants detailed investigation.


\begin{figure}[H]
\centering
\begin{subfigure}[b]{\textwidth}
  \includegraphics[width=1\linewidth]{Figures/results 1/3Dmass.png}
  \caption{}
  \label{fig:3Dmass} 
\end{subfigure}

\begin{subfigure}[b]{\textwidth}
  \includegraphics[width=1\linewidth]{Figures/results 1/3Dion.png}
  \caption{}
  \label{fig:3Dion}
\end{subfigure}
\caption{Three-dimensional visualization of (a) mass flux and (b) pulse-averaged ion current as functions of distance (10--20~cm), magnetic field strength (0--0.25~T), and nitrogen pressure (0--0.3~Pa).}
\label{fig:3D overview}
\end{figure}

To systematically dissect these multidimensional trends, the following subsections examine cross-sections through the parameter space in order of increasing complexity.\\
We begin with the metallic case (vacuum conditions, no reactive gas) to establish baseline behavior as functions of distance and magnetic field. 
Subsequently, we introduce nitrogen pressure as an additional variable and examine its interplay with geometry and magnetic confinement. 
The parameter combinations analyzed in detail: distances of 10, 14, and 20~cm; magnetic fields of 0, 0.15, and 0.25~T; and pressures of 0.1 and 0.3~Pa. These parameters were selected because they correspond to the conditions used for energy-resolved mass spectrometry (ERMS) measurements and thin film deposition, enabling an introduction.


\subsection{Metallic Case (No Nitrogen)}\label{subsec:metal}
In the absence of reactive gas, the plasma expansion and deposition dynamics are governed solely by geometric dilution and magnetic confinement. Figure~\ref{fig:metallic_mass/ion} presents mass flux and ion current as functions of magnetic field strength for three representative distances.
\begin{figure}[h]
	\centering
	\begin{subfigure}{.5\textwidth}
		\centering
		\includegraphics[width=\linewidth]{Figures/results 1/MagField_vs_Distance_P0.0Pa_Mass.png}
		\caption{}
	\end{subfigure}%
	\begin{subfigure}{.5\textwidth}
		\centering
		\includegraphics[width=\linewidth]{Figures/results 1/MagField_vs_Distance_P0.0Pa_IonCurrent.png}
		\caption{}
	\end{subfigure}
	\caption{Metallic case measurements showing (a) mass flux and (b) the ion current, each plotted as a function of magnetic field strength. Data selected for three distances (10, 14, 20 cm), with error bars representing variation within pulses.}
	\label{fig:metallic_mass/ion}
\end{figure}

As shown in Figure \ref{fig:metallic_mass/ion}, both the mass and the ion current increase with applied magnetic field. The enhancement is most pronounced at the shortest distance (10~cm), where the magnetic field can effectively guide ions before significant radial expansion occurs. At 20~cm, the plasma has already expanded substantially, reducing the relative impact of magnetic confinement on the collected flux.\\

A notable anomaly appears at 0.05~T, where both quantities drop below their zero-field values. This counterintuitive behavior is attributed to a magnetic mirror effect at the entrance of the EM coil. When plasma transitions from the weak fringing field ($\sim 0.01$~T) into the stronger coil field (0.05~T), conservation of the magnetic moment causes electrons with significant perpendicular velocity components to be reflected \cite{magnetic_mirror,cathodic_arcs}. This creates a localized space-charge layer that retards ion flow, temporarily reducing both the ion flux and deposition rate. At higher fields (0.1~T and above), the beneficial effects of plasma compression and enhanced ionization \cite{decoupling_kalanov_2025} overcome this mirror loss. Similar behavior is observed across all experimental conditions involving increasing magnetic field.


\subsection{Distance as a variable}\label{subsec:distance}

Figure~\ref{fig:distance_0.25T_mass/ion} examines the effect of source-to-substrate distance under fixed magnetic confinement (0.25~T) for both metallic and reactive conditions. Four nitrogen pressures are compared: 0~Pa (metallic), 0.1~Pa, 0.2~Pa, and 0.3~Pa.

\begin{figure}[ht]
	\centering
	\begin{subfigure}{.5\textwidth}
		\centering
		\includegraphics[width=\linewidth]{Figures/results 1/Distance_vs_Pressure_B0.25T_Mass.png}
		\caption{}
	\end{subfigure}%
	\begin{subfigure}{.5\textwidth}
		\centering
		\includegraphics[width=\linewidth]{Figures/results 1/Distance_vs_Pressure_B0.25T_IonCurrent.png}
		\caption{}
	\end{subfigure}
	\caption{QCM and Ion Probe measurements showing (a) mass flux and (b) the ion current, each plotted as a function of distance. Data shown for representative pressures, with error bars representing variation within pulses.}
	\label{fig:distance_0.25T_mass/ion}
\end{figure}

Both mass flux and ion current decrease with distance, reflecting the natural expansion and dilution of the plasma plume. In the case for the mass, this decay is relatively gradual and follows an approximate $1/r^2$ dependence expected for free expansion \cite[Chap.~4.3]{cathodic_arcs}.\\

In contrast, the ion current exhibits a far more dramatic pressure dependence. The steep drop in measured current when nitrogen is introduced primarily reflects charge-exchange collisions, in which fast metal ions transfer charge to slow nitrogen molecules, producing fast neutral metal atoms and slow N$^+$/N$_2^+$ ions \cite[Chap.~9.4]{cathodic_arcs}. The resulting neutral flux is invisible to the biased probe, leading to an apparent reduction in "ion" current even though the total metal flux (ions plus neutrals) may remain comparable. Additionally, nitrogen ions contribute less to the measured current due to their lower charge states ($Q \approx 1$) compared to multiply charged metal ions ($Q \approx 2$) \cite{bendikt2012}.\\

The contrast between mass with a moderate pressure effect and current with a noticeably stronger pressure effect confirms that charge-exchange neutralization, rather than simple scattering loss, is the dominant process at short distances in reactive mode. This interpretation will be further validated through ERMS charge-state measurements in Section~\ref{sec:erms_results}.\\

To isolate the effect of magnetic confinement in reactive mode, we next examine field strength as an independent variable.



\subsection{Magnetic Field as a variable}\label{subsec:mag_field}

Figure~\ref{fig:mag_field_14cm_mass/ion} presents mass flux and ion current as functions of magnetic field strength at a fixed intermediate distance (14~cm) for the same set of pressures.

\begin{figure}[ht]
	\centering
	\begin{subfigure}{.5\textwidth}
		\centering
		\includegraphics[width=\linewidth]{Figures/results 1/MagField_vs_Pressure_D14cm_Mass.png}
		\caption{}
	\end{subfigure}%
	\begin{subfigure}{.5\textwidth}
		\centering
		\includegraphics[width=\linewidth]{Figures/results 1/MagField_vs_Pressure_D14cm_IonCurrent.png}
		\caption{}
	\end{subfigure}
	\caption{QCM and Ion Probe measurements showing (a) mass flux and (b) the ion current, each plotted as a function of Magnetic Field. Data shown for representative pressures, with error bars representing variation within pulses.}
	\label{fig:mag_field_14cm_mass/ion}
\end{figure}

The magnetic mirror anomaly at 0.05~T, clearly visible in the 10~cm metallic data (Figure~\ref{fig:metallic_mass/ion}), is significantly attenuated at 14~cm. Both mass and current remain approximately constant between 0 and 0.05~T, suggesting that the adverse mirror effect is either weaker after 4~cm of additional expansion or is masked by increased statistical noise at this intermediate distance.\\

Above 0.1~T the trends shows a clear constant increase in both quantities, confirming that magnetic confinement enhances plasma density and ion flux at the substrate.
In the case of the ion current, there exists a pronounced divergence between metallic and reactive conditions at high fields. This behavior is consistent with enhanced charge-exchange rates under strong confinement: tighter magnetic focusing increases the ion path length through the background gas, providing more opportunities for neutralization before reaching the substrate \cite[Chap.~9.4]{cathodic_arcs}. The result is that, although the magnetic field successfully generates more plasma at the source, a larger fraction arrives as less ionized ions or neutrals in reactive mode.\\

These observations demonstrate that background gas pressure has a relatively minor influence on total mass flux compared to distance or magnetic field, but significantly affects the charge-state composition of the arriving flux. This distinction motivates the final cross-section through the parameter space: examining pressure as the primary variable.

\subsection{Nitrogen pressure as a variable}\label{subsec:nitrogen_pressure}
Figure~\ref{fig:pressure_0.25T_mass/ion} presents mass flux and ion current as functions of nitrogen pressure at maximum magnetic confinement (0.25~T) for three distances (10, 14, and 20~cm).

\begin{figure}[ht]
	\centering
	\begin{subfigure}{.5\textwidth}
		\centering
		\includegraphics[width=\linewidth]{Figures/results 1/Pressure_vs_Distance_B0.25T_Mass.png}
		\caption{}
	\end{subfigure}%
	\begin{subfigure}{.5\textwidth}
		\centering
		\includegraphics[width=\linewidth]{Figures/results 1/Pressure_vs_Distance_B0.25T_IonCurrent.png}
		\caption{}
	\end{subfigure}
	\caption{QCM and Ion Probe measurements showing (a) mass flux and (b) the ion current, each plotted as a function of Pressure. Data shown for representative pressures, with error bars representing variation within pulses.}
	\label{fig:pressure_0.25T_mass/ion}
\end{figure}

At 10~cm, both mass and ion current exhibit complex, pressure dependence with substantial variability. This behavior reflects collisional deflection of the plasma plume: nitrogen molecules scatter metal ions through momentum transfer broadening the spatial distribution \cite{anders2008,boxman1995}.\\

At 14~cm and 20~cm, the pressure dependence becomes negligible. Both mass and ion current remain essentially constant across the entire pressure range. After sufficient expansion, the plasma has already undergone extensive collisional scattering regardless of absolute pressure, resulting in a broad, diffuse distribution.\\

The non-monotonic behavior observed at 10~cm, particularly the apparent increase in both quantities at low pressures (0.025--0.075~Pa), suggests complex interactions between unequal plasma expansion dynamics and collisional deflection at short standoff distances. The mechanisms underlying this local maximum warrant further investigation but remain beyond the scope of the present work.


\section{Mass spectrometry Results}
\begin{itemize}
	\item shortly about mean charge state and energies not a focus tho
	\item maybe a table and a picture
\end{itemize}


\begin{figure}[h]
	\centering
	\includegraphics[width=\linewidth]{"Figures/results 2/Distance_0.25T_0.0Pa"}
	\caption{0Pa 0.25T}
	\label{fig:ERMS_distance0Pa}
\end{figure}
\begin{figure}[h]
	\centering
	\includegraphics[width=\linewidth]{"Figures/results 2/Distance_0.25T_0.3Pa"}
	\caption{0.3Pa 0.25T}
	\label{fig:ERMS_distance0.3Pa}
\end{figure}
\begin{figure}[h]
	\centering
	\includegraphics[width=\linewidth]{"Figures/results 2/Pressure_10cm_0.25T"}
	\caption{10cm 0.25T}
	\label{fig:ERMS_pressure10cm0}
\end{figure}
Missing 14cm 0.25T

\newpage
\section{Ex situ Results}
\begin{figure}[ht]
	\centering
	\begin{subfigure}{.5\textwidth}
		\centering
		\includegraphics[width=\linewidth]{Figures/results 2/PXL_20251114_084343196.jpg}
		\caption{}
	\end{subfigure}%
	\begin{subfigure}{.5\textwidth}
		\centering
		\includegraphics[width=\linewidth]{Figures/results 2/PXL_20251117_130844177.jpg}
		\caption{}
	\end{subfigure}
	\caption{\textcolor{red}{to be written picture taken with phone a) with nitrogen and b) metallic... colors are wrong need to change it haha}}
	\label{fig:plasma}
\end{figure}

A total of [insert number of films once i ve done all of them, proabably like 6-8] were taking into account for further measurements and looked at in the following sections. Blablabla
\subsection{Profilometry}\label{sec:profilometry}

Samples were prepared by placing a straight marker line near the edge of each substrate prior to deposition. This marker could be cleanly removed by ultrasonic washing after film growth, revealing the deposition step for thickness analysis. Profilometry measurements were performed at three positions on each sample, one near the center and one near each edge. The three measurements were averaged to obtain the mean film thickness. All resulting values are summarized in Table \ref{tab:profilometry_results}.

\begin{table}[h]
	\centering
	\begin{tabularx}{0.85\textwidth}{|X|X|X|}
		\hline
	\end{tabularx}
	\caption{Profilometry thickness measurements for the deposited films}
	\label{tab:profilometry_results}
\end{table}

\subsection{XRD}
\subsection{XRR}
\subsection{EDX}


\section{Fluxes}\label{sec:flux}
