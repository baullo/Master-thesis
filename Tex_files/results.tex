% !TeX root = main.tex
\chapter{Results}
\section{Langmuir Probe Bias Voltage Characterization}
\subsection{Validation of Langmuir Probe Operation}

Before plasma diagnostics were performed, the functionality of the custom-built Langmuir probe was verified by checking the characteristic voltage current curve (I-V curve) and fitting it against the theoretical relationship \ref{eq:langmuir}. Additionally the appropriate negative bias voltage for ion saturation measurements was determined . The goal was to ensure that the probe operates in a regime where the collected current is dominated by ions, excluding contributions from electrons.\\

The bias voltage tests were performed without nitrogen in the chamber, at a fixed distance of 10 cm from the macroparticle filter, and with an EM coil strength of 0.25 T. Figure \ref{fig:bias_current} displays the measured ion current as a function of bias voltage. The blue data points represent the experimental results, while the green line shows the exponential saturation fit.\\

\begin{figure}[h]
    \centering
    \includegraphics[width=0.9\linewidth]{Figures/results 1/bias_current_voltage_Saturation_thesis_version.png}
    \caption{Measured ion current vs. bias voltage for the Langmuir probe, showing exponential saturation fit $I=6.17(1-e^{-V/1.8})+0.023V$. Conditions: no nitrogen, 10 cm from the macroparticle filter, 0.25 T EM-coil strength.}
    \label{fig:bias_current}
\end{figure}

The relationship between the collected current $I$ and the bias voltage $V$ was analyzed using the modified Langmuir equation \cite{chen1984introduction}:

\begin{align}\label{eq:langmuir}
    I = I_{\text{sat}} \left( 1 - e^{\frac{-V}{V_0}}\right) + m \cdot V
\end{align}

where $I_{\text{sat}}$ is the saturation current and $V_0$ is a characteristic voltage. The term $k$ in the equation accounts for plasma sheath expansion and collisional effects at higher bias voltages. 

\subsection{Analysis of the Ion Saturation Curve}

As the bias voltage increases, the sheath around the probe grows, which can lead to a non-saturating component in the collected current, this is a common effect seen in small probes. Additionally, collisions within the sheath or presheath region can modify the ion trajectory, resulting in a small linear increase in the collected current with voltage. This correction ensures the model accurately describes the probe’s behavior across the full range of applied voltages \cite[Chap. 7]{chen1984introduction}.\\

The experimental data were fitted to this equation using a nonlinear least-squares method in Python, yielding the following parameters:
\[ I_{sat} = 6.17 \, \text{mA}, \hspace{2cm} V_0 = 1.8 \, \text{V}, \hspace{2cm} k = 0.023. \]

The bias voltage test results (Figure \ref{fig:bias_current}) show two distinct regimes:

\begin{enumerate}
    \item \textbf{Transition Regime (0 -- 40 V):} \\
    At low bias voltages, the probe collects both ions and electrons. As the negative bias increases, more electrons are repelled, reducing their contribution to the measured current. This results in a rapid rise in net current as the ion flux begins to dominate. The transition regime is characterized by a balance between the decreasing electron flux and the increasing ion flux.
    \item \textbf{Saturation Regime (40 -- 130V):} \\
    Beyond approximately 40V, the current plateaus, indicating that the probe has entered the ion saturation regime. At this point, the negative bias effectively repels all electrons, and the collected current is dominated by ions. However, the current increases slightly with voltage, which is captured by the linear correction term $k$ = 0.023 in the modified Langmuir equation. 
\end{enumerate}

\subsection{Selection of Operating Bias Voltage}

A bias voltage of -80 V was selected for subsequent measurements to ensure the probe operates well within the ion saturation regime. While the curve begins to saturate around ~40 V, choosing a higher voltage provides confidence that the probe is fully repelling electrons and measuring ion flux reliably.
\subsection{Ion current probe Errors}\label{sec:errorbar}

\section{Quartz crystal Microbalance and Ion current Probe}

Although the measurements were taken at the same time and location, the results will be presented in two subsections for clarity: first, the Quartz Crystal Microbalance (QCM) data, followed by the Ion Current Probe results. This structure is intended to improve the readability and understanding of the trends.\\

\begin{table}[h]
\centering
\begin{tabular}{l|l|l}
\hline
\multicolumn{1}{|l|}{Distance (cm)} & Pressure (Pa) & \multicolumn{1}{l|}{Magnetic Field (T)} \\ \hline
\multicolumn{1}{|l|}{10}            & 0             & \multicolumn{1}{l|}{0}                  \\ \hline
\multicolumn{1}{|l|}{12}            & 0.025         & \multicolumn{1}{l|}{0.05}               \\ \hline
\multicolumn{1}{|l|}{14}            & 0.05          & \multicolumn{1}{l|}{0.1}                \\ \hline
\multicolumn{1}{|l|}{16}            & 0.075         & \multicolumn{1}{l|}{0.15}               \\ \hline
\multicolumn{1}{|l|}{18}            & 0.1           & \multicolumn{1}{l|}{0.2}                \\ \hline
\multicolumn{1}{|l|}{20}            & 0.2           & \multicolumn{1}{l|}{0.25}               \\ \hline
                                    & 0.3           &                                         \\ \cline{2-2}
\end{tabular}
\caption{Parameters used for the QCM/ Ion current probe}
\end{table}


In Figures \ref{fig:3Dmass} and \ref{fig:3Dion}, a complete overview of the dataset of the measured ion currents and the deposited masses depending on three parameters are shown. These are the distance to the macroparticle filter, the magnetic field strength and the nitrogen background pressure in the chamber. Data points for 0 Pa are in reality more in the order of $1.5 \cdot 10^{-5}$ Pa.

\begin{figure}[H]
\centering

\begin{subfigure}[b]{\textwidth}
  \includegraphics[width=1\linewidth]{Figures/results 1/3Dmass.png}
  \caption{}
  \label{fig:3Dmass} 
\end{subfigure}

\begin{subfigure}[b]{\textwidth}
  \includegraphics[width=1\linewidth]{Figures/results 1/3Dion.png}
  \caption{}
  \label{fig:3Dion}
\end{subfigure}

\caption[Two numerical solutions]{%
(a) The deposited mass
(b) The ion current vs. distance, magnetic field strength, and nitrogen background pressure over 64 pulses. Conditions: nitrogen background pressure from 0 Pa to 0.3 Pa, distance from 10 cm to 20 cm from the macroparticle filter, and magnetic field strength from 0 T to 0.25 T.}
\end{figure}


General trends that can be seen already in these figure include:
\begin{itemize}
    \item The deposited mass and ion current decreases as the distance from the macroparticle filter increases (see Section \ref{subsec:distance}).
    \item Higher magnetic field strengths tend to increase the deposited mass and the ion current (see Section \ref{subsec:mag_field}).
    \item The deposited mass shows a complex dependence on nitrogen background pressure. Whereas the ion current seems to be decreasing with increasing nitrogen contents(see Section \ref{subsec:nitrogen_pressure}).
\end{itemize}


This overview sets the stage for a more detailed analysis of each variable's impact on the deposited mass and ion current. In the following subchapters, we will first examine the metallic case (no nitrogen) with respect to distance and magnetic field strength. Then, we will focus on the specific effects of distance (10 cm, 14 cm, and 20 cm), magnetic field strength (0 T, 0.15 T, and 0.25 T), and nitrogen background pressure (0.1 Pa, and 0.3 Pa) as these are the parameters, which are looked at in more detail with the help of 

\subsection{Metallic Case (No Nitrogen)}
In this section, we analyze the deposited mass as a function of distance and magnetic field strength in the absence of nitrogen.

\begin{figure}
	\centering
	\begin{subfigure}{.5\textwidth}
		\centering
		\includegraphics[width=\linewidth]{Figures/results 1/mass_vs_magfield_0Pa.png}
		\caption{}
		\label{fig:metallic_mass}
	\end{subfigure}%
	\begin{subfigure}{.5\textwidth}
		\centering
		\includegraphics[width=\linewidth]{Figures/results 1/ions_vs_magfield_0Pa.png}
		\caption{}
		\label{fig:metallic_ion}
	\end{subfigure}
	\caption{yeah dunno (a) mass and (b) ion current}
	\label{fig:test}
\end{figure}

\subsection{Distance as a variable}\label{subsec:distance}
\subsection{Magnetic field as a variable}\label{subsec:mag_field}
\subsection{Nitrogen pressure as a variable}\label{subsec:nitrogen_pressure}

