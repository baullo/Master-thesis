% !TeX root = main.tex
\chapter{Results}\label{chap:results}
\section{Langmuir Probe Bias Voltage Characterization}
\subsection{Validation of Langmuir Probe Operation}

Before plasma diagnostics were performed, the functionality of the custom-built Langmuir probe was verified by checking the characteristic voltage current curve (I-V curve) and fitting it against the theoretical relationship \ref{eq:langmuir}. Additionally the appropriate negative bias voltage for ion saturation measurements was determined . The aim of this test was to ensure that the probe operates in a regime where the collected current is dominated by ions, excluding contributions from electrons.\\

The bias voltage tests were performed without nitrogen in the chamber, at a fixed distance of 10 cm from the macroparticle filter, and with an EM coil strength of 0.25 T. Figure \ref{fig:bias_current} displays the measured ion current as a function of bias voltage. The blue data points are the experimental results, with the green line shows the exponential saturation fit.\\

\begin{figure}[h]
    \centering
    \includegraphics[width=0.9\linewidth]{Figures/results 1/bias_current_voltage_Saturation_thesis_version.png}
    \caption{Measured ion current vs. bias voltage for the Langmuir probe, showing exponential saturation fit $I=6.17\cdot (1-e^{-V/1.8})+0.023 \cdot V$. Conditions: no nitrogen, 10 cm from the macroparticle filter, 0.25 T EM-coil strength.}
    \label{fig:bias_current}
\end{figure}

The relationship between the collected current $I$ and the bias voltage $V$ was analyzed using the modified Langmuir equation \cite{chen1984introduction}:

\begin{align}\label{eq:langmuir}
    I = I_{\text{sat}} \left( 1 - e^{\frac{-V}{V_0}}\right) + m \cdot V
\end{align}

where $I_{\text{sat}}$ is the saturation current and $V_0$ is a characteristic voltage. The term $k$ in the equation accounts for plasma sheath expansion and collisional effects at higher bias voltages. 

\subsection{Analysis of the Ion Saturation Curve}

As the bias voltage increases, the sheath around the probe grows, which can lead to a non-saturating component in the collected current, this is a common effect seen in small probes. Additionally, collisions within the sheath or presheath region can modify the ion trajectory, resulting in a small linear increase in the collected current with voltage. This correction ensures the model accurately describes the probe’s behavior across the full range of applied voltages \cite[Chap. 7]{chen1984introduction}.\\

The experimental data were fitted to this equation using a nonlinear least-squares method in Python, yielding the following parameters:
\[ I_{sat} = 6.17 \, \text{mA}, \hspace{2cm} V_0 = 1.8 \, \text{V}, \hspace{2cm} k = 0.023. \]

The bias voltage test results (Figure \ref{fig:bias_current}) show two distinct regimes:

\begin{enumerate}
    \item \textbf{Transition Regime (0 -- 40 V):} \\
    At low bias voltages the probe collects both ions and electrons. As the negative bias increases, more electrons are repelled, reducing their contribution to the measured current. This results in a rapid rise in net current as the ion flux begins to dominate. The transition regime is characterized by the decreasing electron flux and the increasing ion flux.
    \item \textbf{Saturation Regime (40 -- 130V):} \\
    Beyond approximately 40V, the current plateaus, indicating that the probe has entered the ion saturation regime. At this point, the negative bias effectively repels all electrons, and the collected current is dominated by ions. However, the current increases slightly with voltage, which is captured by the linear correction term $k$ = 0.023 in the modified Langmuir equation. 
\end{enumerate}

\subsection{Selection of Operating Bias Voltage}

A bias voltage of -80 V was selected for subsequent measurements to ensure the probe operates well within the ion saturation regime. While the curve begins to saturate around ~40 V, choosing a higher voltage provides confidence that the probe is fully repelling electrons and measuring ion flux reliably.
\subsection{Ion current Variation over different pulses}\label{sec:errorbar}

To analyze the pulse-to-pulse variation, approximately 30 single pulses were recorded for each magnetic field strength. The average ion current of each pulse was calculated over the 0--1~ms pulse interval. The final mean and standard deviation were then determined by combining these averages, allowing for an assessment of the differences between pulses. Notably, the 0.05~T data point in Fig.~\ref{fig:pulsevariabilityplotmean} includes a significant outlier, with a recorded ion current of approximately 4.8~mA, as detailed in Table~\ref{tab:pulse_measurements}.

\begin{figure}[h]
	\centering
	\includegraphics[width=0.75\textwidth]{"Figures/results 1/pulse_variability_plot_mean"}
	\caption{Pulse-to-Pulse Variation in Mean Ion Current at Varying Magnetic Fields (0.1 Pa N$_2$, 10 cm distance)}
	\label{fig:pulsevariabilityplotmean}
\end{figure}

\begin{table}[h]
	\centering
	\begin{tabularx}{\textwidth}{|X|X|X|X|X|X|}
		\hline
		\textbf{EM-coil field (T)} & \textbf{Mean $I_{\text{ion}}$ (mA)} & \textbf{Std (mA)} & \textbf{Std (\%)} & \textbf{Range (mA)} & \textbf{\# Pulses} \\ \hline
		0.00 & 1.931 & 0.266 & 13.8 & [1.13, 2.26] & 30 \\ \hline
		0.05 & 1.622 & 0.602 & 37.1 & [1.16, 4.85] & 32 \\ \hline
		0.10 & 4.279 & 0.542 & 12.7 & [2.72, 4.88] & 31 \\ \hline
		0.15 & 6.323 & 0.414 & 6.6 & [4.99, 6.92] & 27 \\ \hline
		0.20 & 7.996 & 0.399 & 5.0 & [7.26, 8.78] & 22 \\ \hline
		0.25 & 9.460 & 0.363 & 3.8 & [8.61, 10.02] & 30 \\ \hline
	\end{tabularx}
	\caption{Summary of pulse measurement statistics for varying magnetic field strengths.}
	\label{tab:pulse_measurements}
\end{table}

In all other measurements, the oscilloscope averaged the voltage drop of the ion current over multiple pulses to produce smoother curves. Therefore, the error reported in subsequent sections primarily reflects variations in ion current within individual pulses, rather than differences between distinct pulses, as previously discussed in Section~\ref{sec:ion/qcm_error}.

\section{Quartz crystal Microbalance and Ion current Probe}

Results will be presented in two subsections for clarity: first, the Quartz Crystal Microbalance (QCM) data, followed by the Ion Current Probe results. This structure is intended to improve the readability and understanding of the trends. At this juncture the quantities of Mass (ng) and Ion current (mA) are not comparable. Overall every permutation of the three parameters in the Table \ref{tab:parameters} were recorded.\\

\begin{table}[h]
	\centering
	\begin{tabularx}{0.85\textwidth}{|X|X|X|}
		\hline
		\textbf{Distance (cm)} & \textbf{Pressure (Pa)} & \textbf{Magnetic Field (T)} \\ \hline
		10 & 0 & 0 \\ \hline
		12 & 0.025 & 0.05 \\ \hline
		14 & 0.05 & 0.1 \\ \hline
		16 & 0.075 & 0.15 \\ \hline
		18 & 0.1 & 0.2 \\ \hline
		20 & 0.2 & 0.25 \\ \hline
		& 0.3 & \\ \hline
	\end{tabularx}
	\caption{Parameters used for the QCM/Ion current probe}
	\label{tab:parameters}
\end{table}

In Figures \ref{fig:3Dmass} and \ref{fig:3Dion}, a complete overview of the dataset of the deposited masses and the measured ion currents depending on three parameters are shown. These are the distance to the macroparticle filter, the magnetic field strength and the nitrogen background pressure in the chamber. Data points for 0 Pa are in reality more in the order of $1.5 \cdot 10^{-5}$ Pa.

\begin{figure}[H]
\centering

\begin{subfigure}[b]{\textwidth}
  \includegraphics[width=1\linewidth]{Figures/results 1/3Dmass.png}
  \caption{}
  \label{fig:3Dmass} 
\end{subfigure}

\begin{subfigure}[b]{\textwidth}
  \includegraphics[width=1\linewidth]{Figures/results 1/3Dion.png}
  \caption{}
  \label{fig:3Dion}
\end{subfigure}

\caption[Two numerical solutions]{%
(a) The deposited mass
(b) The ion current vs. distance, magnetic field strength, and nitrogen background pressure over 64 pulses. Conditions: nitrogen background pressure from 0 Pa to 0.3 Pa, distance from 10 cm to 20 cm from the macroparticle filter, and magnetic field strength from 0 T to 0.25 T.}
\end{figure}

\textit{Note}: The mass data presented here are based on 64 pulses. This number was chosen because it provides a sufficient mass difference even in the weakest deposition conditions, and it also reduces pulse-to-pulse variability in both the QCM and ion-current probe measurements.\\


General trends that can be seen already in these figure include:\\

\begin{compactitem}
    \item The deposited mass and ion current decreases as the distance from the macroparticle filter increases (see Section \ref{subsec:distance}).
    \item Higher magnetic field strengths tend to increase the deposited mass and the ion current (see Section \ref{subsec:mag_field}).
    \item The deposited mass shows a complex dependence on nitrogen background pressure. Whereas the ion current seems to be decreasing with increasing nitrogen contents(see Section \ref{subsec:nitrogen_pressure}).
\end{compactitem}
\vspace{11pt}

This overview sets the stage for a more detailed analysis of each variable's impact on the deposited mass and ion current. In the following subchapters, we will first examine the metallic case (no nitrogen) with respect to distance and magnetic field strength. Then, we will focus on the specific effects of distance (10 cm, 14 cm and 20 cm), magnetic field strength (0 T, 0.15 T and 0.25 T), and nitrogen background pressure (0.1 Pa and 0.3 Pa) as these are the parameters.\\

The above stated parameters represent a subset, which as can be seen in Table \ref{tab:comprehensive_measurements} include the parameters used in further investigations with ERMS and with deposited films.

\subsection{Metallic Case (No Nitrogen)}\label{subsec:metal}
In this section, we analyze the deposited mass as a function of distance and magnetic field strength in the absence of nitrogen.

\begin{figure}[h]
	\centering
	\begin{subfigure}{.5\textwidth}
		\centering
		\includegraphics[width=\linewidth]{Figures/results 1/MagField_vs_Distance_P0.0Pa_Mass.png}
		\caption{}
	\end{subfigure}%
	\begin{subfigure}{.5\textwidth}
		\centering
		\includegraphics[width=\linewidth]{Figures/results 1/MagField_vs_Distance_P0.0Pa_IonCurrent.png}
		\caption{}
	\end{subfigure}
	\caption{Metallic case measurements showing (a) the deposited mass after 64 pulses and (b) the ion current averaged over a single pulse, each plotted as a function of magnetic field strength. Data selected for three distances (10, 14, 20 cm), with error bars representing variation within pulses.}
	\label{fig:metallic_mass/ion}
\end{figure}

As shown in Figure \ref{fig:metallic_mass/ion}, both the deposited mass (after 64 pulses) and the pulse-averaged ion current increase with applied magnetic field. The enhancement is strongest at the shortest distance, indicating that the magnetic field improves plasma focusing and increases the fraction of energetic, highly ionized species reaching the substrate.\\

At 0.05 T both the ion current and deposited mass fall below the values measured at 0 T. This unexpected drop is likely due to a magnetic mirror effect happening at the coil entrance. As plasma moves from a weak magnetic field into a stronger, the conservation of the particles’ magnetic moment causes electrons (which have a lot of perpendicular gyroenergy) to either bounce back or speed up along the field’s gradient. This creates localized space-charge and potential structures that can cut down the overall ion flow further downstream. As a result, you might see a lower ion current and less deposited mass at moderate field strengths, before the effects of focusing and ionization take over at higher fields \cite[Chap. 7.6]{cathodic_arcs} \cite{magnetic_mirror}. This effect is noticeable, in different intensities, in all measurements.

\subsection{Distance as a variable}\label{subsec:distance}

In this section, we look at another slice of our dataset looking at the change in mass and ion current vs distance, while taking multiple representable nitrogen pressures (0.1, 0.2, 0.3 Pa) and the metallic case into account at a magnetic field strength of 0.25T. 

\begin{figure}[ht]
	\centering
	\begin{subfigure}{.5\textwidth}
		\centering
		\includegraphics[width=\linewidth]{Figures/results 1/Distance_vs_Pressure_B0.25T_Mass.png}
		\caption{}
	\end{subfigure}%
	\begin{subfigure}{.5\textwidth}
		\centering
		\includegraphics[width=\linewidth]{Figures/results 1/Distance_vs_Pressure_B0.25T_IonCurrent.png}
		\caption{}
	\end{subfigure}
	\caption{QCM and Ion Probe measurements showing (a) the deposited mass after 64 pulses and (b) the ion current averaged over a single pulse, each plotted as a function of distance. Data shown for representative pressures, with error bars representing variation within pulses.}
	\label{fig:distance_0.25T_mass/ion}
\end{figure}

The deposited mass and the ion current both decrease with increasing distance, consistent with the expansion of the plasma. The mass shows a decline with higher chamber pressure resulting in a reduced deposition. In contrast, the ion current exhibits a much steeper drop at short distances, less than 16cm, when the nitrogen content in the chamber is increased.\\

This significant change is only present when enhancing the plasma with a magnetic field. In the case like in Fig. \ref{fig:distance_0T_mass/ion}, the drop off between metallic and reactive depositions for the ion current is less pronounced, with the ion current being seemingly the same for all pressures at a given distance.\\

\textcolor{red}{This is in contrast to the measurements done in kalanov paper where reactive had a higher yield, while I get seemingly less or the same mass}


\subsection{Magnetic Field as a variable}\label{subsec:mag_field}

\textcolor{red}{here for 14 cm the result is contradictory to the previous statement where added nitrogen leads to higher deposited mass.. funny enough this is the only distance where this is the case as can be seen in fig. \ref{fig:mag_field_14cm_mass/ion}}

The next slice of the dataset we take a closer look at is the magnetic field as a variable like in sec. \ref{subsec:metal}, with the difference being that here the distance is fixed at 14 cm and the nitrogen pressures (0.1, 0.2, 0.3 Pa) and the metallic case are plotted.

\begin{figure}[ht]
	\centering
	\begin{subfigure}{.5\textwidth}
		\centering
		\includegraphics[width=\linewidth]{Figures/results 1/MagField_vs_Pressure_D14cm_Mass.png}
		\caption{}
	\end{subfigure}%
	\begin{subfigure}{.5\textwidth}
		\centering
		\includegraphics[width=\linewidth]{Figures/results 1/MagField_vs_Pressure_D14cm_IonCurrent.png}
		\caption{}
	\end{subfigure}
	\caption{QCM and Ion Probe measurements showing (a) the deposited mass after 64 pulses and (b) the ion current averaged over a single pulse, each plotted as a function of Magnetic Field. Data shown for representative pressures, with error bars representing variation within pulses.}
	\label{fig:mag_field_14cm_mass/ion}
\end{figure}

In Fig. \ref{fig:mag_field_14cm_mass/ion}, the magnetic-mirror effect is still present, but weaker than in the 10 cm case. At 14 cm, both the deposited mass and the ion current remain essentially constant, whereas at 10 cm a clear decrease was observed, as discussed in subsec. \ref{subsec:metal}. The trends observed here suggest that additional nitrogen pressure atmosphere has the smallest influence on the deposited mass compared to distance or the magnetic field. Only for the ion current does the metallic configuration exhibit a slight increase at higher magnetic-field strengths compared to the reactive cases.

\subsection{Nitrogen pressure as a variable}\label{subsec:nitrogen_pressure}

\begin{figure}[ht]
	\centering
	\begin{subfigure}{.5\textwidth}
		\centering
		\includegraphics[width=\linewidth]{Figures/results 1/Pressure_vs_Distance_B0.25T_Mass.png}
		\caption{}
	\end{subfigure}%
	\begin{subfigure}{.5\textwidth}
		\centering
		\includegraphics[width=\linewidth]{Figures/results 1/Pressure_vs_Distance_B0.25T_IonCurrent.png}
		\caption{}
	\end{subfigure}
	\caption{QCM and Ion Probe measurements showing (a) the deposited mass after 64 pulses and (b) the ion current averaged over a single pulse, each plotted as a function of Pressure. Data shown for representative pressures, with error bars representing variation within pulses.}
	\label{fig:pressure_0.25T_mass/ion}
\end{figure}

\section{Mass spectroscopy Results}
\begin{itemize}
	\item shortly about mean charge state and energies not a focus tho
	\item maybe a table and a picture
\end{itemize}


\begin{figure}
	\centering
	\includegraphics[width=\linewidth]{"Figures/results 2/Distance_0.25T_0.0Pa"}
	\caption{0Pa 0.25T}
	\label{fig:ERMS_distance0Pa}
\end{figure}
\begin{figure}
	\centering
	\includegraphics[width=\linewidth]{"Figures/results 2/Distance_0.25T_0.3Pa"}
	\caption{0.3Pa 0.25T}
	\label{fig:ERMS_distance0.3Pa}
\end{figure}
\begin{figure}
	\centering
	\includegraphics[width=\linewidth]{"Figures/results 2/Pressure_10cm_0.25T"}
	\caption{10cm 0.25T}
	\label{fig:ERMS_pressure10cm0}
\end{figure}
Missing 14cm 0.25T

\newpage
\section{Ex situ Results}
\begin{figure}[ht]
	\centering
	\begin{subfigure}{.5\textwidth}
		\centering
		\includegraphics[width=\linewidth]{Figures/results 2/PXL_20251114_084343196.jpg}
		\caption{}
	\end{subfigure}%
	\begin{subfigure}{.5\textwidth}
		\centering
		\includegraphics[width=\linewidth]{Figures/results 2/PXL_20251117_130844177.jpg}
		\caption{}
	\end{subfigure}
	\caption{\textcolor{red}{to be written picture taken with phone a) with nitrogen and b) metallic... colors are wrong need to change it haha}}
	\label{fig:plasma}
\end{figure}

A total of [insert number of films once i ve done all of them, proabably like 6-8] were taking into account for further measurements and looked at in the following sections. Blablabla
\subsection{Profilometry}

Samples were prepared by placing a straight marker line near the edge of each substrate prior to deposition. This marker could be cleanly removed by ultrasonic washing after film growth, revealing the deposition step for thickness analysis. Profilometry measurements were performed at three positions on each sample, one near the center and one near each edge. The three measurements were averaged to obtain the mean film thickness. All resulting values are summarized in Table \ref{tab:profilometry_results}.

\begin{table}[h]
	\centering
	\begin{tabularx}{0.85\textwidth}{|X|X|X|}
		\hline
	\end{tabularx}
	\caption{Profilometry thickness measurements for the deposited films}
	\label{tab:profilometry_results}
\end{table}

\subsection{XRD}
\subsection{XRR}
\subsection{EDX}


\section{Fluxes}\label{sec:flux}
