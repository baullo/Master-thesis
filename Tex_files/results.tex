% !TeX root = main.tex
\chapter{Results}\label{chap:results}

\section{Mass Deposition Rate and Ion Current Measurements}

To systematically characterize the plasma dynamics and deposition behavior, measurements were conducted using a quartz crystal microbalance (QCM) for mass deposition rate and a biased ion collector probe for ion current density. 

The experimental parameter space spanned three variables: distance from the macroparticle filter (10--20~cm), applied magnetic field strength (0--0.25~T), and nitrogen background pressure (0--0.3~Pa), with all permutations measured as listed in Table~\ref{tab:comprehensive_measurements}. At this stage, the raw quantities mass deposition rate (in ng~cm$^{-2}$~s$^{-1}$) and ion current (in mA) are presented to establish trends before flux calculations in Section~\ref{sec:flux_calculations}.

Figure~\ref{fig:3D overview} provides a three-dimensional visualization of the complete dataset, illustrating how mass deposition rate and ion current vary simultaneously with all three control parameters. Several global trends are immediately apparent: both quantities decrease with increasing distance due to plasma expansion, increase with applied magnetic field due to enhanced plasma confinement and ionization, and exhibit complex pressure dependence that warrants detailed investigation.


\begin{figure}[H]
\centering
\begin{subfigure}[b]{\textwidth}
  \includegraphics[width=1\linewidth]{Figures/results 1/3Dmass.png}
  \caption{}
  \label{fig:3Dmass} 
\end{subfigure}

\begin{subfigure}[b]{\textwidth}
  \includegraphics[width=1\linewidth]{Figures/results 1/3Dion.png}
  \caption{}
  \label{fig:3Dion}
\end{subfigure}
\caption{Three-dimensional visualization of (a) mass deposition rate (ng~cm$^{-2}$~s$^{-1}$) and (b) ion current (mA) as functions of distance (10--20~cm), magnetic field strength (0--0.25~T), and nitrogen pressure (0--0.3~Pa). Measurements performed over 64 arc pulses at 5~Hz repetition rate.}
\label{fig:3D overview}
\end{figure}

To systematically dissect these multidimensional trends, the following subsections examine cross-sections through the parameter space in order of increasing complexity. We begin with the metallic case (no nitrogen, Section~\ref{subsec:metallic}) to establish baseline behavior without reactive gas complications. Sections~\ref{subsec:distance}--\ref{subsec:nitrogen_pressure} then isolate the effects of distance, magnetic field, and nitrogen pressure. The parameter combinations analyzed in detail---distances of 10, 14, and 20~cm; magnetic fields of 0, 0.15, and 0.25~T; and nitrogen pressures of 0.1 and 0.3~Pa---were selected because they correspond to conditions used for energy-resolved mass spectrometry (ERMS) measurements and thin film deposition, enabling direct correlation between plasma diagnostics and film properties in later sections.

\subsection{Metallic Case (No Nitrogen)}\label{subsec:metallic}
In the absence of reactive gas, plasma expansion and deposition dynamics are governed solely by geometric dilution and magnetic confinement. Figure~\ref{fig:metallic_mass/ion} presents mass deposition rate and ion current as functions of magnetic field strength for three representative distances.

\begin{figure}[h]
	\centering
	\begin{subfigure}{.5\textwidth}
		\centering
		\includegraphics[width=\linewidth]{Figures/results 1/MagField_vs_Distance_P0.0Pa_Mass.png}
		\caption{}
	\end{subfigure}%
	\begin{subfigure}{.5\textwidth}
		\centering
		\includegraphics[width=\linewidth]{Figures/results 1/MagField_vs_Distance_P0.0Pa_IonCurrent.png}
		\caption{}
	\end{subfigure}
	\caption{Metallic case measurements showing (a) mass flux and (b) the ion current, each plotted as a function of magnetic field strength. Data selected for three distances (10, 14, 20 cm), with error bars representing variation within pulses.}
	\label{fig:metallic_mass/ion}
\end{figure}

As shown in Figure~\ref{fig:metallic_mass/ion}, both mass deposition rate and ion current increase with applied magnetic field. The enhancement is most pronounced at the shortest distance (10~cm), where the magnetic field can effectively guide ions before significant radial expansion occurs. At 20~cm, the plasma has already expanded substantially, reducing the relative impact of magnetic confinement on the collected flux.

A notable anomaly appears at 0.05~T, where both quantities drop below their zero-field values before recovering and increasing at higher fields (0.1~T and above). Similar behavior is observed across all experimental conditions involving increasing magnetic field strength.

\subsection{Distance as a variable}\label{subsec:distance}

Figure~\ref{fig:distance_0.25T_mass/ion} examines the effect of source-to-substrate distance under fixed magnetic confinement (0.25~T) for both metallic and reactive conditions. Four nitrogen pressures are compared: 0~Pa (metallic), 0.1~Pa, 0.2~Pa, and 0.3~Pa.

\begin{figure}[ht]
	\centering
	\begin{subfigure}{.5\textwidth}
		\centering
		\includegraphics[width=\linewidth]{Figures/results 1/Distance_vs_Pressure_B0.25T_Mass.png}
		\caption{}
	\end{subfigure}%
	\begin{subfigure}{.5\textwidth}
		\centering
		\includegraphics[width=\linewidth]{Figures/results 1/Distance_vs_Pressure_B0.25T_IonCurrent.png}
		\caption{}
	\end{subfigure}
	\caption{Mass deposition rate and ion current as functions of distance at 0.25~T magnetic field for four nitrogen pressures. (a) Mass deposition rate shows moderate pressure dependence. (b) Ion current exhibits much stronger pressure dependence, particularly at short distances. Error bars represent intra-pulse variability.}
	\label{fig:distance_0.25T_mass/ion}
\end{figure}


Both mass deposition rate and ion current decrease with distance, reflecting plasma expansion and dilution. In metallic mode (0~Pa), mass deposition rate follows an approximate $1/r^2$ dependence expected for free expansion \cite[Chap.~4.3]{cathodic_arcs}, with 20~cm values approximately 25--30\% of those measured at 10~cm.\\

Ion current exhibits a far more dramatic pressure dependence than mass deposition rate. When nitrogen is introduced, the measured current drops steeply, while the QCM mass deposition rate shows more modest changes. The contrast between the two measurements is most pronounced at short distances and diminishes at 14--20~cm.\\

Additional measurements at constant magnetic field (0~T) and varying distance show similar trends and are presented in Appendix~\ref{fig:distance_0T_mass/ion}.

\subsection{Magnetic Field as a variable}\label{subsec:mag_field}

Figure~\ref{fig:mag_field_14cm_mass/ion} presents mass flux and ion current as functions of magnetic field strength at a fixed intermediate distance (14~cm) for the same set of pressures.

\begin{figure}[ht]
	\centering
	\begin{subfigure}{.5\textwidth}
		\centering
		\includegraphics[width=\linewidth]{Figures/results 1/MagField_vs_Pressure_D14cm_Mass.png}
		\caption{}
	\end{subfigure}%
	\begin{subfigure}{.5\textwidth}
		\centering
		\includegraphics[width=\linewidth]{Figures/results 1/MagField_vs_Pressure_D14cm_IonCurrent.png}
		\caption{}
	\end{subfigure}
	\caption{Mass deposition rate and ion current as functions of magnetic field strength at 14~cm distance for four nitrogen pressures. (a) Mass deposition rate increases monotonically above 0.1~T. (b) Ion current shows pronounced divergence between metallic and reactive conditions at high fields. Error bars represent intra-pulse variability.}
	\label{fig:mag_field_14cm_mass/ion}
\end{figure}

The magnetic mirror anomaly at 0.05~T, clearly visible in the 10~cm metallic data (Figure~\ref{fig:metallic_mass/ion}), is significantly attenuated at 14~cm. Both mass deposition rate and ion current remain approximately constant between 0 and 0.05~T, suggesting that the adverse mirror effect is either weaker after 4~cm of additional expansion or is masked by increased statistical noise at this intermediate distance.\\

Above 0.1~T, both quantities show a clear, consistent increase for all conditions, confirming that magnetic confinement enhances plasma density and flux at the substrate. The enhancement is approximately linear with field strength in the 0.1--0.25~T range. Mass deposition rate (Figure~\ref{fig:mag_field_14cm_mass/ion}a) increases by a factor of approximately 5--6$\times$ from zero field to 0.25~T, with all pressure conditions following similar trajectories. This similarity indicates that the magnetic field effect on total mass flux is largely independent of nitrogen pressure at this distance.\\

Ion current (Figure~\ref{fig:mag_field_14cm_mass/ion}b) also increases with magnetic field, but exhibits greater variability, particularly in reactive mode. At 0.25~T, the ion current in metallic mode (0~Pa) reaches approximately 5~mA, while reactive mode conditions (0.1--0.3~Pa show currents in the range of 4~mA with substantial error bars. The overlap between metallic and reactive conditions at high fields, combined with the large variation of ion current during a pulse, makes it difficult to draw definitive conclusions about pressure effects on ion current at 14~cm distance. \\

The similar behavior of mass deposition rate across all pressures, combined with the more variable ion current measurements, suggests that at 14~cm the magnetic field primarily affects the total particle flux rather than selectively enhancing ionization. The reduced contrast between metallic and reactive mode compared to the 10~cm measurements (Section~\ref{subsec:distance}) is consistent with the plasma having undergone substantial expansion and equilibration by this distance.\\

To isolate the effect of nitrogen pressure more clearly, the next section examines pressure as the primary variable under fixed magnetic field conditions.

\subsection{Nitrogen pressure as a variable}\label{subsec:nitrogen_pressure}
Figure~\ref{fig:pressure_0.25T_mass/ion} presents mass flux and ion current as functions of nitrogen pressure at maximum magnetic confinement (0.25~T) for three distances (10, 14, and 20~cm).

\begin{figure}[ht]
	\centering
	\begin{subfigure}{.5\textwidth}
		\centering
		\includegraphics[width=\linewidth]{Figures/results 1/Pressure_vs_Distance_B0.25T_Mass.png}
		\caption{}
	\end{subfigure}%
	\begin{subfigure}{.5\textwidth}
		\centering
		\includegraphics[width=\linewidth]{Figures/results 1/Pressure_vs_Distance_B0.25T_IonCurrent.png}
		\caption{}
	\end{subfigure}
	\caption{QCM and Ion Probe measurements showing (a) mass flux and (b) the ion current, each plotted as a function of Pressure. Data shown for representative pressures, with error bars representing variation within pulses.}
	\label{fig:pressure_0.25T_mass/ion}
\end{figure}

At 10~cm, the mass deposition rate exhibits non-monotonic pressure dependence, with a local maximum near 0.05--0.1~Pa before decreasing at higher pressures. The ion current shows substantial scatter across the low pressure ranges this is an artefact from the experimental setup. A general decreasing trend is evident despite the variability.\\


At 14~cm and 20~cm, both mass deposition rate and ion current become largely independent of nitrogen pressure. Within experimental uncertainty, both quantities remain essentially constant across the 0--0.3~Pa range investigated.\\

The measurements presented in this section establish the macroscopic trends in mass deposition rate and ion current across the experimental parameter space. However, these raw quantities do not directly reveal the underlying plasma composition. To address these questions, energy-resolved mass spectrometry was performed under selected conditions to characterize the ion energy distributions and mean charge states, as presented in the following section.


\section{Energy-Resolved Mass Spectrometry Results}\label{sec:erms_results}

Energy-resolved mass spectrometry (ERMS) measurements were performed to characterize charge-state distributions and ion energies under selected conditions corresponding to film depositions. The measurements provide charge-state-resolved ion energy distribution functions (IEDFs) for Al$^{1+,2+,3+}$, Ti$^{1+,2+,3+,4+}$ and N/N$_2^{1+}$, from which mean charge states $\langle Q \rangle$, kinetic energies $E_{\rm kin}$, and potential energies $E_{\rm pot}$ are derived. All measurements were performed at 0.25~T magnetic field strength, with particular emphasis on the 14~cm distance selected for systematic film depositions, as well as for all the above mentioned species, even if nitrogen should not be present in metallic plasmas.  

\subsection{Distance Dependence}\label{subsec:erms_distance}

Figures~\ref{fig:ERMS_distance0Pa} and \ref{fig:ERMS_distance0.3Pa} present mean charge states, potential energies, and kinetic energies as functions of distance for metallic mode (0~Pa) and reactive mode (0.3~Pa N$_2$), respectively.

\begin{figure}[h]
	\centering
	\includegraphics[width=\linewidth]{"Figures/results 2/Distance_0.25T_0.0Pa"}
	\caption{Mean charge state, potential energy, and kinetic energy as functions of distance in metallic mode (0~Pa, 0.25~T). Al and Ti show mean charge states of $\langle Q \rangle \approx 2.0$ and 2.7 respectively, with minimal variation across 10--20~cm. Total ion energies remain above 40~eV at all distances.}
	\label{fig:ERMS_distance0Pa}
\end{figure}


In metallic mode (Figure~\ref{fig:ERMS_distance0Pa}), titanium ions exhibit $\langle Q \rangle \approx 2.6$ and aluminum ions $\langle Q \rangle \approx 2.0$ across all distances. These charge states reflect the high electron temperatures ($T_e \approx 5$--10~eV) characteristic of cathode spot plasmas. The mean charge states remain remarkably constant with distance, varying by less than 10\% between 10 and 20~cm, indicating that recombination is negligible during the transit time through this region on the order of $\sim$100~$\mu$s scale.\\


Potential energies scale directly with charge state, with Ti showing $E_{\rm pot} \approx 38$~eV and Al $E_{\rm pot} \approx 26$~eV, remaining nearly constant across the measurement range. Kinetic energies are in the range of 20--30~eV for both species, with a slight decreasing trend with distance due to minor collisional thermalization. Importantly, total ion energies ($E_{\rm kin} + E_{\rm pot}$) exceed 40~eV at all distances, well above the $\approx$30~eV threshold for subplantation-driven densification \cite[Chap.~8.1]{cathodic_arcs}. This confirms that energetic condensation should remains accessible across the entire 10--20~cm range without substrate heating.\\

\begin{figure}[h]
	\centering
	\includegraphics[width=\linewidth]{"Figures/results 2/Distance_0.25T_0.3Pa"}
	\caption{Mean charge state, potential energy, and kinetic energy as functions of distance in reactive mode (0.3~Pa N$_2$, 0.25~T). Metal ion charge states decrease by 20--30\% compared to metallic mode due to charge-exchange collisions. Kinetic energies show increased scatter due to collisional thermalization.}
	\label{fig:ERMS_distance0.3Pa}
\end{figure}

Introducing nitrogen at 0.3~Pa substantially alters the plasma composition (Figure~\ref{fig:ERMS_distance0.3Pa}). Metal ion charge states decrease by up to 50\%, coinciding with charge-exchange conditions where multiply charged metal ions interact with nitrogen molecules \cite{bendikt2012}:
\begin{equation*}
	\text{Ti}^{3+} + \text{N}_2 \rightarrow \text{Ti}^{2+} + \text{N}_2^{+} \quad \text{(or lower charge states)}
\end{equation*}

Potential energies decrease proportionally with the reduced charge states, falling from $\approx$35--40~eV (metallic) to $\approx$20--30~eV in reactive mode. Kinetic energies show a slight increase in reactive mode. Despite these changes, total ion energies remain in the range of 30~eV at 10~cm and 35~eV at 20~cm. At the 14~cm deposition distance, energies are approximately 25--35~eV.\\

\subsection{Pressure Dependence}\label{subsec:erms_pressure}

Figures~\ref{fig:ERMS_10cm_pressure} and \ref{fig:ERMS_14cm_pressure} present the pressure dependence of charge states and ion energies at 10~cm and 14~cm, respectively. The 14~cm measurements are emphasized as this distance was selected for film depositions.

\begin{figure}[h]
	\centering
	\includegraphics[width=\linewidth]{"Figures/results 2/Pressure_10cm_0.25T"}
	\caption{Mean charge state, potential energy, and kinetic energy as functions of nitrogen pressure at 10~cm distance (0.25~T). Only metal ion (Al, Ti) charge states are shown. Metal ion charge states decrease systematically with increasing pressure due to charge-exchange collisions. Kinetic energies exhibit non-monotonic behavior, with a local maximum near 0.1~Pa.}
	\label{fig:ERMS_10cm_pressure}
\end{figure}

\begin{figure}[h]
	\centering
	\includegraphics[width=\linewidth]{"Figures/results 2/Pressure_14cm_0.25T"}
	\caption{Mean charge state, potential energy, and kinetic energy as functions of nitrogen pressure at 14~cm distance (0.25~T). Only metal ion (Al, Ti) charge states are shown. Trends are similar to 10~cm but with reduced absolute values due to plasma expansion. This distance was selected for film depositions (Section~\ref{sec:film_char}).}
	\label{fig:ERMS_14cm_pressure}
\end{figure}

At both distances, metal ion charge states generally decrease with increasing nitrogen pressure as charge-exchange collisions accumulate. At 14~cm, Ti decreases from $\langle Q \rangle \approx 2.7$ (0~Pa) to $\approx 1.7$ (0.3~Pa), and Al from $\approx 1.9$ to $\approx 1.3$. This trend is non-monotonic, with a slight increase observed between 0.2 and 0.3~Pa at 14~cm. This deviation from systematic reduction suggests that charge-exchange collisions are not the sole process affecting charge-state distributions. The overall charge-state reduction is qualitatively similar at both distances.\\

Kinetic energies show more complex behavior. At both 10~cm and 14~cm, Ti and Al exhibit non-monotonic trends with a local maximum near 0.1~Pa. This behavior may reflect competing effects: at low pressures, nitrogen collisions reduce forward momentum through elastic scattering, while at intermediate pressures, reactive processes at the cathode surface (type-1 vs.\ type-2 spot transitions \cite[Chap.~9.3]{cathodic_arcs}) may alter the initial ion velocity distribution. The kinetic energy measurements also show increased scatter at higher pressures, consistent with collisional thermalization. Despite this complexity, the general trend toward slightly reduced kinetic energy with increasing nitrogen pressure is evident at both distances.\\

The similarity in pressure-dependent trends between 10~cm and 14~cm and especially the increased kinetic energy around 0.1~Pa nitrogen pressure, makes these configurations especially interesting. 

\newpage

\section{Thin Film Characterization}\label{sec:film_char}

Thin films were deposited on Si(100) substrates positioned at 14~cm from the macroparticle filter exit under conditions listed in Table~\ref{tab:comprehensive_measurements}. The substrates were mounted on the substrate holder with the same positions in space used for QCM and ion probe measurements to ensure identical positioning. Films were deposited under four representative conditions spanning the parameter space: metallic mode (0~Pa) and reactive mode (0.1, 0.2, 0.3~Pa) and at high magnetic field strength (0.25~T). All depositions used 850~A arc current, 1~ms pulse width, and 5~Hz repetition rate.

\begin{figure}[ht]
	\centering
	\begin{subfigure}{.5\textwidth}
		\centering
		\includegraphics[width=\linewidth]{Figures/results 2/PXL_20251114_084343196.jpg}
		\caption{Reactive mode (0.3\,Pa N$_2$)}
	\end{subfigure}%
	\begin{subfigure}{.5\textwidth}
		\centering
		\includegraphics[width=\linewidth]{Figures/results 2/PXL_20251117_130844177.jpg}
		\caption{Metallic mode (vacuum)}
	\end{subfigure}
	\caption{Visual comparison of cathodic arc plasma plumes exiting the macroparticle filter. (a) Reactive mode operation with 0.3\,Pa N$_2$ background pressure shows broader, more diffuse plasma emission with characteristic orange-red glow from nitrogen excitation. (b) Metallic mode in vacuum shows tighter confinement of the ionized metal plasma plume with blue-white emission characteristic of highly ionized Ti and Al species. (Camera settings: f/1.7, 1/100\,s, ISO 90.)}
	\label{fig:plasma}
\end{figure}

\subsection{Profilometry}\label{sec:profilometry}

Samples were prepared by placing a straight marker line near the edge of each substrate prior to deposition. This marker could be cleanly removed by ultrasonic washing after film growth, revealing the deposition step for thickness analysis. Profilometry measurements were performed at three positions on each sample: one near the center and one near each edge. The three measurements were averaged to obtain the mean film thickness.\\

To compensate for the reduced deposition rate observed in reactive mode, the number of pulses for nitrogen-containing depositions was increased from 6000 (metallic mode) to 8000 pulses. This adjustment aimed to achieve film thicknesses sufficient for reliable characterization while maintaining comparable total deposition times. All resulting thickness values are summarized in Table~\ref{tab:profilometry_results}.


\begin{table}[h]
	\centering
	\caption{Profilometry thickness measurements for the deposited films at the two positions where the QCM and Ion probe were before}
	\label{tab:profilometry_results}
	\begin{tabularx}{\textwidth}{|c|c|c|c|X|X|}
		\hline
	Film ID & Distance (cm) & Pressure (Pa) & Pulses & Thickness QCM (nm) & Thickness Ion probe (nm) \\ \hline
	8		& 14					& 	0	&	6000	 &   73             & 68 				     \\ \hline
	9		& 14				 & 	0.3		&	6000	 &  42             & 35				 \\ \hline
	10	&	14                     & 0       &  6000       & 65            & 57                \\ \hline
	11	&	14                     & 0.3     &  8000      & 60              & 42                \\ \hline
	13	&	14                     & 0.1     &  8000      & 59             & 41            		 \\ \hline
	14	&	14                     & 0.2      & 8000      & 46             & 38                \\ \hline
	\end{tabularx}	
\end{table}

The measured thicknesses show good agreement between QCM-derived values and profilometry, with differences on the order of 10~\%, which is within the combined measurement uncertainties of both techniques. Films deposited in metallic mode (films 8 and 10, 6000 pulses) show thicknesses of 57--68~nm by profilometry, while films deposited in reactive mode show lower values despite the increased pulse count to 8000. This reduction reflects the lower mass deposition rate in reactive mode observed in Figure~\ref{fig:pressure_0.25T_mass/ion}a, which is not fully compensated by the increased pulse count.\\

Several competing effects influence the thickness trends. Films 8 and 10 (metallic mode, 6000 pulses) show thicknesses of 65--73~nm by profilometry, while films deposited in reactive mode show lower values despite the increased pulse count. Film 11 (0.3~Pa, 8000 pulses) achieved 60~nm. This reduction reflects the lower mass deposition rate in reactive mode (Figure~\ref{fig:pressure_0.25T_mass/ion}a), which is not fully compensated by the increased pulse count.\\

An apparent inconsistency emerges when comparing films 9 (0.3~Pa, 6000 pulses, 35~nm) and 11 (0.3~Pa, 8000 pulses, 42~nm). The thickness increase from 35 to 42~nm represents only a 20~\% gain for a 33~\% increase in pulses, suggesting either non-linear deposition behavior or position-dependent variations. Additionally, Figure~\ref{fig:pressure_0.25T_mass/ion}a shows that at 14~cm, the mass flux should remain relatively constant across the pressure range (0--0.3~Pa), yet film 9 shows substantially lower thickness than predicted by this trend.\\

\textcolor{red}{[NOTE: Add discussion of apparent inconsistency at 14~cm between Figure~\ref{fig:pressure_0.25T_mass/ion}a trend (constant mass flux vs pressure) and observed film thicknesses. Potential explanations: spatial gradients between QCM/substrate positions, pulse-to-pulse variability (Table~4.1).]}


\subsection{X-ray Diffraction (XRD)}

X-ray diffraction measurements were performed to characterize the crystallographic structure and phase composition of the deposited films as a function of nitrogen pressure. Figure~\ref{fig:xrd_stacked} shows the stacked diffraction patterns for films deposited at 14~cm distance with 0.25~T magnetic field under varying N$_2$ pressures from 0~Pa (metallic mode) to 0.3~Pa (reactive mode).

\begin{figure}[h]
	\centering
	\includegraphics[width=0.95\linewidth]{Figures/results 2/XRD_stacked_qcm.png}
	\caption{Grazing incidence XRD patterns for Ti--Al and Ti--Al--N films deposited at different nitrogen pressures. Films were deposited at 14~cm distance with 0.25~T magnetic field. The intense peaks between 52$^\circ$ and 57$^\circ$ originate from the Si substrate. Vertical dashed lines indicate reference positions for cubic (Ti$_{0.65}$Al$_{0.35}$)N (111), (200), and (220) reflections at 37.0$^\circ$, 43.0$^\circ$, and 62.5$^\circ$ respectively (ICDD PDF 04-017-5094), as well as hexagonal Ti$_{0.7}$Al$_{0.3}$ (002) at 38.8$^\circ$ (ICDD PDF 04-004-9157).}
	\label{fig:xrd_stacked}
\end{figure}


\subsubsection{Phase Identification}

The diffraction patterns reveal a clear transition from metallic to nitride phases with increasing nitrogen pressure:



\paragraph{Metallic mode (0~Pa N$_2$):}

At zero nitrogen pressure, the film exhibits a weak, broad peak at approximately 38.8$^\circ$, consistent with the (002) reflection of hexagonal Ti$_{0.7}$Al$_{0.3}$ (space group P6$_3$/mmc, PDF 04-004-9157). The (101) reflection expected at 40.9$^\circ$ (I/I$_{\text{max}} = 100\%$ in the reference pattern) is not observed.


\paragraph{Reactive mode (0.1--0.3~Pa N$_2$):}

Introduction of nitrogen fundamentally alters the film structure. At 0.1~Pa N$_2$, three distinct peaks emerge at approximately 37$^\circ$, 43$^\circ$, and 62$^\circ$, corresponding to the (111), (200), and (220) reflections of the cubic B1 (NaCl-type) (Ti$_{0.65}$Al$_{0.35}$)N phase (space group Fm$\bar{3}$m, PDF 04-017-5094). The hexagonal TiAl peak disappears completely, indicating full conversion to the nitride phase.\\

As the nitrogen pressure increases to 0.2~Pa and 0.3~Pa, all three nitride reflection peaks sharpen and their positions align more closely with reference values. The (200) peak at 43$^\circ$ becomes the most prominent reflection, matching the highest relative intensity in the reference pattern (I/I$_{\text{max}} = 1000$) \cite{ICDD_TiAlN}.\\

Peak positions at 0.2~Pa and 0.3~Pa agree closely with reference values. Small deviations at 0.1~Pa correlate with the lower nitrogen content measured by EDX (33~at.\% vs 39~at.\%). The reference cubic nitride exhibits a lattice parameter of $a = 4.202$~\AA\ and a calculated density of 4.8~g\,cm$^{-3}$ \cite{ICDD_TiAlN}, which will be compared with the film densities determined by XRR (Section~\ref{sec:XRR}).\\


\subsection{XRR}






\subsection{Energy dispersive X-ray spectroscopy (EDX)}\label{sec:results_EDX}

Energy dispersive X-ray spectroscopy was performed to determine the elemental composition of the deposited films as a function of nitrogen pressure. All films were deposited at 14~cm distance with a magnetic field of 0.25~T. Three measurement points were taken on each sample and averaged to obtain representative composition values.


\begin{table}[h]
	\centering
	\caption{EDX film composition (atomic \%) as a function of nitrogen pressure and the resulting effective molar mass of the films (deposited at 14~cm, 0.25~T)}
	\label{tab:edx}
	\begin{tabular}{|c|c c c|c|}
		\hline
		Nitrogen (Pa) & N (at.\%) & Al (at.\%) & Ti (at.\%) & $M_{\text{eff}}$ (g $\cdot$ mol$^{-1}$) \\ \hline
		0.0 & 0 & 21 & 79 & 43.5\\ 
		0.1 & 33 & 17 & 50 & 33.2\\ 
		0.2 & 38 & 15 & 47 & 31.9\\ 
		0.3 & 39 & 14 & 47 &31.8\\ \hline
	\end{tabular}
\end{table}

The results in Table~\ref{tab:edx} reveal several key trends:

\textbf{Metallic mode (0~Pa):} In vacuum deposition, the film composition shows a Ti:Al ratio of approximately 79:21, compared to the cathode composition (62.8~at.\% Ti; 37.2~at.\% Al). The effective molar mass is 43.5~g$\cdot$mol$^{-1}$, between pure aluminum (27.0~g$\cdot$mol$^{-1}$) and pure titanium (47.9~g$\cdot$mol$^{-1}$).

\textbf{Reactive mode (0.1--0.3~Pa N$_2$):} Nitrogen incorporation reaches approximately 33~at.\% at 0.1~Pa and saturates near 39~at.\% above 0.2~Pa. The final composition at 0.3~Pa is Ti$_{0.47}$Al$_{0.14}$N$_{0.39}$, close to the reference material composition used for XRD analysis \cite{ICDD_TiAlN}. The Ti:Al ratio shifts from approximately 3.75:1 in metallic mode to approximately 3.4:1 in reactive mode.


\textbf{Effective molar mass trends:} The effective molar mass decreases from 43.5~g$\cdot$mol$^{-1}$ (metallic mode) to approximately 32~g$\cdot$mol$^{-1}$ (reactive mode), a reduction of approximately 25\%. The molar mass saturates above 0.2~Pa at $M_{\text{eff}} \approx 31.8$~g$\cdot$mol$^{-1}$, consistent with the compositional saturation observed in nitrogen content.

The effective molar mass is calculated from the atomic fractions $x_i$ and atomic masses $M_i$ according to:
\begin{equation}
	M_{\text{eff}} = \sum_i x_i M_i = x_{\text{Ti}} \cdot 47.867 + x_{\text{Al}} \cdot 26.982 + x_{\text{N}} \cdot 14.007
\end{equation}
where the atomic fractions sum to unity ($x_{\text{Ti}} + x_{\text{Al}} + x_{\text{N}} = 1$).


\subsection{Scanning Electron Microscopy (SEM)}\label{sec:result_sem}

Scanning electron microscopy was performed to examine the surface morphology of selected TiAl and TiAlN films. Figure~\ref{fig:sem_figures} shows representative images for films deposited under different conditions.\\


\begin{figure}[h]
	\centering
	\subcaptionbox{Metallic film (0~Pa)\label{fig:sem_metallic}}{\includegraphics[width=0.48\textwidth]{Figures/results 2/251209_TiAlN10_01.png}}
	\subcaptionbox{Macroparticle detail\label{fig:sem_macro}}{\includegraphics[width=0.48\textwidth]{Figures/results 2/251209_TiAlN11_09.png}}
	\subcaptionbox{Reactive film (0.3~Pa)\label{fig:sem_reactive}}{\includegraphics[width=0.48\textwidth]{Figures/results 2/251209_TiAlN13_05.png}}
	\caption{SEM surface morphology of deposited films at 14~cm distance, 0.25~T magnetic field: (a) metallic TiAl film showing surface swelling features and scattered macroparticles (10,000$\times$, 1~$\mu$m scale), (b) higher magnification view of an individual macroparticle on a reactive TiAlN film (50,000$\times$, 200~nm scale), (c) reactive TiAlN film with characteristic splash patterns from macroparticle impacts (10,000$\times$, 1~$\mu$m scale).}
	\label{fig:sem_figures}
\end{figure}

The metallic TiAl film [Fig.~\ref{fig:sem_figures}(a)] exhibits localized surface swelling features, likely resulting from stress relaxation induced by energetic ion bombardment ($E_{\text{tot}} \approx 60$~eV). Scattered macroparticles are visible as bright spots across the surface, consistent with typical cathodic arc deposition despite magnetic filtering.\\

The high-magnification image [Fig.~\ref{fig:sem_figures}(b)] shows a direct comparison between secondary electron imaging (left, sensitive to topography) and backscattered electron imaging (right, sensitive to atomic number contrast) of the same macroparticle on a TiAlN film. The absence of compositional contrast in the BSE image confirms that the macroparticle has the same elemental composition as the surrounding film, indicating it originated from molten cathode material rather than external contamination.\\

The reactive TiAlN film [Fig.~\ref{fig:sem_figures}(c)] macroparticle are present, but lower quantities than for the metallic case. The overall surface remains relatively smooth between defects, with no evidence of large-scale roughening or columnar grain boundaries at this magnification.\\


These observations confirm that the magnetic macroparticle filter effectively reduces but does not completely eliminate macroparticle contamination, consistent with the filtered cathodic arc deposition process described in Section~\ref{chap:methods}.

\section{Particle Flux Calculations}\label{sec:flux}

The ion current and mass deposition rate measurements represent electrical and mass quantities that must be converted to particle fluxes using charge-state data from ERMS (Section~\ref{sec:erms_results}) and compositional data from EDX (Section~\ref{sec:results_EDX}). This section presents the derived ion flux $\Gamma_{\text{ion}}$ and total atomic flux $\Phi_{\text{total}}$ for the 14~cm deposition distance.

\subsection{Calculation Methods}

The ion flux (ions$\cdot$cm$^{-2}$$\cdot$s$^{-1}$) is calculated from the ion current measurements:
\begin{equation}
	\Gamma_{\text{ion}} = \frac{I_{\text{ion}}}{e \langle Q \rangle A_{\text{probe}}}
\end{equation}
where $I_{\text{ion}}$ is the measured ion current, $e = 1.602 \times 10^{-19}$~C, $A_{\text{probe}} = 0.196$~cm$^2$, and $\langle Q \rangle$ is the weighted mean charge state from ERMS:
\begin{equation}
	\langle Q \rangle = \frac{\sum_i Q_i \cdot I_i^{\text{ERMS}}}{\sum_i I_i^{\text{ERMS}}}
\end{equation}
where the sum includes all charge states of Ti and Al.

The total atomic flux (atoms$\cdot$cm$^{-2}$$\cdot$s$^{-1}$) is calculated from QCM measurements:
\begin{equation}
	\Phi_{\text{total}} = \frac{\Delta m \cdot N_A}{A_{\text{QCM}} \cdot \Delta t \cdot M_{\text{eff}}}
\end{equation}
where $\Delta m$ is the mass change over time interval $\Delta t$, $A_{\text{QCM}} = 0.5027$~cm$^2$, $N_A = 6.022 \times 10^{23}$~mol$^{-1}$, and $M_{\text{eff}}$ is the effective molar mass from Table~\ref{tab:edx}.

\subsection{Flux Measurements at 14~cm Distance}

Figure~\ref{fig:flux_vs_pressure} presents the calculated ion flux and atomic flux as functions of nitrogen pressure at 14~cm distance with 0.25~T magnetic field. 

\begin{figure}[h]
	\centering
	\includegraphics[width=0.95\linewidth]{Figures/results 2/FLUX_Pressure_14cm_0.25T.png}
	\caption{Ion flux and atomic flux as functions of nitrogen pressure at 14~cm distance with 0.25~T magnetic field. Ion flux (blue, left axis) remains approximately constant at $(0.7$--$1.0) \times 10^{17}$~ions$\cdot$cm$^{-2}$$\cdot$s$^{-1}$ across the pressure range. Atomic flux (orange, right axis) exhibits non-monotonic pressure dependence with a maximum near 0.2~Pa. Error bars on ion flux represent combined uncertainties from probe current and ERMS charge-state measurements.}
	\label{fig:flux_vs_pressure}
\end{figure}

The ion flux remains relatively constant at $(0.7$--$1.0) \times 10^{17}$~ions$\cdot$cm$^{-2}$$\cdot$s$^{-1}$ across the measured pressure range, with large error bars ($\pm 30$--$50\%$) reflecting the combined uncertainties from probe current measurements and ERMS charge-state determination. The pressure independence is consistent with the ion current behavior observed in Figure~\ref{fig:pressure_0.25T_mass/ion}b at 14~cm distance.\\

The atomic flux shows non-monotonic pressure dependence, increasing from $2.6 \times 10^{17}$~atoms$\cdot$cm$^{-2}$$\cdot$s$^{-1}$ in metallic mode to a maximum of $4.0 \times 10^{17}$~atoms$\cdot$cm$^{-2}$$\cdot$s$^{-1}$ at 0.2~Pa, before decreasing slightly to $3.3 \times 10^{17}$~atoms$\cdot$cm$^{-2}$$\cdot$s$^{-1}$ at 0.3~Pa.

\subsection{Observations on Flux Magnitudes}

The atomic flux is systematically higher than the ion flux by factors of 3--4 across all pressure conditions. In metallic mode (0~Pa), where film composition is purely Ti and Al without nitrogen, the atomic flux is approximately 3.7 times larger than the ion flux. This ratio increases to approximately 4 in reactive mode.\\

Several factors may contribute to this discrepancy:
\begin{itemize}
	%\item Spatial gradients between the probe and QCM measurement positions (Section~\ref{sec:holder_assembly_figure})
	%\item Uncertainties in mean charge state determination from ERMS peak integration
	\item The QCM to flux conversion using $M_{\text{eff}}$ of the resulting film instead of the plasma
	\item Systematic errors in probe collection or measurement resistance values
	%\item Time-averaging differences between continuous QCM measurements and pulsed probe measurements
\end{itemize}

The origin of the flux magnitude difference requires further investigation. The relative trends (pressure independence of ion flux, non-monotonic behavior of atomic flux) are considered robust, while the absolute magnitudes carry larger systematic uncertainties. Detailed analysis of this discrepancy is deferred to future work.


