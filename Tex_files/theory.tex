\chapter{Theoretical Background}

\section{Plasma Generation and Composition}

\subsection{Cathode Spot Plasma Generation}

Cathodic‐arc plasmas are born in microscopic emission centers, so‐called cathode spots, on an otherwise cold metal electrode under vacuum. A spot is ignited when the local cathode surface, often via a breakdown of adsorbates or field-enhanced thermionic emission, undergoes a rapid, explosive release of electrons and vaporized metal. During a single spot pulse, a few nanograms of the cathode material rapidly heat up, vaporize, and ionize, producing a dense, quasineutral plasma plume composed mostly of metal ions and electrons. The peak spot current densities reach \(10^{10}\)–\(10^{12}\)\,A\,m\(^{-2}\), far above steady-state thermionic or field emission limits. These microexplosions, known as ectons (explosive electron emission centers), are characterized by localized, nanosecond-scale bursts of plasma and are sustained by repetitive ecton events at the same nominal location \cite[Chap.~3.3--3.4]{cathodic_arcs}.\\




Key Characteristics of Spot-Generated Plasma:
\begin{itemize}
    \item High degree of ionization: >90 \% of the ejected metal atoms emerge as ions, a consequence of the extreme power density in the cathode spot \cite[Chap.~3.5]{cathodic_arcs}.
    \item Multiply charged ions: the charge state distributions extend to \(Q=3\)–4 for refractory metals, such as Ti and Al, due to the high electron temperature and density in the spot plasma \cite[Chap.~3.5]{cathodic_arcs}.
    \item Transient, localized heating: the sub-µm, sub-100 ns pulse produces “atomic‐scale heating”, where the energy of individual ions is deposited in a highly localized region upon impact, influencing film growth and microstructure \cite[Chap.~3.6]{cathodic_arcs}.
\end{itemize}

Spot ignition and quenching occur on the order of 10–100 ns, with each pulse launching a fully ionized slug of metal vapour. The sustained arc discharge therefore comprises a continual overlap of these microplasma bursts, producing a metal-rich, high-flux ion stream ideal for energetic thin-film growth.\\


\subsection{Plasma Composition and Expansion}

Once generated at the cathode spots, the metal-rich plasma expands into the chamber or through the magnetic macroparticle filter. In the region near the cathode (within a few centimeters of the spot), plasma densities are on the order of \(10^{18}\) cm$^{-3}$ and electron temperatures \(T_e\approx5\ \text{–}\ 10\) eV. As the plume propagates, its density decreases according to
\begin{equation}
    n(r) = \frac{C\,I_{\rm arc}}{r^2}
\end{equation}
where \(I_{\rm arc}\) is the arc current $r$ the distance, and $C$ a constant related to the ion‐erosion rate of the cathode material. This $\frac{1}{r^2}$ scaling assumes free expansion, but deviations can occur due to magnetic fields, collisions, or reactive gases, which may alter the plasma trajectory or cause recombination \cite[Chap.~4.3; Eq.~4.3, p.~178]{cathodic_arcs}.\\

In cathodic‐arc discharges from titanium cathodes, whether pure Ti or Ti–Al compounds, ions generally carry an average charge state \(\langle Q\rangle\approx2.1\text{–}2.2\) \cite[Chap.~4.1; App.~B.8]{cathodic_arcs}. This high degree of ionization reflects the extreme power density of the spot and follows the cohesive energy rule, which links \(\langle Q\rangle\) to the cohesive energy of the cathode material (Table B.8) \cite[App.~B.8]{cathodic_arcs}.

%Applying an external axial magnetic field at the source (the “EM-coil” configuration) can boost ⟨Q⟩ by 10–30 \% and increase total ion flux by up to an order of magnitude. Effects attributed to enhanced magnetic insulation and longer plasma spot interaction times \cite{decoupling_kalanov_2025}. In contrast, imposing a DC bias on the source raises the ions’ kinetic energy with minimal change to ⟨Q⟩ or flux, offering a route to decouple kinetic and potential-energy contributions in film-growth studies.\\


\section{Ion Energy and Flux}

\subsection{Ion Energies: Origins and Implications}
\label{section:ion_energies}

Ion energies in cathodic arc plasmas are well-documented, with Ti$^{2+}$ and Al$^{2+}$ ions carrying kinetic energies of $E_{kin}^{\text{Ti}^{2+}} \approx 58.9\ \text{eV}$ and $E_{kin}^{\text{Al}^{2+}} \approx 27.5\ \text{eV}$, respectively \cite[Chap.~4.4; Table B.8]{cathodic_arcs}. Combined with potential energy from ionization, these ions reach total energies ($E_{tot}^{\text{Ti}^{2+}} \approx 79.5\ \text{eV}$, $E_{tot}^{\text{Al}^{2+}} \approx 52.0\ \text{eV}$) that exceed the $\approx 30\ \text{eV}$ threshold for subplantation, enabling densification and crystallinity in Ti--Al--N films \cite[Chap.~8.1--8.2]{cathodic_arcs}.\\

While ion energies contribute to film properties, this work prioritizes quantifying the flux of excited nitrogen species alongside metal ions to understand their combined role in film growth and densification.\\

In metallic mode, the narrow ion energy distribution simplifies flux measurements, allowing direct correlation with deposition outcomes.

In reactive mode, collisions with N$_2$ not only broaden the energy distribution but also generate excited nitrogen (e.g., N$^+$, N$_2^+$, and metastable neutrals), which must be resolved in QMS spectra \cite{bendikt2012}.\\

To isolate the effects of ion flux ($\Gamma$) and excited nitrogen, we systematically vary the N$_2$ pressure over a wide range and adjust the field strength of the EM coil, ensuring that changes in film properties reflect controlled variations in plasma composition and flux rather than incidental energy shifts.



%i dont htink this below is needed
\iffalse

\subsection{Ion Energies in Cathodic Arcs}\label{section:ion energies}

Time‐of‐flight measurements in vacuum arcs \cite[Chap.~4.4; Table B.8]{cathodic_arcs} report most likely drift velocities of:
\begin{align*}
    v_{mp}^{\text{Ti}} \approx 1.54 \times 10^4\ \text{m/s}\ &; E_{kin}^{\text{Ti}^{2+}} \approx 58.9\ \text{eV}\\
    v_{mp}^{\text{Al}} \approx 1.39  \times 10^4\ \text{m/s}\ &; E_{kin}^{\text{Al}} \approx 27.5\ \text{eV}
\end{align*}

These values reflect the directed kinetic energy of ions in the plasma plume. However, charge-exchange collisions (for example, with the background gas in reactive mode) can reduce the drift velocity and broaden the ion energy distribution \cite[Chap.~9.4]{cathodic_arcs}.\\

The potential energy component follows from the cumulative ionization enthalpies \cite[Chap.~4.2; Table B.4]{cathodic_arcs}, and therefore the total energy carried by each ion upon impact is:
\begin{align*}
     E_{pot}^{\text{Ti}^{2+}} \approx E_{0 \rightarrow 2} = 20.6\ \text{eV},&\qquad  
     E_{pot}^{\text{Al}^{2+}} \approx E_{0 \rightarrow 2} = 24.5\ \text{eV}\\
     E_{tot}^{\text{Ti}^{2+}} \approx 58.9 + 20.6 \approx 79.5\ \text{eV},&\qquad  
     E_{tot}^{\text{Al}^{2+}} \approx 27.5 + 24.5 \approx 52.0\ \text{eV}
\end{align*}

Since both Ti$^{2+}$ ($\approx$ 80 eV) and Al$^{2+}$ ions ($\approx$ 50 eV) carry energies well above the $\approx$ 30 eV threshold for subplantation in transition metal nitrides, these ions will implant beneath the growing surface and generate localized 'atomic scale heating'.  That process is what drives densification and improved crystallinity without global substrate heating \cite[Chap.~8.1–8.2]{cathodic_arcs}.\\ 


The ion‐energy distribution function (IEDF) can additionally be modeled as a shifted Maxwellian:
\begin{equation}
    f(E) = \frac{E}{(kT_i)^{3/2}} \exp\!\Big(-\frac{(\sqrt{E}-\sqrt{E_{dir}})^2}{kT_i}\Big)
\end{equation}

Here, \(T_i\) represents the \textit{ion temperature}, characterizing the random thermal motion superimposed on the directed drift velocity \(v_{i,\text{drift}}\).\\

In this model, \(E_{dir} = \tfrac12\,m_i\,v_{i,\rm drift}^2\) represents the kinetic energy obtained from acceleration away from the cathode spot, while \(T_i\) reflects the random thermal motion of the ions. \cite[Chap.~4.4; Eq.~4.27]{cathodic_arcs}. In metallic mode, the IEDF is dominated by $E_{dir}$, with a typical width (FWHM) of 10 – 20 eV due to thermal broadening and spot-to-spot fluctuations.\\

However, in reactive mode, collisions with the background gas (e.g. N$_2$) introduce additional scattering, which broadens the IEDF and may reduce the average ion energy \cite[Chap.~9.4]{cathodic_arcs}. This effect is particularly important in our experiments, where N$_2$ is introduced to study the growth of the TiAlN film.\\

N$_2$ not only reduces \(E_{\text{dir}}\) but also introduces low-energy nitrogen ions (N$_2^+$, N$^+$), which contribute to film nitridation but complicate energy distribution measurements due to overlapping peaks in QMS spectra \cite{bendikt2012}. By fitting the measured IEDFs to the shifted Maxwellian model, we can quantify how $E_{dir}$ and $T_i$ vary with magnetic field strength and gas pressure, providing insight into the energy deposition mechanisms during film growth.\\


Modern source modifications allow selective tuning of these energy channels:
\begin{itemize}
    \item DC biasing increases \(E_{\rm kin}\) via an additional sheath acceleration without appreciably changing Q or total flux.
    \item An external EM coil increases Q (and therefore \(E_{\rm pot}\)) by 10-30 \% while leaving \(v_{mp}\) essentially constant \cite{decoupling_kalanov_2025,unutulmazsoy}.
\end{itemize}

This ability to decouple kinetic and potential contributions is central to our strategy of correlating ion‐probe currents (flux × charge) with QCM‐measured deposition rates under rigorously controlled energy conditions.

\fi




\subsection{Ion Flux and Diagnostics}
\label{section:ion_flux}

The ion flux, \(\Gamma\) (ions\(\cdot\)cm$^{-2} \cdot$s$^{-1}$), represents the number of ions impinging per unit area per unit time. In a multiply charged plasma, the total measured ion current density \(J_i\) (A\(\cdot\)cm$^{-2}$) relates to \(\Gamma\) via:

\begin{equation}
    \Gamma = \frac{J_i}{e\,\langle Q\rangle}.
\end{equation}

This relationship is central to our experiments, as the time-averaged ion flux \(\Gamma\) is expected to correlate with the total deposited mass measured by the QCM (Section~\ref{section:QCM}).\\

To compare ion fluxes across different materials and arc currents, we use the particle system coefficient:

\begin{equation}
    k_{\rm part} = \frac{I_i}{\langle Q\rangle\,I_{\rm arc}},
\end{equation}

which normalizes the probe current \(I_i\) by the arc current $I_{\rm arc}$ and accounts for variations in the average charge state \(\langle Q\rangle\) \cite[Chap.~6.5]{cathodic_arcs}. An external magnetic field can increase \(\Gamma\) by up to an order of magnitude by confining the plasma and prolonging the ion residence time near the cathode \cite[Chap.~6.5]{cathodic_arcs}. This enhancement is particularly relevant for our study, where magnetic fields are used to tune plasma properties and investigate their influence on the deposited mass.\\

In vacuum cathodic arcs, the burning voltage remains nearly constant at 20--30 V for arc currents up to 1 kA, so the plasma generation rate - and thus \(\Gamma\) - increases almost linearly with \(I_{\rm arc}\) \cite[Chap.~6.5]{cathodic_arcs}. Introducing source modifications, such as an external magnetic coil, increases the spot power density and prolongs the plasma residence time at the cathode, further increasing the total ion flux by up to an order of magnitude.\\

To link ion arrival with film growth, we mounted a QCM adjacent to the ion probe. Each ion that sticks to the crystal shifts its resonance frequency by \(\Delta f\). Using the Sauerbrey equation, we convert \(\Delta f\) into a mass flux \(\dot{m}\) and then divide by the density of the film \(\rho_{\rm film}\) to obtain the thickness growth rate:

\begin{equation}
    R = \frac{\dot{m}}{\rho_{\rm film}} = \frac{m_{\rm ion}\,\Gamma\,S}{\rho_{\rm film}},
\end{equation}

where \(m_{\rm ion}\) is the mass of an ion, \(\Gamma\) the ion flux measured by the probe, \(S\) the sticking coefficient (\(\approx 1\)), and \(\rho_{\rm film}\) the density of the film.\\

In previous research by Unutulmazsoy et al.\ (2023), when ion energy is fixed, \(R\) rises almost linearly with \(\Gamma\). Any deviation from this linear behavior signals that other processes, such as densification or adatom crowding, are beginning to influence film growth \cite{unutulmazsoy}.\\

To connect the theory of ion flux and energy to experimental data, we employ two primary diagnostics:

\begin{itemize}
    \item A current-density probe to record the time-resolved ion current $I_i(t)$, from which we compute the flux $\Gamma(t)=\frac{I_i(t)}{(e\langle Q\rangle )}$.
    \item A quadrupole mass spectrometer (QMS) to resolve the ion-energy distribution (IEDF) and charge-state spectrum, yielding $E_{kin}$ and $\langle Q\rangle $.
\end{itemize}


Together, these tools let us map how variations in $\Gamma$, $E_{kin}$ and $\langle Q\rangle$ translate into film growth. The detailed design of the probe, the geometry of the QMS orifice, the calibration procedures, and the data acquisition settings are described in Chapter \ref{sec:QMS}.

\section{Plasma-Surface Interactions and Film Growth}
\subsection{Energetic Condensation and Subplantation}

When energetic metal ions strike the growing film, they can penetrate the surface and deposit their energy in a shallow 'collision cascade'.  This subplantation process leads to two key effects:

\begin{itemize}
    \item \textbf{Localized densification}:\\
    Ions with kinetic energies above the $\approx$ 20–30 eV threshold implant beneath the growing surface, filling interstitial sites and displacing adatoms via knock-on collisions. This process reduces porosity and increases the density of the film, which is particularly critical for Ti–Al–N coatings \cite[Chap.~8.1]{cathodic_arcs}.
    
    \item \textbf{Atomic‐scale heating}:\\
    The deposition of kinetic energy and the release of potential energy (ionization enthalpy) generate localized, nanosecond-scale temperature spikes, enhancing adatom mobility and promoting crystallite coalescence without global substrate heating \cite[Chap.~8.2]{cathodic_arcs}.
\end{itemize}


As \(E_{\rm kin}\) and \(E_{\rm pot}\) increase, transition from porous, amorphous structures to dense, crystalline coatings with compressive stresses of several GPa, driven by atomic peening \cite[Chap.~8.1–8.4]{cathodic_arcs}.  For example, TiN films grown with \(E_{\rm kin}\approx40\) eV and \(E_{\rm pot}\approx20\) eV develop a preferred cubic (111) texture and hardness > 30 GPa.

In this work, we systematically investigate how controlling $E_{\rm kin}$, $E_{\rm pot}$ (Section~\ref{}), and tuning the ion flux $\Gamma$ (Section~\ref{}) influence densification, texture evolution, and stress development in Ti–Al–N coatings. The results are presented in Chapter~\ref{}.

%add reference to which chapter


\subsection{Reactive vs Metallic Mode}

Cathodic-arc deposition can operate in two distinct regimes: metallic mode and reactive mode. In metallic mode, the cathode surface remains uncovered, and the plasma consists exclusively of metal ions. This regime is characterized by a stable burning voltage, minimal macroparticle emission, and a high metal ion flux. In reactive mode, a background gas (e.g., N$_2$) adsorbs onto the cathode surface, forming a compound layer that 'poisons' the cathode. This alters both the behaviour of the cathode spot and the composition of the plasma \cite[Chap.~9.2]{cathodic_arcs}.\\

When N$_2$ is introduced, a dynamic equilibrium forms between compound formation (adsorption and reaction at the cathode, which suppresses metal emission) and compound removal (via explosive ecton events that clean the spot and eject both metal and compound fragments) \cite[Chap.~9.3]{cathodic_arcs}.\\

The equilibrium position depends on gas pressure, arc current, and cathode composition. At low N$_2$ pressures or high power densities, type-2 (metal-rich) spots prevail, maintaining a predominantly metal ion flux. At higher pressures or for larger cathode areas, type-1 (poisoned) spots dominate, producing a mixed plasma of metal and N$_x$ ions while suppressing macroparticle emission \cite[Chap.~9.4]{cathodic_arcs}. This poisoning effect is reversible and depends on the balance between compound formation and ecton-induced cleaning \cite[Chap.~9.3]{cathodic_arcs}.\\

Reactive mode affects both plasma diagnostics and film growth:

\begin{itemize}
    \item The probe current includes contributions from N$^+$ and N$_2^+$ ions in addition to metal ions, necessitating mass-resolved QMS analysis to distinguish between species \cite{bendikt2012}.

    \item Charge exchange with N$_2$ reduces the average charge state of metal ions (Q) and introduces additional gas ion charge states, altering the overall potential energy budget \cite[Chap.~9.4]{cathodic_arcs}.

    \item Collisions between ions and N$_2$ molecules during plasma expansion lower ion drift velocities, thereby reducing their kinetic energy before substrate impact \cite[Chap.~9.4]{cathodic_arcs}.
\end{itemize}


In our Pi-PVD system, N$_2$ is introduced via a ring-manifold inlet located downstream of the magnetic filter, minimizing pressure gradients and enabling reproducible reactive mode operation. By comparing depositions in vacuum and under varying N$_2$ concentrations, we investigate how reactive mode influences $\Gamma$, $E_{\rm kin}$, and Q, and how these parameters affect the deposited mass (measured by QCM) and the resulting film microstructure (Chapter~\ref{}).

\section{Crystal Structure and Densification}

\subsection{Nucleation and Growth Modes}
