\chapter{Experimental Methodology}\label{chap:methods}

\section{Experimental Apparatus and Setup}

\subsection{Vacuum \& Gas Infrastructure}
The chamber was evacuated using a two-stage pumping system consisting of a turbo-molecular pump (backed by a rotary vane pump for initial roughing) and a cryogenic pump, achieving a base pressure on the order of $ 1 \times 10^{-5}$~Pa. Nitrogen gas (N$_2$, 99.999\% purity) was introduced via a mass flow controller (MFC), with chamber pressures ranging from 0.025--0.3~Pa during experiments,\\

Plasma was generated using a water-cooled anode and a cathode with the aforementioned composition of 75~wt\% Ti; 25~wt\% Al (62.8~at\%; Ti 37.2~at\% Al) with a diameter of 6.35~mm and 38.1~mm long. The arc power supply operated in pulsed DC mode, delivering up to 450~A at a pulse frequency of 0.2--5~Hz, as well as powering the 90$^\circ$ curved macroparticle filter in series. An accelerator coil (EM-coil), capable of currents up to 850~A, was pulsed 200~$\mu$s before arc ignition, to stabilize the magnetic field. The QCM and Langmuir probe were mounted on a custom movable block at 10--20~cm from the filter exit, while the energy-resolving mass spectrometer (ERMS) was positioned using a linear feedthrough for external adjustment. Depositions were done with another movable mount and positioned at the necessary distances to the filter exit.\\

The vacuum chamber setup for cathodic arc plasma diagnostics and thin film deposition is shown in Figure~\ref{fig:vacuumchamber}.
The arc power supply generates and steers the plasma, while the EM-coil power supply enhances the plasma energy and confinement.\\

The green arrow illustrates the trajectory of the plasma plume as it is expanding from the cathode, passing through the macroparticle filter, and reaching the diagnostics.

An energy-resolving mass spectrometer (ERMS) analyzes the ion energy and mass distribution, with its position adjusted relative to the macroparticle filter, to study spatial variations in the plasma plume. A Langmuir probe measures the ion current and a quartz crystal microbalance (QCM) monitors the deposited mass. An oscilloscope records time-resolved electrical signals: channels 1 and 2 measure the voltage drop at the cathode, while channels 3 and 4 capture the current supplied to the arc and EM-coil, as well as the ion current collected by the probe on another channel.
The delay generator acts as a master clock, triggering the arc power supply, EM-coil activation, and diagnostic tools with precise timing to ensure that ion flux, energy, and deposition rate measurements are directly comparable and time-correlated.




\begin{figure}[!ht]
\centering
\resizebox{1\textwidth}{!}{%
\begin{circuitikz}
\tikzstyle{every node}=[font=\Large]
\draw  (4,10) ellipse (0.25cm and 1.5cm);
\node [font=\LARGE] at (4,9.5) {};
\node [font=\LARGE] at (4,9.5) {};
\node [font=\LARGE] at (4,9.5) {};
\draw [ rotate around={-12:(4.25,10)}] (4.25,10) ellipse (0.25cm and 1.5cm);
\draw [ rotate around={-25:(4.5,10)}] (4.5,10) ellipse (0.25cm and 1.5cm);
\draw [ rotate around={-56:(5.25,9.25)}] (5.25,9.25) ellipse (0.25cm and 1.5cm);
\draw [ rotate around={-37:(4.75,9.75)}] (4.75,9.75) ellipse (0.25cm and 1.5cm);
\draw [ rotate around={-69:(5.5,9)}] (5.5,9) ellipse (0.25cm and 1.5cm);
\draw [ rotate around={-88:(5.5,8.5)}] (5.5,8.5) ellipse (0.25cm and 1.5cm);
\draw [ rotate around={-44:(5,9.5)}] (5,9.5) ellipse (0.25cm and 1.5cm);
\draw [ color={rgb,255:red,224; green,27; blue,36} ] (5.5,7.75) ellipse (1cm and 0.25cm);
\draw [short] (7,8.5) -- (7,6.25);
\draw [short] (4,11.5) -- (4,12.75);
\draw [short] (4,12.75) -- (11,12.75);
\draw [short] (5.5,5.25) -- (5.5,4.25);
\draw [ fill={rgb,255:red,222; green,221; blue,218} , line width=1pt ] (5,6.5) rectangle (6,5.5);
\draw [ color={rgb,255:red,224; green,27; blue,36}, short] (11,6.75) -- (5.75,6.75);
\draw [ color={rgb,255:red,224; green,27; blue,36}, short] (11.5,8) -- (5.75,8);
\draw [ fill={rgb,255:red,119; green,118; blue,123} ] (5.25,4.75) rectangle (5.75,6.75);
\node [font=\LARGE] at (4.75,5.25) {};
\draw [short] (7,6.25) -- (6,6.25);
\draw [ color={rgb,255:red,224; green,27; blue,36} ] (5.5,7) ellipse (1cm and 0.25cm);
\draw [ color={rgb,255:red,224; green,27; blue,36} ] (5.5,7.25) ellipse (1cm and 0.25cm);
\draw [ color={rgb,255:red,224; green,27; blue,36} ] (5.5,7.5) ellipse (1cm and 0.25cm);
\draw [line width=1.5pt, short] (5,6.5) -- (5,7.75);
\draw [line width=1.5pt, short] (6,6.5) -- (6,7.75);
\draw [ fill={rgb,255:red,246; green,245; blue,244} ] (-9,7.25) rectangle  node {\Large Ion probe circuit} (-3.75,6);
\draw [ fill={rgb,255:red,246; green,245; blue,244} ] (-5.75,5.5) rectangle  node {\Large QCM} (-3.25,3.75);
\node [font=\LARGE] at (-1.5,11.5) {};
\node [font=\Large] at (-1.5,12) {};
\node [font=\normalsize] at (-0.75,11.5) {};
\draw [ fill={rgb,255:red,246; green,245; blue,244} ] (10.25,14.75) rectangle  node {\Large Arc PSU} (15.25,13);
\draw [ fill={rgb,255:red,246; green,245; blue,244} ] (10.25,6) rectangle  node {\Large EM-coil PSU} (15.25,4.25);
\draw [ color={rgb,255:red,224; green,27; blue,36}, short] (11.5,8) -- (11.5,6);
\draw [ color={rgb,255:red,224; green,27; blue,36}, short] (11,6.75) -- (11,6);
\draw (14.75,6) to[short, -o] (14.75,7) ;
\draw (14.75,13) to[short, -o] (14.75,12) ;
\draw (-7.25,6) to[short, -o] (-7.25,5) ;
\draw [ fill={rgb,255:red,246; green,245; blue,244} ] (-9,3.25) rectangle  node {\Large Oscilloscope} (-3.25,0);
\draw  (10,8) ellipse (0.25cm and 0.5cm);
\draw  (7.75,12.75) ellipse (0.25cm and 0.5cm);
\draw [short] (10,8.5) -- (10,9);
\draw [short] (7.75,13.25) -- (7.75,13.75);
\draw (7.75,13.75) to[short, -o] (7.75,14) ;
\draw (10,9) to[short, -o] (10,9.25) ;
\draw [ color={rgb,255:red,129; green,61; blue,156}, line width=1pt, ->, >=Stealth, dashed] (13,10.75) -- (11.5,10.75);
\draw [ color={rgb,255:red,129; green,61; blue,156}, line width=1pt, ->, >=Stealth, dashed] (13,11.5) -- (10.25,12.5);
\node [font=\large, color={rgb,255:red,229; green,165; blue,10}] at (14,11.75) {};
\node [font=\Large, color={rgb,255:red,129; green,61; blue,156}] at (13.5,11.5) {(1)};
\node [font=\Large, color={rgb,255:red,129; green,61; blue,156}] at (13.5,10.75) {(2)};
\node [font=\Large, color={rgb,255:red,229; green,165; blue,10}] at (7.25,14) {(3)};
\node [font=\Large, color={rgb,255:red,229; green,165; blue,10}] at (10,9.75) {(4)};
\node [font=\Large, color={rgb,255:red,224; green,27; blue,36}] at (14.75,11.5) {(a)};
\node [font=\Large, color={rgb,255:red,224; green,27; blue,36}] at (14.75,7.5) {(b)};
\node [font=\Large, color={rgb,255:red,224; green,27; blue,36}] at (-7.25,4.5) {(c)};
\node [font=\Large, color={rgb,255:red,224; green,27; blue,36}] at (-9.75,10.25) {(d)};
\draw (-8.5,10.25) to[short, -o] (-9.25,10.25) ;
\draw [ color={rgb,255:red,129; green,61; blue,156}, ->, >=Stealth, dashed] (-3.25,2.75) -- (-1.5,2.75);
\draw [ color={rgb,255:red,129; green,61; blue,156}, ->, >=Stealth, dashed] (-3.25,2) -- (-1.5,2);
\draw (-3.25,1.25) to[short, -o] (-1.75,1.25) ;
\draw (-3.25,0.5) to[short, -o] (-1.75,0.5) ;
\node [font=\Large, color={rgb,255:red,129; green,61; blue,156}] at (-1,2.75) {(1)};
\node [font=\Large, color={rgb,255:red,129; green,61; blue,156}] at (-1,2) {(2)};
\node [font=\Large, color={rgb,255:red,229; green,165; blue,10}] at (-1,1.25) {(3)};
\node [font=\Large, color={rgb,255:red,229; green,165; blue,10}] at (-1,0.5) {(4)};
\draw (-9,1) to (-9.25,1) node[ground]{};
\draw (-5.75,4.25) to (-6.5,4.25) node[ground]{};
\draw (-9,6.75) to (-9.5,6.75) node[ground]{};
\draw (10.25,5.25) to (9.5,5.25) node[ground]{};
\draw (10.25,14) to (9.5,14) node[ground]{};
\draw (-7.5,9.5) to (-7.5,9.25) node[ground]{};
\draw [ line width=0.5pt ] (-3.5,9) rectangle (-2.5,8);
\draw [line width=0.7pt, <->, >=Stealth] (-1.5,9.25) -- (3.5,9.25);
\draw [ color={rgb,255:red,51; green,209; blue,122}, line width=2pt, ->, >=Stealth] (5.5,6.75) .. controls (5.5,9.75) and (5.25,10.5) .. (-0.5,10.5) ;
\node [font=\Large, color={rgb,255:red,46; green,194; blue,126}] at (2,10.75) {Plasma};
\draw [ fill={rgb,255:red,246; green,245; blue,244} ] (9.5,3.25) rectangle  node {\Large Delay generator} (15.25,0);
\draw (9.5,2.75) to[short, -o] (8.25,2.75) ;
\draw (9.5,2) to[short, -o] (8.25,2) ;
\draw (9.5,1.25) to[short, -o] (8.25,1.25) ;
\draw (9.5,0.5) to[short, -o] (8.25,0.5) ;
\node [font=\Large, color={rgb,255:red,224; green,27; blue,36}] at (7.5,2.75) {(a)};
\node [font=\Large, color={rgb,255:red,224; green,27; blue,36}] at (7.5,2) {(b)};
\node [font=\Large, color={rgb,255:red,224; green,27; blue,36}] at (7.5,1.25) {(c)};
\node [font=\Large, color={rgb,255:red,224; green,27; blue,36}] at (7.5,0.5) {(d)};
\draw [ color={rgb,255:red,224; green,27; blue,36}, line width=0.7pt, <->, >=Stealth] (-2.25,10.5) -- (-2.25,8);
\draw [ color={rgb,255:red,224; green,27; blue,36}, line width=0.7pt, dashed] (-2.25,7.75) -- (-4.5,7.75);
\draw [ color={rgb,255:red,224; green,27; blue,36}, line width=0.7pt, dashed] (-4.5,10.75) -- (-2.25,10.75);
\draw  (-6.5,10.5) rectangle (-2.5,10.25);
\draw [ fill={rgb,255:red,246; green,245; blue,244} ] (-8.5,11.25) rectangle  node {\Large ERMS} (-6.25,9.5);
\draw [ rotate around={-75:(5.5,8.75)}] (5.5,8.75) ellipse (0.25cm and 1.5cm);
\draw [short] (11,12.75) -- (11,13);
\draw [short] (11.25,13) -- (11.25,10.75);
\draw [short] (11.25,10.75) -- (8.5,10.75);
\draw [short] (8.5,10.75) -- (8.5,4.25);
\draw [short] (8.5,4.25) -- (5.5,4.25);
\draw [short] (-3.25,6.75) -- (-3.75,6.75);
\draw [short] (-2.75,8) -- (-2.75,4.75);
\draw [short] (-3.25,4.75) -- (-2.75,4.75);
\draw [line width=0.7pt, short] (-7.75,6) -- (-7.75,3.25);
\node [font=\Large] at (1,8.75) {Distance variation};
\draw [short] (-3.25,6.75) -- (-3.25,8);
\end{circuitikz}
}%

\caption{Schematic of the vacuum chamber setup for cathodic arc plasma diagnostics and thin film deposition. (a) Arc power supply, (b) EM-coil power supply, (c) Langmuir probe and QCM, (d) energy-resolving mass spectrometer (ERMS). The delay generator (a–d) synchronizes the arc power supply, EM-coil activation, and diagnostic tools.}


\label{fig:vacuumchamber}
\end{figure}

\subsection{Power \& Triggering}
The arc power supply operated in pulsed mode at a frequencies between 0.2--5~Hz with a pulse width of 1~ms. The EM-coil was activated 200~$\mu$s before arc ignition and lasts 1.5~ms, to ensure steady-state magnetic field conditions. A delay generator (SRS DG645) served as the master clock, distributing triggers to:
\begin{itemize}[noitemsep]
    \item the arc power supply (channel~\textbf{a}),
    \item the EM-coil power supply (channel~\textbf{b}),
    \item the diagnostics (ERMS, QCM, and Langmuir probe; channels~\textbf{c--d}).
    \item the oscilloscope is triggered on the rising edge of the arc power supply unit (PSU) voltage (Channel \textcolor{mypurple}{(1)})
\end{itemize}
This setup enabled time-resolved measurements of ion flux and energy, fully synchronized with plasma generation.

\begin{figure}[h]
	\centering
	\includegraphics[width=0.85\textwidth]{"Figures/experimental methods/Pulse waveform"}
	\caption{Example Pulse waveform with the triggering timings (a) and (b) for the Arc-PSU and the EM-coil PSU marked with the orange dashed line}
	\label{fig:pulse-waveform}
\end{figure}

To approximate the magnetic field generated within the EM-coil solenoid, the following equation was utilized:

\begin{align}
	B = \frac{\mu_0 N I}{L}
\end{align}
with the length of the solenoid $L = 0.02 m$, the number of turns in the solenoid $N = 5$ and the vacuum permeability $\mu_0 = 1.256 \cdot 10^{-6} \frac{T\cdot m}{A}$. The electrical current value was determined by the peak current recorded with the oscilloscope, typically observed around the 0 ms point, as illustrated in Fig. \ref{fig:pulse-waveform}. This approach was adopted because the shape of the current curve varies significantly depending on the input voltage and the resulting current. Figure \ref{fig:pulse-waveform} depicts the current curve achieved for a 250V input. Additionally the current curve for a 100V input is provided in the Appendix (Fig. \ref{fig:pulse-waveform-appendix}).


\section{In situ Diagnostics}

\subsection{Ion-current Probe- Langmuir probe}\label{sec:langmuir_probe}


The in-house-built ion collector probe (Figure~\ref{fig:langmuir_probe}) was designed to measure the current density of ions (\(J_i\)) in ion saturation mode. The probe consisted of a 5~mm diameter copper stick milled down to resemble a nail, it was then covered with Katpon tape to ensure insulation from the holder assembly \ref{fig:holder assembly}.



\begin{figure}[!ht]
\centering
\resizebox{0.5\textwidth}{!}{%
\begin{circuitikz}
\tikzstyle{every node}=[font=\Huge]
\draw [ fill={rgb,255:red,255; green,120; blue,0} ] (0,17) rectangle (0.75,12.5);
\draw [<->, >=Stealth] (-1.25,17) -- (-1.25,12.5)node[pos=0.5, fill=white]{5 mm};
\draw [<->, >=Stealth] (0,11.75) -- (25.5,11.75)node[pos=0.5, fill=white]{60 mm};
\draw [ fill={rgb,255:red,255; green,120; blue,0} ] (22.5,15) rectangle (25,14.5);
\draw [ fill={rgb,255:red,255; green,120; blue,0} ] (0.75,16.5) rectangle (22.75,13);
\draw [ fill={rgb,255:red,255; green,190; blue,111} ] (0.75,16.75) rectangle (22,12.75);
\end{circuitikz}
}%
\caption{In-house built ion collector probe wrapped in Kapton tape for electrical insulation from the aluminum assembly holder. The probe includes an attachment point for a screw terminal connector, enabling connection to the ion probe circuit (Figure \ref{fig:probe_circuit})}
\label{fig:langmuir_probe}
\end{figure}

To guarantee full ion collection, the probe was negatively biased at \(V_b = -80\)~V, as determined in sec. \ref{sec:ionprobe80V}.


\begin{figure}[!ht]
\centering
\resizebox{0.8\textwidth}{!}{%
\begin{circuitikz}
\tikzstyle{every node}=[font=\LARGE]
\draw [ line width=0.5pt](5.75,13.25) to[short] (7.25,13.25);
\node at (7.25,13.25) [circ] {};
\draw [ line width=0.5pt](7.25,13.25) to[short] (7.75,13.25);
\node at (7.75,13.25) [circ] {};
\draw [ line width=0.5pt](7.75,14.25) to[short] (7.75,13.25);
\draw [ line width=0.5pt](7.75,13.25) to[short] (7.75,12.5);
\draw [line width=0.5pt](7.75,14.25) to[C,l={ \normalsize 22nF}] (9.75,14.25);
\draw [line width=0.5pt](7.75,12.5) to[C,l={ \normalsize 0.5$\mu$F}] (9.75,12.5);
\draw [ line width=0.5pt](9.75,14.25) to[short] (9.75,12.5);
\draw [ line width=0.5pt](9.75,13.25) to[short] (10.25,13.25);
\draw [ line width=0.5pt](10.25,13.25) to[short] (12.25,13.25);
\node at (10.25,13.25) [circ] {};
\node at (9.75,13.25) [circ] {};
\draw [ line width=0.5pt](7.25,13.25) to[short] (7.25,12.25);
\draw [ line width=0.5pt](10.25,13.25) to[short] (10.25,12.25);
\draw [ line width=0.5pt](7.25,12.25) to[european resistor,l={ \normalsize 1k$\Omega$}] (7.25,11);
\draw [ line width=0.5pt](10.25,12.25) to[european resistor,l={ \normalsize 10k$\Omega$}] (10.25,11);
\draw [ line width=0.5pt](7.25,11) to[D] (7.25,10.25);
\draw [ line width=0.5pt](10.25,10.25) to[D] (10.25,11);
\draw [ line width=0.5pt](7.75,9.75) to[american controlled voltage source,l={ \normalsize 80 V}] (9.75,9.75);
\draw [ line width=0.5pt](10.25,10.25) to[short] (10.25,9.75);
\draw [ line width=0.5pt](10.25,9.75) to[short] (9.75,9.75);
\draw [ line width=0.5pt](7.25,10.25) to[short] (7.25,9.75);
\draw [ line width=0.5pt](7.25,9.75) to[short] (7.75,9.75);
\node at (12.25,13.25) [circ] {};
\draw [ line width=0.5pt](12.25,13.25) to[european resistor,l={ \normalsize 400$\Omega$}] (12.25,10.25);
\draw [ line width=0.5pt ] (14.5,12.5) rectangle  node {\large Oscilloscope} (17.5,11.25);
\draw [ line width=0.5pt](12.25,13.25) to[short] (14.75,13.25);
\draw [ line width=0.5pt](14.75,13.25) to[short] (14.75,12.5);
\draw [ line width=0.5pt](14.75,11.25) to[short] (14.75,10.25);
\draw [ line width=0.5pt](14.75,10.25) to[short] (13.25,10.25);
\draw [line width=0.5pt](13,10.25) to (13,9.75) node[ground]{};
\draw [ line width=0.5pt](12.25,10.25) to[short] (13.25,10.25);
\draw [ line width=0.6pt ] (5.75,14.25) circle (0.75cm) node {\normalsize Probe} ;
\draw [line width=0.5pt, short] (5.75,13.25) -- (5.75,13.5);


\end{circuitikz}
}%
\caption{Schematic of the ion-flux probe circuit. The $\SI{400}{\ohm}$ resistor converts ion current to voltage, while the $\SI{0.5}{\micro F}$ capacitor and $\SI{400}{\ohm}$ resistor form a high-pass filter with a $\SI{795}{Hz}$ cutoff.}
\label{fig:probe_circuit}
\end{figure}


The circuit incorporates a $\SI{0.5}{\micro F}$ capacitor in series with the $\SI{400}{\ohm}$ resistor, forming a high-pass filter with a cutoff frequency of $\SI{795}{Hz}$. This configuration effectively blocks DC and low-frequency noise, ensuring that only plasma fluctuations above $\SI{795}{Hz}$ are recorded. The high-pass characteristic is essential for isolating the dynamic ion current signal from any static offsets or drift.\\

To maintain a stable bias voltage, a $\SI{22}{nF}$ capacitor and $\SI{1}{k\ohm}$ resistor are included in the bias supply line. This combination acts as a low-pass filter, smoothing the bias voltage and minimizing high-frequency ripple.\\

The processed voltage signal is recorded using a 20~MHz bandwidth oscilloscope. The ion flux $\Gamma_i$ is then calculated from the recorded voltage $V$ as:
\begin{equation}
    \Gamma_i = \frac{V}{e A R},
\end{equation}

where $A = \SI{19.63}{mm^2}$ is the probe area, $e$ is the elementary charge, and $R = \SI{400}{\ohm}$. This setup ensures accurate measurement of the ion flux while minimizing the impact of noise and DC offsets.




\subsection{Quartz Crystal Microbalance}\label{section:QCM}

A quartz crystal microbalance (QCM) was employed to measure the amount of material deposited during cathodic arc sputtering. The configuration used in this work was the INFICON Cool Drawer\texttrademark\ with a single drawer in standard orientation. The sensor is water-cooled to ensure thermal stability, and a 14 mm diameter, 6 MHz AT-cut quartz crystal was operated with a SQM-160 controller for electronic readout. \\

\begin{figure}[h]
    \centering
    \includegraphics[width=0.7\linewidth]{Figures/experimental methods/QCM_head.png}
    \caption{Sensor head of the INFICON Cool Drawer\texttrademark{} Quartz Crystal Microbalance (QCM) used for in-situ mass deposition monitoring during cathodic arc sputtering. The assembly includes a water-cooled housing, a 14~mm diameter AT-cut quartz crystal (6~MHz), and electrode leads for connection to the SQM-160 controller. (Schematic adapted from INFICON STP file, available at \url{https://www.inficon.com/en/products/thin-film-technology/cool-drawer-single-sensor}).}
    \label{fig:QCM_sensor_head}
\end{figure}


The measurement principle follows the Sauerbrey equation \cite{Sauerbrey_1959}, which relates the change in resonance frequency of the quartz crystal to the deposited mass:  

\begin{equation}\label{eq.sauerbrey}
    \Delta m = \frac{N_{\text{AT}}\cdot d_q \cdot \pi r^2}{F_q^2} \cdot \Delta F_c 
    = 18.8146023 \cdot 10^{-9} \frac{g}{Hz} \cdot \Delta F .
\end{equation}

Here $d_q = \SI{2.649}{\frac{g}{cm^3}}$ is the quartz density, the exposed area of the QCM is $r = 0.7 cm$, $N_{\text{AT}} = \SI{166100}{Hz\ cm}$ the frequency constant of the AT-cut, $F_q = \SI{6}{MHz}$ the uncoated resonance frequency, and $\Delta F$ the measured frequency shift.\\

The Sauerbrey relation is accurate as long as $\Delta F \lesssim 0.05 F_q$ (about \SI{0.3}{MHz} for a 6 MHz crystal). For larger mass loadings, the linear approximation fails and the Z-match\texttrademark\ technique is used. This method, introduced by Lu and Lewis in 1972 on the basis of Miller and Bolef’s theoretical treatment \cite{Miller1968,Lu1972}, incorporates the acoustic properties of both the quartz and the deposited film via the acoustic impedance ratio  

\begin{equation}
    Z = \left(\frac{d_q \, \mu_q}{d_f \, \mu_f}\right)^{1/2},
\end{equation}

with $d$ and $\mu$ denoting the density and shear modulus of quartz ($q$) and film ($f$), respectively \cite{SQM160_manual}. In practice, the controller applies a correction function $f(Z)$ to the Sauerbrey relation,  

\begin{equation}
    m_f = \frac{N_{\text{AT}} \, d_q \, \pi r^2}{F_q^2} \cdot \Delta F \cdot f(Z),
\end{equation}


which compensates for the acoustic mismatch and extends the validity of thickness determination up to $\sim 0.4 F_q$.\\ 

In the present experiments, the observed frequency shifts ranged from about \SI{1}{Hz} to \SI{50}{Hz}. With the SQM-160 resolution of approximately \SI{0.03}{Hz} at \SI{6}{MHz}, even the smallest shifts were well above the noise floor, yet orders of magnitude below the Sauerbrey breakdown limit. The Sauerbrey approximation was therefore fully sufficient, and Z-match corrections were not required.

\subsection{Comparability with QCM Measurements}

The ion collector probe, with its 5 mm diameter, was mounted through a precision-milled pass-through hole in the aluminum mounting block, while the QCM was secured in a dedicated cutout and fixed via screws.

\begin{figure}[ht]
  \centering
  \includemedia[
    width=0.8\linewidth,
    activate=pageopen,
    passcontext,
    3Dmenu,
    3Dcoo=4.639472961425781 -21.05224609375 27.223785400390625,
    3Dc2c=92.426 9.840 -436.780,
    3Droo=200,
  ]{
    \includegraphics[width=0.8\linewidth]{Figures/experimental methods/mounting assembly thingy.png}
  }{Figures/experimental methods/fun try out(1).u3d}
    \caption{Holder Assembly for In-Situ Plasma Diagnostics: Integrated QCMs and Langmuir Ion Collector Probe (interactive 3D model; static preview shown in non-Adobe viewers).}
    \label{fig:holder assembly}
\end{figure}

This configuration ensured rigid mechanical alignment between the two diagnostics. The exposed area of the quartz crystal (8 mm diameter) was selected to encompass the probe’s collection area, enabling spatially resolved comparisons of ion current density and deposited mass.\\

This design accounts for the radial gradients in plasma density and ion charge state distribution, which are inherent to expanding cathodic arc plasmas \cite[Chap. 6.2]{cathodic_arcs}. By positioning the probe right next to the QCM, the ion flux measurements directly reflect the plasma conditions governing deposition on the crystal surface.




\subsection{Quadrupole Mass Spectrometer}\label{sec:QMS}

A quadrupole mass spectrometer (QMS, Hiden EQP 1000) was used to measure the ion energy distribution functions (IEDFs) and charge-state-resolved fluxes of plasma species generated during pulsed cathodic arc deposition. The system utilizes an electrostatic energy analyzer with a quadrupole mass filter to separate ions by their kinetic energy and mass-to-charge ratio (\(m/q\)).\\

Ions enter the QMS through a \SI{50}{\micro\meter} sampling orifice and are first transported to the energy analyzer, where their kinetic energy \(E_i\) is selected according to the relationship:
%
\begin{equation}\label{eq:ion_energy}
    E_i = \left(V_{\text{ENERGY}} + \frac{R}{d} V_{\text{PLATES}} - V_{\text{AXIS}}\right) n \times e.
\end{equation}
%

Here, \(V_{\text{ENERGY}}\) and \(V_{\text{AXIS}}\) are opposing potentials applied to the analyzer, \(R\) is the mean radius of the cylindrical sector, \(d\) is the plate separation, \(V_{\text{PLATES}}\) is the potential difference across the sector plates, and \(n \times e\) is the total charge of the ion \cite{hiden_eqp_manual}. The selected ions are then injected into the quadrupole mass filter, where a combination of AC and DC electric fields creates a stability region dependent on \(m/q\), described by the Mathieu equations \cite{dawson_quadrupole_1997}. Only ions with trajectories stable in both radial and axial directions reach the detector. The potential in the quadrupole is described by:
%
\begin{equation}
    V(x,y,t) = \frac{U_0 \cos(\omega t)}{r_0^2} (x^2 - y^2),
\end{equation}
%
where \(U_0\) is the amplitude of the AC voltage, \(\omega\) is the angular frequency, and \(r_0\) is the field radius. The stability of ion motion is determined by the dimensionless parameters:
%
\begin{equation}
    a = \frac{8 e U_{\text{DC}}}{m r_0^2 \omega^2}, \quad q = \frac{4 e U_0}{m r_0^2 \omega^2},
\end{equation}
%
with \(U_{\text{DC}}\) as the superimposed DC voltage. For a given \(m/q\), stable transmission occurs only within specific \((a,q)\) regions, enabling mass separation \cite{dawson_quadrupole_1997, march_quadrupole_1989}.\\

\begin{figure}[h]
	\centering
	\includegraphics[width=0.9\linewidth]{"Figures/experimental methods/hiden eqp"}
	\caption{ERMS, Hiden EQP HE 1000: (1) Sampling Orifice, (2) Electron Impact Ion Source, (3) Transfer Ion Optics, (4) Quadrupole Lens, (5) Energy Filter, (6) Decelerating Lens, (7) Quadrupole Mass Filter, (8) Detector, (9) Differential Pump Port \cite{hidenanalytical}}
	\label{fig:hiden-eqp}
\end{figure}


To obtain the ion energy distribution functions (IEDFs), the energy-to-charge (E/Q) distributions were measured for different mass-to-charge ratios (M/Q). The energy distributions for ions of different charge states were derived by multiplying the E/Q values by the corresponding charge state number \(Q\). This correction accounts for the charge-dependent scaling of ion energies.\\

To reduce interference from the arc's magnetic field, the QMS was equipped with a grounded, mu-metal shield \cite{RN5}. 

%The \(m/q\) range was calibrated using N\(_2^+\) as reference ions, with typical operating parameters set to \(U_0 = \SI{100}{V}\), \(U_{\text{DC}} = \SI{5}{V}\), and \(\omega/2\pi = \SI{1.2}{MHz}\). Under these settings, the system provided a mass resolution of \(\Delta m/m \approx 0.5\) (FWHM) for ions in the \(\SI{0}{--}\SI{150}{eV}\) energy range \cite{hiden_eqp_manual}. The energy resolution was adjusted to \(\Delta E/E \approx 0.02\) by tuning \(V_{\text{PLATES}}\) and optimizing the alignment of the orifice and detector.
%%%

IEDFs were measured using a double trigger acquisition scheme synchronized with the arc pulses, which had a \(\SI{1}{ms}\) duration and a \(\SI{5}{Hz}\) repetition rate. For each \(m/q\) value, two \(\SI{20}{ms}\) acquisition windows were recorded, activated \(\SI{10}{ms}\) before the onset of the pulse. The combined \(\SI{40}{ms}\) of data for each point were averaged to obtain the final IEDF. Measurements were performed for charge states \(1^+\), \(2^+\), and \(3^+\) of aluminum ions, and for charge states \(1^+\), \(2^+\), \(3^+\), and \(4^+\) of titanium ions. For nitrogen ion species (N and N\(_2\)), only the \(1^+\) ionization level was measured. This was achieved by scanning \(V_{\text{ENERGY}}\) while fixing the quadrupole mass filter to the corresponding \(m/q\) values.

\begin{table}[h]
\centering
\begin{tabular}{c|cccc|}
\cline{2-5}
                                 & \multicolumn{4}{c|}{Molar mass over charge ratio of:}                                     \\ \hline
\multicolumn{1}{|c|}{Ionization} & \multicolumn{1}{c|}{Al}   & \multicolumn{1}{c|}{Ti}     & \multicolumn{1}{c|}{N}  & N$_2$ \\ \hline
\multicolumn{1}{|c|}{1+}         & \multicolumn{1}{c|}{27}   & \multicolumn{1}{c|}{47.867} & \multicolumn{1}{c|}{14} & 28    \\ \hline
\multicolumn{1}{|c|}{2+}         & \multicolumn{1}{c|}{13.5} & \multicolumn{1}{c|}{23.933} & \multicolumn{1}{c|}{-}  & -     \\ \hline
\multicolumn{1}{|c|}{3+}         & \multicolumn{1}{c|}{9}    & \multicolumn{1}{c|}{15.955} & \multicolumn{1}{c|}{-}  & -     \\ \hline
\multicolumn{1}{|c|}{4+}         & \multicolumn{1}{c|}{-}    & \multicolumn{1}{c|}{11.966} & \multicolumn{1}{c|}{-}  & -     \\ \hline
\end{tabular}
\end{table}

%The integrated ion flux \(\Gamma_i\) was determined by normalizing the IEDFs to the plasma density, as measured by a Langmuir probe (\ref{sec:langmuir_probe}). The observed charge-state distributions of Ti\(^+\) and Al\(^+\) ions aligned with earlier findings for cathodic arc plasmas \cite{RN6,unutulmazsoy}.


\section{Ex situ Measurements}\label{sec:film_characterization}

\subsection{X-ray Diffraction (XRD)}\label{sec:XRD}

X-ray diffraction (XRD) was employed to analyze the crystallographic structure of the thin films deposited during the experiments. XRD is a non-destructive technique that provides detailed information about the crystalline phases present in the material, as well as their lattice parameters, crystallite size, and strain.\\

Due to the small film thickness, out-of-plane diffraction techniques were used to enhance the film signal relative to the substrate. Two measurement geometries are employed \cite{mitsunaga2009}:

\begin{itemize}
	\item \textbf{Symmetrical reflection ($2\theta/\theta$ scan):} Both incident and diffracted beams make equal angles with the sample surface. This geometry probes lattice planes parallel to the substrate and is suitable for textured films, but substrate peaks can obscure weak film signals.
	
	\item \textbf{Asymmetrical reflection (thin-film method):} The incident beam is fixed at a small grazing angle $\alpha$, while the detector scans in $2\theta$. This reduces the X-ray penetration depth from tens of micrometers to a few micrometers, greatly enhancing sensitivity to thin films \cite{mitsunaga2009}.
\end{itemize}

\begin{figure}
	\centering
	\includegraphics[width=0.75\linewidth]{"Figures/experimental methods/xrd_in+outofplae"}
	\caption{Schematic of out-of-plane diffraction geometries: symmetrical reflection (left) and asymmetrical reflection (right) for thin film analysis. Taken from \cite{mitsunaga2009}.}
	\label{fig:xrdinoutofplane}
\end{figure}

The XRD measurements were performed using a \textcolor{red}{model name of xrd machine} diffractometer equipped with \textcolor{red}{non monochromatic $Cu_{alpha}$ source with smt smt wavelength-} The asymmetrical reflection scans were performed with an \textcolor{red}{add incidence angle and speed and step size used}\\

The crystallographic structure was determined by analyzing the diffraction patterns. The Bragg equation was used to identify the crystalline phases:
\begin{equation}
	2d \sin(\theta) = n\lambda,
\end{equation}
where $d$ is the spacing between atomic planes, $\theta$ is the diffraction angle, $n$ is an integer, and $\lambda$ is the X-ray wavelength.\\



\subsection{X-ray Reflectometry (XRR)}\label{sec:XRR}
\textcolor{red}{need to look into before writing about it, more precisely about the specs of the device}
\iffalse
X-ray reflectometry (XRR) is a non-destructive technique used to determine the thickness, density, and interfacial roughness of thin films and multilayer structures \cite{tolan1999,holy1999}. The method works by measuring the intensity of X-rays reflected from a sample surface as a function of the incident angle, typically in the range of grazing incidence (0.1--5$^\circ$). 

When X-rays strike a flat surface at shallow angles, total external reflection occurs below a critical angle $\theta_c$, which is related to the electron density of the material. Above the critical angle, X-rays penetrate the film and reflect from interfaces, creating interference patterns (Kiessig fringes) whose period is directly related to the film thickness \cite{kiessig1931}. For a single-layer film, the thickness $d$ can be estimated from the angular spacing $\Delta\theta$ between interference fringes using:
\begin{equation}
	d = \frac{\lambda}{2\Delta\theta}
	\label{eq:xrr_thickness}
\end{equation}
where $\lambda$ is the X-ray wavelength.

XRR is particularly well-suited for films in the range of 1--150~nm, with thickness accuracy on the order of 0.1--0.2~nm and roughness sensitivity down to a few angstroms \cite{daillant2009}. Unlike optical methods, XRR is a first-principles technique requiring no calibration and is applicable to crystalline, polycrystalline, and amorphous materials. The measured reflectivity curves are typically analyzed using recursive Parratt formalism combined with roughness corrections, with fitting parameters including layer thicknesses, densities (determining refractive index), and interfacial roughnesses \cite{parratt1954}.

In this work, XRR measurements were performed to verify film thicknesses determined by QCM and profilometry, as well as to characterize the density and surface roughness of deposited TiAlN films under various deposition conditions.
\fi



\subsection{Scanning Electron Microscopy (SEM)}\label{sec:SEM}
\textcolor{red}{need to look into before writing about it, more precisely about the specs of the device}
\iffalse

Scanning electron microscopy (SEM) was employed to characterize the surface morphology, cross-sectional structure, and microstructural features of the deposited films. SEM uses a focused electron beam (typically 1--30~keV) that is raster-scanned across the sample surface, generating various signals through electron-specimen interactions \cite{goldstein2017}. The primary signals detected include secondary electrons (SE) for high-resolution topographic imaging and backscattered electrons (BSE) for compositional contrast.

Secondary electrons, emitted from the near-surface region ($\sim$5--50~nm depth), provide information about surface texture and topography with typical resolutions of 1--10~nm depending on the instrument and operating conditions \cite{reimer1998}. Backscattered electrons, which retain most of their incident energy, are sensitive to atomic number ($Z$-contrast), allowing different phases or compositional variations to be distinguished. The intensity of BSE signal increases with increasing atomic number, making heavier elements appear brighter.

For this study, samples were prepared by cleaving or cutting silicon wafer substrates to expose cross-sections of the deposited films. Conductive coatings were not required for the metallic TiAlN films. SEM imaging was performed to:
\begin{itemize}[noitemsep]
	\item Characterize surface morphology and the presence of macroparticles typical of cathodic arc deposition
	\item Measure film thickness from cross-sectional views to validate QCM measurements
	\item Examine film uniformity and interface quality with the substrate
	\item Identify columnar growth structures and grain morphology
\end{itemize}

Additionally, energy-dispersive X-ray spectroscopy (EDS) was used in conjunction with SEM to perform elemental analysis, confirming the Ti:Al ratio and nitrogen incorporation in the films \cite{williams2009}.
\fi
\subsection{Energy-dispersive X-ray spectroscopy (EDX)}\label{sec:EDX}
\textcolor{red}{need to look into before writing about it, more precisely about the specs of the device}
\subsection{Profilometry}\label{sec:profil}

Stylus profilometry was used to measure film thickness by mechanically tracing the surface topography using a diamond-tipped stylus. The technique provides direct measurement of step heights between masked and deposited regions, making it particularly useful for verifying film thickness values obtained by QCM \cite{poon1995}.\\


In profilometry, a stylus with a small tip radius is dragged across the sample surface with a controlled force of 3~mg while its vertical displacement is monitored electromagnetically. The resulting trace provides a profile of the surface from which the step height (film thickness) with vertical resolution down to $\sim$ 1~nm can be extracted.\\


%The Dektak stylus profilometer used in this study has a measurement range from 6.5~nm to 800~$\mu$m with repeatability of 0.4~nm. For thickness measurements, silicon substrates were partially masked with a marker line, which created a well-defined step edge during deposition. This marker line was subsequently removed with Isopropanol in a ultrasonic bath. Multiple scans across each step were performed to ensure reproducibility, and the average thickness was calculated from at least three different positions on each sample.\\


One limitation of contact profilometry is the potential for stylus-induced damage on very soft films, though this was not a concern for the hard TiAlN coatings investigated here. The technique is complementary to XRR, with profilometry providing rapid, direct thickness measurements while XRR offers higher accuracy for very thin films and additional information on density and roughness \cite{windover1999}.






\section{Data Processing}

Experimental data were processed using custom Python scripts to ensure consistency and reproducibility. A central Excel logbook served as the reference for all measurements, each identified by a unique suffix and linked to its corresponding data files. The logbook recorded experimental parameters such as date, distance, pressures, MFC flow rate, cryopump position, power supply settings, and pulse characteristics, as well as initial and final QCM frequencies for deposited mass determination. Associated oscilloscope waveforms were stored as CSV files for ion current analysis.\\


A Python script automated data handling by matching logbook entries to raw data, averaging ion current waveforms over multiple pulses to reduce noise when possible, and compiling all parameters into a unified dataset. This ensured uniform processing and efficient preparation for subsequent analysis.\\


QMS data were evaluated with a separate Python script that integrated raw spectra, applied mass transmission corrections to account for detection biases, and extracted parameters such as mean ion energy, charge state distribution, and potential and kinetic energy components for each species (different ionization levels of the four atoms measured). The processed results were visualized and exported into a structured CSV file for detailed examination of the ion energy distributions.



\section{Error Handling}
\subsection{Mass Spectrometry Measurements}

In mass spectrometry measurements, particularly in the context of cathodic arc processes, the standard deviation is crucial for characterizing data variability. The non-stationary nature of cathode spots in cathodic arcs leads to significant fluctuations in ion flux and charge composition from pulse to pulse \cite{cathodic_arcs}. Calibrating mass spectrometry measurements can be challenging due to these fluctuations, as random errors at each standard point used to determine the calibration curve can lead to a distribution of values for the same observed response for the unknown \cite{massspec_err}. As a result, achieving optimal calibration in such conditions may not be straightforward, and the accuracy of the calibration could be affected.

In this study, while efforts were made to calibrate the mass spectrometer accurately, the inherent difficulties associated with the calibration process in cathodic arc environments suggest that the calibration might not have been optimal. To account for this potential source of error, the impact of truncating the measurement artifact (e.g., signal noise or background interference) at different points was evaluated, and its effect on the data was analyzed in detail. This analysis will be further discussed in Section \ref{massspec_results}.

As detailed in Appendix \ref{appendix:code}, the standard deviation of energy measurements is calculated using the \verb|calculate_energy_stats| function, providing both average energy and standard deviation. This statistical approach offers a visual representation of measurement variability through error bars. The use of standard deviation in mass spectrometry is well-documented in scientific literature. For instance, standard deviation is commonly used to represent uncertainties in measurements and to describe the scatter among measured data points \cite{basic_massspec}. This statistical approach is crucial for characterizing data variability and ensuring the accuracy of the measurements.

\subsection{QCM and Ion Current Probe Measurements}\label{sec:ion/qcm_error}

For the ion current probe measurements, the standard deviation of the averaged measurements is calculated to characterize the variability in the data. This variability within a pulse can be attributed to several factors: the fluctuations in the plasma potential at the beginning of the pulse, charge exchange reactions between ions and neutrals and variations in the arc current and pulse parameters. These factors can cause fluctuations in the ion current during the pulse, leading to the observed variability \cite{evolution_pulse}. This kind of instability within the pulse will be henceforth shown with the help of error bars.

However, certain sources of error specific to the ion probe must be considered. The design of the ion probe can influence the measurements due to sheath effects, where the sheath around the probe can expand based on the probe's size and the biasing applied \cite[Appendix A.2]{cathodic_arcs}. Additionally, secondary electron emissions from the probe surface can affect the ion current measurements \cite[Chap. 8.2]{cathodic_arcs}. These emissions can be caused by interactions between the ions and the probe surface leading to inaccuracies in the measurements. As stated before, each pulse varies in terms of energies and amount of particles; these variations are investigated briefly in Sec.~\ref{sec:errorbar}, but cannot be considered for every data point.

The QCM measurements are subject to inaccuracies in terms of the error in the resonance frequency measurements, which is based on the manufacturer's specifications. The given measurement inaccuracy is $\Delta f = 0.03$~Hz, which will be taken for error propagation \cite{SQM160_manual}. Additionally, the relative position of the QCM sensor with respect to the ion flux affects the accuracy of the measurements especially at close distances.

\subsection{Error Propagation Analysis}\label{sec:error_propagation}

To properly characterize the uncertainties in the derived quantities, a comprehensive error propagation analysis was performed. The primary measured quantities with associated uncertainties are: the ion current $I_{\text{ion}}$ with standard deviation $\sigma_{I}$, the deposited mass $m$ measured by QCM with uncertainty $\sigma_m$, and the mean charge state $\bar{Q}$ with standard deviation $\sigma_Q$ determined from mass spectrometry measurements. These independent measurements are combined to calculate particle fluxes, necessitating careful propagation of uncertainties through the calculation chain.

\subsubsection{Ion Flux from Current Measurements}
The ion flux in mass units $\Phi_{\text{ion}}$ (in $\mu$g/cm$^2$/s) is calculated from the measured ion current according to:
\begin{equation}
	\Phi_{\text{ion}} = \frac{I_{\text{ion}}}{A \cdot \bar{Q} \cdot e}
	\label{eq:ion_flux_mass}
\end{equation}
where $e$ is the elementary charge, $A$ is the collection area of the ion probe and $\bar{Q}$ is the mean charge state.\\


For error propagation, considering the independent uncertainties in $I_{\text{ion}}$, $\bar{Q}$, the standard Gaussian error propagation formula gives \cite{taylor1997}:
\begin{equation}
	\left(\frac{\sigma_{\Phi_{\text{ion}}}}{\Phi_{\text{ion}}}\right)^2 = \left(\frac{\sigma_I}{I_{\text{ion}}}\right)^2 + \left(\frac{\sigma_Q}{\bar{Q}}\right)^2
\end{equation}

This shows that the relative uncertainties add in quadrature, which is typical for multiplicative error propagation.

\subsubsection{Atomic Flux from QCM Measurements}
The atomic flux $\Phi_{\text{atom}}$ (in atoms/cm$^2$/s) is determined from the mass flux measured by the QCM:
\begin{equation}
	\Phi_{\text{atom}} = \frac{\Phi_{\text{mass}} \cdot N_A}{M_{\text{eff}}}
	\label{eq:atomic_flux}
\end{equation}
where $\Phi_{\text{mass}} = m/(A_{\text{QCM}} \cdot t)$ is the mass flux and $N_A = 6.022 \times 10^{23}$~mol$^{-1}$ is Avogadro's constant. Following the same approach: \textcolor{red}{this depends what kind of data i get from EDX, else just standard error}
\begin{equation}
	\left(\frac{\sigma_{\Phi_{\text{atom}}}}{\Phi_{\text{atom}}}\right)^2 = \left(\frac{\sigma_m}{m}\right)^2 + \left(\frac{\sigma_{M_{\text{eff}}}}{M_{\text{eff}}}\right)^2
	\label{eq:rel_error_atomic_flux}
\end{equation}

The QCM mass uncertainty $\sigma_m$ is calculated from the frequency measurement uncertainty $\Delta f = 0.03$~Hz using the Sauerbrey relation.




