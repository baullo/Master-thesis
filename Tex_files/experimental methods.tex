% !TeX root = main.tex
\chapter{Experimental Methodology}\label{chap:methods}

This chapter describes the experimental apparatus, diagnostic techniques, and data processing methods used to investigate the correlation between ion flux and mass deposition rate in pulsed filtered cathodic arc deposition of Ti--Al--N thin films. The approach combines three complementary diagnostics: a Langmuir probe for ion current measurements, a quartz crystal microbalance (QCM) for real-time mass deposition monitoring, and an energy-resolving mass spectrometer (ERMS) for charge-state-resolved ion energy distributions.


\section{Experimental  Setup}

\subsection{Vacuum \& Gas Infrastructure}

The vacuum chamber was evacuated using a two-stage pumping system consisting of a dry rotor vacuum pump (Leybold ECODRY+) for roughing and a cryogenic pump (Leybold COOLVAC) for the main pumping line, achieving a base pressure on the order of $1 \times 10^{-5}$~Pa with the cryopump fully open. To control the working pressure during reactive deposition, the cryopump gate valve was partially closed to position 345 (approximately 34.5\% open), which reduced the effective pumping speed and increased the base pressure to approximately $1 \times 10^{-4}$~Pa. This configuration allowed finer control of the nitrogen partial pressure.\\
Nitrogen gas (N$_2$, 99.999\% purity) was introduced via a mass flow controller \textcolor{red}{(MKS [MODEL NUMBER]}), with chamber pressures monitored using a \textcolor{red}{[GAUGE MODEL]}. The nitrogen flow rate of 1 to 12.8 ccms leads to chamber pressure ranges between 0.025 and 0.3 Pa.

Plasma was generated using a water-cooled cylindrical anode and a rod cathode with composition 75~wt.\% Ti; 25~wt.\% Al (corresponding to 62.8~at.\% Ti; 37.2~at.\% Al). The cathode had a diameter of 6.35~mm and length of 38.1~mm. The expanding plasma was guided through a 90$^\circ$ curved magnetic macroparticle filter connected in series with the arc power supply, which simultaneously powered both the arc and the filter coils.

\subsection{Power Circuits and Arc Operation}

The arc power supply \textcolor{red}{([MODEL])} operated in pulsed DC mode, delivering arc currents up to 450~A at pulse frequencies ranging from 0.2 to 5~Hz, with a typical pulse width of 1~ms. The arc current was monitored using a \textcolor{red}{[CURRENT PROBE MODEL]} connected to a Tektronix MSO64 oscilloscope. The cathode voltage relative to ground and anode voltage relative to ground were measured on additional channels to determine the burning voltage during the pulse.\\

A coil wound around the cylindrical anode (EM-coil) was used to enhance the ion charge states leaving the source through magnetic confinement. This coil was connected to a separate pulsing unit ([MODEL]) capable of delivering currents up to 850~A. The EM-coil was triggered 200~$\mu$s before arc ignition to ensure the magnetic field reached steady-state conditions before plasma generation. The coil current was monitored using a \textcolor{red}{[CURRENT PROBE MODEL]}, and the pulse duration was offset with 200~$\mu$s to stabilize the magnetic field before and during the arc pulse.\\

The magnetic field strength within the EM-coil solenoid was estimated using:
\begin{equation}
	B = \frac{\mu_0 N I}{L}
\end{equation}
where $L = 0.02$~m is the solenoid length, $N = 5$ is the number of turns, $\mu_0 = 1.256 \times 10^{-6}$~T\,m\,A$^{-1}$ is the vacuum permeability, and $I$ is the peak coil current. The peak current was determined from oscilloscope measurements at the beginning of the pulse, as this represents the maximum field strength before resistive losses cause the current to decay. The current waveform drops within a pulse due to the design of the pulsing unit in the power supplies. Figure~\ref{fig:pulse-waveform} shows example current waveforms for both the arc and EM-coil for a 250~V input to the coil power supply. An additional example for 100~V input is provided in Appendix~\ref{fig:pulse-waveform-appendix}.

\begin{figure}[h]
	\centering
	\includegraphics[width=0.85\textwidth]{"Figures/experimental methods/Pulse waveform"}
	\caption[[Pulse waveforms for arc and EM-coil]]{Example pulse waveforms showing the arc current (red) and EM-coil current (black) as functions of time. The orange dashed lines indicate the trigger timing: (a) EM-coil trigger at $t = -0.2$~ms, (b) arc trigger at $t = 0$~ms. The waveforms shown are pulses at 250~V EM-coil input voltage. Distance: 10~cm, pressure: 0~Pa, magnetic field: 0.25~T.}
		\label{fig:pulse-waveform}
\end{figure}

\subsection{Diagnostic Positioning and Synchronization}

A delay generator (Stanford Research Systems DG645) served as the master clock for the experimental system, providing precisely timed trigger signals to:
\begin{itemize}[noitemsep]
	\item the arc power supply (channel~\textbf{a}),
	\item the EM-coil power supply (channel~\textbf{b}),
	\item the energy-resolving mass spectrometer (channel~\textbf{c}).
\end{itemize}


The oscilloscope was triggered on the \textcolor{red}{is it actually} falling edge of the cathode voltage (channel 1), which marks the onset of the arc pulse. Channels 1 and 2 of the oscilloscope measured the cathode-to-ground voltage and anode-to-ground voltage, respectively. Channels 3 and 4 recorded the arc current and EM-coil current via current probes. The ion current was recorded separately, more on that in Section~\ref{sec:langmuir_probe}.\\

The ion current probe and quartz crystal microbalance (QCM) were not triggered by the delay generator. The ion probe signal was recorded directly by the oscilloscope (triggered on the cathode voltage), automatically synchronizing ion current measurements with the arc pulses. The QCM operated continuously, with frequency measurements recorded before and after each deposition run to determine the accumulated mass change over a fixed number of pulses.

For in situ plasma diagnostics, the Langmuir probe and QCM were mounted on a custom movable assembly positioned at distances ranging from 10 to 20~cm from the macroparticle filter exit. The energy-resolving mass spectrometer (ERMS) was mounted on a separate linear feedthrough, allowing independent adjustment of its axial position (distance from the filter exit) for spatial characterization of the plasma. Film depositions were performed using silicon substrates (Si (100)) mounted on a third movable holder, also positioned at controlled distances from the filter exit.\\

The vacuum chamber setup for cathodic arc plasma diagnostics and thin film deposition is shown in Figure~\ref{fig:vacuumchamber}. The green arrow illustrates the trajectory of the plasma plume as it expands from the cathode, passes through the 90$^\circ$ macroparticle filter, and reaches the diagnostics and substrate region. This experimental configuration enabled systematic investigation of plasma properties and film growth as functions of distance, magnetic field strength, and nitrogen pressure, as detailed in the following sections.



\begin{figure}[!ht]
\centering
\resizebox{1\textwidth}{!}{%
\begin{circuitikz}
\tikzstyle{every node}=[font=\Large]
\draw  (4,10) ellipse (0.25cm and 1.5cm);
\draw [ rotate around={-12:(4.25,10)}] (4.25,10) ellipse (0.25cm and 1.5cm);
\draw [ rotate around={-25:(4.5,10)}] (4.5,10) ellipse (0.25cm and 1.5cm);
\draw [ rotate around={-56:(5.25,9.25)}] (5.25,9.25) ellipse (0.25cm and 1.5cm);
\draw [ rotate around={-37:(4.75,9.75)}] (4.75,9.75) ellipse (0.25cm and 1.5cm);
\draw [ rotate around={-69:(5.5,9)}] (5.5,9) ellipse (0.25cm and 1.5cm);
\draw [ rotate around={-88:(5.5,8.5)}] (5.5,8.5) ellipse (0.25cm and 1.5cm);
\draw [ rotate around={-44:(5,9.5)}] (5,9.5) ellipse (0.25cm and 1.5cm);
\draw [ color={rgb,255:red,224; green,27; blue,36} ] (5.5,7.75) ellipse (1cm and 0.25cm);
\draw [short] (7,8.5) -- (7,6.25);
\draw [short] (4,11.5) -- (4,12.75);
\draw [short] (4,12.75) -- (11,12.75);
\draw [short] (5.5,5.25) -- (5.5,4.25);
\draw [ fill={rgb,255:red,222; green,221; blue,218} , line width=1pt ] (5,6.5) rectangle (6,5.5);
\draw [ color={rgb,255:red,224; green,27; blue,36}, short] (11,6.75) -- (5.75,6.75);
\draw [ color={rgb,255:red,224; green,27; blue,36}, short] (11.5,8) -- (5.75,8);
\draw [ fill={rgb,255:red,119; green,118; blue,123} ] (5.25,4.75) rectangle (5.75,6.75);
\node [font=\LARGE] at (4.75,5.25) {};
\draw [short] (7,6.25) -- (6,6.25);
\draw [ color={rgb,255:red,224; green,27; blue,36} ] (5.5,7) ellipse (1cm and 0.25cm);
\draw [ color={rgb,255:red,224; green,27; blue,36} ] (5.5,7.25) ellipse (1cm and 0.25cm);
\draw [ color={rgb,255:red,224; green,27; blue,36} ] (5.5,7.5) ellipse (1cm and 0.25cm);
\draw [line width=1.5pt, short] (5,6.5) -- (5,7.75);
\draw [line width=1.5pt, short] (6,6.5) -- (6,7.75);
\draw [ fill={rgb,255:red,246; green,245; blue,244} ] (-9,7.25) rectangle  node {\Large Ion probe circuit} (-3.75,6);
\draw [ fill={rgb,255:red,246; green,245; blue,244} ] (-5.75,5.5) rectangle  node {\Large QCM} (-3.25,3.75);
\node [font=\LARGE] at (-1.5,11.5) {};
\node [font=\Large] at (-1.5,12) {};
\node [font=\normalsize] at (-0.75,11.5) {};
\draw [ fill={rgb,255:red,246; green,245; blue,244} ] (10.25,14.75) rectangle  node {\Large Arc PSU} (15.25,13);
\draw [ fill={rgb,255:red,246; green,245; blue,244} ] (10.25,6) rectangle  node {\Large EM-coil PSU} (15.25,4.25);
\draw [ color={rgb,255:red,224; green,27; blue,36}, short] (11.5,8) -- (11.5,6);
\draw [ color={rgb,255:red,224; green,27; blue,36}, short] (11,6.75) -- (11,6);
\draw (14.75,6) to[short, -o] (14.75,7) ;
\draw (14.75,13) to[short, -o] (14.75,12) ;
\draw (-7.25,6) to[short, -o] (-7.25,5) ;
\draw [ fill={rgb,255:red,246; green,245; blue,244} ] (-9,3.25) rectangle  node {\Large Oscilloscope} (-3.25,0);
\draw  (10,8) ellipse (0.25cm and 0.5cm);
\draw  (7.75,12.75) ellipse (0.25cm and 0.5cm);
\draw [short] (10,8.5) -- (10,9);
\draw [short] (7.75,13.25) -- (7.75,13.75);
\draw (7.75,13.75) to[short, -o] (7.75,14) ;
\draw (10,9) to[short, -o] (10,9.25) ;
\draw [ color={rgb,255:red,129; green,61; blue,156}, line width=1pt, ->, >=Stealth, dashed] (13,10.75) -- (11.5,10.75);
\draw [ color={rgb,255:red,129; green,61; blue,156}, line width=1pt, ->, >=Stealth, dashed] (13,11.5) -- (10.25,12.5);
\node [font=\large, color={rgb,255:red,229; green,165; blue,10}] at (14,11.75) {};
\node [font=\Large, color={rgb,255:red,129; green,61; blue,156}] at (13.5,11.5) {(1)};
\node [font=\Large, color={rgb,255:red,129; green,61; blue,156}] at (13.5,10.75) {(2)};
\node [font=\Large, color={rgb,255:red,229; green,165; blue,10}] at (7.25,14) {(3)};
\node [font=\Large, color={rgb,255:red,229; green,165; blue,10}] at (10,9.75) {(4)};
\node [font=\Large, color={rgb,255:red,224; green,27; blue,36}] at (14.75,11.5) {(a)};
\node [font=\Large, color={rgb,255:red,224; green,27; blue,36}] at (14.75,7.5) {(b)};
\node [font=\Large, color={rgb,255:red,224; green,27; blue,36}] at (-7.25,4.5) {(c)};
\node [font=\Large, color={rgb,255:red,224; green,27; blue,36}] at (-9.75,10.25) {(d)};
\draw (-8.5,10.25) to[short, -o] (-9.25,10.25) ;
\draw [ color={rgb,255:red,129; green,61; blue,156}, ->, >=Stealth, dashed] (-3.25,2.75) -- (-1.5,2.75);
\draw [ color={rgb,255:red,129; green,61; blue,156}, ->, >=Stealth, dashed] (-3.25,2) -- (-1.5,2);
\draw (-3.25,1.25) to[short, -o] (-1.75,1.25) ;
\draw (-3.25,0.5) to[short, -o] (-1.75,0.5) ;
\node [font=\Large, color={rgb,255:red,129; green,61; blue,156}] at (-1,2.75) {(1)};
\node [font=\Large, color={rgb,255:red,129; green,61; blue,156}] at (-1,2) {(2)};
\node [font=\Large, color={rgb,255:red,229; green,165; blue,10}] at (-1,1.25) {(3)};
\node [font=\Large, color={rgb,255:red,229; green,165; blue,10}] at (-1,0.5) {(4)};
\draw (-9,1) to (-9.25,1) node[ground]{};
\draw (-5.75,4.25) to (-6.5,4.25) node[ground]{};
\draw (-9,6.75) to (-9.5,6.75) node[ground]{};
\draw (10.25,5.25) to (9.5,5.25) node[ground]{};
\draw (10.25,14) to (9.5,14) node[ground]{};
\draw (-7.5,9.5) to (-7.5,9.25) node[ground]{};
\draw [ line width=0.5pt ] (-3.5,9) rectangle (-2.5,8);
\draw [line width=0.7pt, <->, >=Stealth] (-1.5,9.25) -- (3.5,9.25);
\draw [ color={rgb,255:red,51; green,209; blue,122}, line width=2pt, ->, >=Stealth] (5.5,6.75) .. controls (5.5,9.75) and (5.25,10.5) .. (-0.5,10.5) ;
\node [font=\Large, color={rgb,255:red,46; green,194; blue,126}] at (2,10.75) {Plasma};
\draw [ fill={rgb,255:red,246; green,245; blue,244} ] (9.5,3.25) rectangle  node {\Large Delay generator} (15.25,0);
\draw (9.5,2.75) to[short, -o] (8.25,2.75) ;
\draw (9.5,2) to[short, -o] (8.25,2) ;
\draw (9.5,1.25) to[short, -o] (8.25,1.25) ;
\draw (9.5,0.5) to[short, -o] (8.25,0.5) ;
\node [font=\Large, color={rgb,255:red,224; green,27; blue,36}] at (7.5,2.75) {(a)};
\node [font=\Large, color={rgb,255:red,224; green,27; blue,36}] at (7.5,2) {(b)};
\node [font=\Large, color={rgb,255:red,224; green,27; blue,36}] at (7.5,1.25) {(c)};
\node [font=\Large, color={rgb,255:red,224; green,27; blue,36}] at (7.5,0.5) {(d)};
\draw [ color={rgb,255:red,224; green,27; blue,36}, line width=0.7pt, <->, >=Stealth] (-2.25,10.5) -- (-2.25,8);
\draw [ color={rgb,255:red,224; green,27; blue,36}, line width=0.7pt, dashed] (-2.25,7.75) -- (-4.5,7.75);
\draw [ color={rgb,255:red,224; green,27; blue,36}, line width=0.7pt, dashed] (-4.5,10.75) -- (-2.25,10.75);
\draw  (-6.5,10.5) rectangle (-2.5,10.25);
\draw [ fill={rgb,255:red,246; green,245; blue,244} ] (-8.5,11.25) rectangle  node {\Large ERMS} (-6.25,9.5);
\draw [ rotate around={-75:(5.5,8.75)}] (5.5,8.75) ellipse (0.25cm and 1.5cm);
\draw [short] (11,12.75) -- (11,13);
\draw [short] (11.25,13) -- (11.25,10.75);
\draw [short] (11.25,10.75) -- (8.5,10.75);
\draw [short] (8.5,10.75) -- (8.5,4.25);
\draw [short] (8.5,4.25) -- (5.5,4.25);
\draw [short] (-3.25,6.75) -- (-3.75,6.75);
\draw [short] (-2.75,8) -- (-2.75,4.75);
\draw [short] (-3.25,4.75) -- (-2.75,4.75);
\draw [line width=0.7pt, short] (-7.75,6) -- (-7.75,3.25);
\node [font=\Large] at (1,8.75) {Distance variation};
\draw [short] (-3.25,6.75) -- (-3.25,8);
\end{circuitikz}
}%

\caption[[Vacuum chamber setup with diagnostics]]{Schematic of the vacuum chamber setup for cathodic arc plasma diagnostics and thin film deposition. (a) Arc power supply, (b) EM-coil power supply, (c) Langmuir probe and QCM, (d) energy-resolving mass spectrometer (ERMS). The delay generator (a–d) synchronizes the arc power supply, EM-coil activation, and diagnostic tools.}


\label{fig:vacuumchamber}
\end{figure}


\section{Plasma Diagnostics}

\subsection{Ion Current Probe (Langmuir Probe)}\label{sec:langmuir_probe}

An in-house-built ion collector probe was designed to measure the ion current 
density in ion saturation mode. The probe consisted of a 5~mm diameter copper 
rod machined to form a cylindrical collection surface with a nail-head geometry. 
This design prevented short circuits between the probe and the grounded mount 
due to metallic deposition. The probe body was wrapped in Kapton tape for 
electrical insulation from the aluminum mounting assembly 
(Figure~\ref{fig:holder_assembly}).


\begin{figure}[h!]
	\centering
	\includegraphics[width=0.7\textwidth]{"Figures/experimental methods/langmuirprobe"}
	\caption[Ion collector probe construction]{In-house built ion collector probe wrapped in Kapton tape for electrical insulation from the aluminum assembly holder. The probe includes an attachment point for a screw terminal connector, enabling connection to the ion probe circuit.}
	\label{fig:langmuirprobe}
\end{figure}



To ensure full ion collection and suppress electron current, the probe was 
negatively biased at $V_b = -80$~V to operate in the ion saturation 
regime~\cite{chen1984introduction}, as determined through bias voltage 
characterization measurements (Section~\ref{sec:bias_characterization}).



\begin{figure}[!ht]
	\centering
	\resizebox{0.8\textwidth}{!}{%
		\begin{circuitikz}
			\tikzstyle{every node}=[font=\LARGE]
			\draw [ line width=0.5pt](5.75,13.25) to[short] (7.25,13.25);
			\node at (7.25,13.25) [circ] {};
			\draw [ line width=0.5pt](7.25,13.25) to[short] (7.75,13.25);
			\node at (7.75,13.25) [circ] {};
			\draw [ line width=0.5pt](7.75,14.25) to[short] (7.75,13.25);
			\draw [ line width=0.5pt](7.75,13.25) to[short] (7.75,12.5);
			\draw [line width=0.5pt](7.75,14.25) to[C,l={ \normalsize 22nF}] (9.75,14.25);
			\draw [line width=0.5pt](7.75,12.5) to[C,l={ \normalsize 0.5$\mu$F}] (9.75,12.5);
			\draw [ line width=0.5pt](9.75,14.25) to[short] (9.75,12.5);
			\draw [ line width=0.5pt](9.75,13.25) to[short] (10.25,13.25);
			\draw [ line width=0.5pt](10.25,13.25) to[short] (12.25,13.25);
			\node at (10.25,13.25) [circ] {};
			\node at (9.75,13.25) [circ] {};
			\draw [ line width=0.5pt](7.25,13.25) to[short] (7.25,12.25);
			\draw [ line width=0.5pt](10.25,13.25) to[short] (10.25,12.25);
			\draw [ line width=0.5pt](7.25,12.25) to[european resistor,l={ \normalsize 1k$\Omega$}] (7.25,11);
			\draw [ line width=0.5pt](10.25,12.25) to[european resistor,l={ \normalsize 10k$\Omega$}] (10.25,11);
			\draw [ line width=0.5pt](7.25,11) to[D] (7.25,10.25);
			\draw [ line width=0.5pt](10.25,10.25) to[D] (10.25,11);
			\draw [ line width=0.5pt](7.75,9.75) to[american controlled voltage source,l={ \normalsize 80 V}] (9.75,9.75);
			\draw [ line width=0.5pt](10.25,10.25) to[short] (10.25,9.75);
			\draw [ line width=0.5pt](10.25,9.75) to[short] (9.75,9.75);
			\draw [ line width=0.5pt](7.25,10.25) to[short] (7.25,9.75);
			\draw [ line width=0.5pt](7.25,9.75) to[short] (7.75,9.75);
			\node at (12.25,13.25) [circ] {};
			\draw [ line width=0.5pt](12.25,13.25) to[european resistor,l={ \normalsize 400$\Omega$}] (12.25,10.25);
			\draw [ line width=0.5pt ] (14.5,12.5) rectangle  node {\large Oscilloscope} (17.5,11.25);
			\draw [ line width=0.5pt](12.25,13.25) to[short] (14.75,13.25);
			\draw [ line width=0.5pt](14.75,13.25) to[short] (14.75,12.5);
			\draw [ line width=0.5pt](14.75,11.25) to[short] (14.75,10.25);
			\draw [ line width=0.5pt](14.75,10.25) to[short] (13.25,10.25);
			\draw [line width=0.5pt](13,10.25) to (13,9.75) node[ground]{};
			\draw [ line width=0.5pt](12.25,10.25) to[short] (13.25,10.25);
			\draw [ line width=0.6pt ] (5.75,14.25) circle (0.75cm) node {\normalsize Probe} ;
			\draw [line width=0.5pt, short] (5.75,13.25) -- (5.75,13.5);
			
			
		\end{circuitikz}
	}%
	\caption[Ion probe circuit diagram]{Schematic of the ion-flux probe circuit. The $\SI{400}{\ohm}$ resistor converts ion current to voltage, while the $\SI{0.5}{\micro F}$ capacitor and $\SI{400}{\ohm}$ resistor form a high-pass filter with a $\SI{795}{Hz}$ cutoff.}
	\label{fig:probe_circuit}
\end{figure}

The probe circuit (Figure~\ref{fig:probe_circuit}) converts the collected ion 
current to a voltage signal. The ion current passes through a 0.5~$\mu$F 
coupling capacitor and develops a voltage across the 400~$\Omega$ measurement 
resistor. This RC configuration forms a high-pass filter with cutoff frequency:

\begin{equation}
	f_c = \frac{1}{2\pi RC} = \frac{1}{2\pi \cdot 400\,\Omega \cdot 0.5 \times 10^{-6}\,\text{F}} \approx 795\,\text{Hz}
\end{equation}

The high-pass filter blocks DC offsets while passing the pulsed ion current 
signal. The RC time constant $\tau = 200~\mu$s is much shorter than the 2~s 
interval between pulses, ensuring complete capacitor discharge between 
measurements. The bias supply line incorporates a 22~nF capacitor and 
1~k$\Omega$ resistor as a low-pass filter to smooth the $-80$~V bias voltage.\\


\begin{figure}[h]
	\centering
	\includegraphics[width=0.7\linewidth]{Figures/experimental methods/ion_probe_voltage}
	\caption[Voltage waveform across the shunt resistor]{Example voltage waveform measured across the 400~$\Omega$ resistor 
		during a single arc pulse. The signal shows transients at pulse ignition 
		(0~ms) and termination (1~ms), with gradual decay during the pulse due to 
		the high-pass filter characteristics.}
	\label{fig:ion_current_waveform}
\end{figure}

The ion current is calculated from the measured voltage as:
\begin{equation}
	I_{\text{ion}} = \frac{V_{\text{measured}}}{400\,\Omega}
\end{equation}


The voltage signal was recorded using a Tektronix MSO64 oscilloscope. 
Figure~\ref{fig:ion_current_waveform} shows an example waveform of 64 pulses averaged. The ion current density $J_i$ is calculated as:
\begin{equation}
	J_i = \frac{V_{\text{measured}}}{A \cdot R}
\end{equation}
where $A = 0.1963$~cm$^2$ is the probe collection area and $R = 400~\Omega$. 


\subsection{Quartz Crystal Microbalance}\label{section:QCM}
A quartz crystal microbalance (QCM) was used to measure the deposited mass during pulsed cathodic arc operation. The system consisted of an INFICON Cool Drawer\texttrademark\  with a single sensor in standard orientation, water-cooled to ensure thermal stability. A 14~mm diameter, 6~MHz AT-cut quartz crystal was operated with an SQM-160 controller for electronic readout.\\



\begin{figure}[h]
    \centering
    \includegraphics[width=0.7\linewidth]{Figures/experimental methods/QCM_head.png}
    \caption[QCM sensor assembly]{Sensor head of the INFICON Cool Drawer\texttrademark{} Quartz Crystal Microbalance (QCM) used for in-situ mass operation monitoring during cathodic arc sputtering. The assembly includes a water-cooled housing, a 14~mm diameter AT-cut quartz crystal (6~MHz), and electrode leads for connection to the SQM-160 controller. (Schematic adapted from INFICON STP file, available at \url{https://www.inficon.com/en/products/thin-film-technology/cool-drawer-single-sensor}).}
    \label{fig:QCM_sensor_head}
\end{figure}

The measurement principle follows the Sauerbrey equation \cite{Sauerbrey_1959}, which relates the change in resonance frequency of the quartz crystal to the deposited mass:  


\begin{equation}\label{eq:sauerbrey}
	\Delta m = \frac{N_{\text{AT}} \rho_q \pi r^2}{F_q^2} \Delta F 
	= 18.8146023 \times 10^{-9}\,\frac{\text{g}}{\text{Hz}} \cdot \Delta F
\end{equation}

Here $\rho_q = 2.649$~g\,cm$^{-3}$ is the quartz density, $r = 0.375$~cm is the radius of the exposed crystal area, $N_{\text{AT}} = 166\,100$~Hz\,cm is the frequency constant of the AT-cut quartz, $F_q = 6$~MHz is the uncoated resonance frequency, and $\Delta F$ is the measured frequency shift.\\

The Sauerbrey relation is accurate as long as $\Delta F \lesssim 0.05 F_q$ (approximately 0.3~MHz for a 6~MHz crystal), for more details on the correction above this limit look in Appendix \ref{long_qcm}. 
In the present experiments, the observed frequency shifts ranged from approximately 1~Hz to 50~Hz over 64 pulses. These shifts are well above the SQM-160 resolution of approximately 0.03~Hz at 6~MHz, yet orders of magnitude below the Sauerbrey breakdown limit. The Sauerbrey approximation was therefore fully sufficient for all measurements in this work.\\


\subsection{Simultaneous Ion Probe and QCM Measurements}

To enable direct comparison between ion current and deposited mass, the Langmuir probe and QCM were mounted on the same movable aluminum assembly. The probe was positioned through a precision-milled pass-through hole, while the QCM was secured in a dedicated cutout and fixed via screws (see Appendix~\ref{sec:holder_assembly_figure} for assembly details). This configuration ensured rigid mechanical alignment between the two diagnostics throughout all measurements.


The probe collection area ($A_{\text{probe}} = 0.196$~cm$^2$, diameter 5~mm) was smaller than but contained within the QCM active area ($A_{\text{QCM}} = 0.442$~cm$^2$, diameter 7.5~mm). Both diagnostics were positioned as close as possible to each other to minimize spatial gradients between measurement locations. This design is meant to mitigate radial plasma density gradients inherent to expanding cathodic arc plasmas \cite[Chap.~6.2]{cathodic_arcs}, though even with close proximity, gradients cannot be completely eliminated and remained evident in the experimental results.




\subsection{Energy-Resolving Mass Spectrometer (ERMS)}\label{sec:QMS}


An energy-resolving mass spectrometer (ERMS, Hiden EQP 1000) was used to measure ion energy distribution functions (IEDFs) and charge-state-resolved fluxes of plasma species. The system combines an electrostatic energy analyzer with a quadrupole mass filter to measure distributions of energy-to-charge ratios (E/Q) at fixed mass-to-charge ratios (M/Q), and vice versa.\\

Ions enter the ERMS through a 50~$\mu$m sampling orifice and are transported to the energy analyzer, where their kinetic energy $E_i$ is selected according to:
\begin{equation}\label{eq:ion_energy}
	E_i = \left(V_{\text{ENERGY}} + \frac{R}{d} V_{\text{PLATES}} - V_{\text{AXIS}}\right) Q \cdot e
\end{equation}

Here, $V_{\text{ENERGY}}$ and $V_{\text{AXIS}}$ are opposing potentials applied to the analyzer, $R$ is the mean radius of the cylindrical sector, $d$ is the plate separation, $V_{\text{PLATES}}$ is the potential difference across the sector plates, $Q$ is the ion charge state, and $e$ is the elementary charge \cite{hiden_eqp_manual}. \\

The energy-selected ions are then injected into the quadrupole mass filter, where a combination of AC and DC electric fields creates a stability region dependent on $M/Q$, described by the Mathieu equations \cite{dawson_quadrupole_1997}. The potential in the quadrupole is:
\begin{equation}
	V(x,y,t) = \frac{U_0 \cos(\omega t)}{r_0^2} (x^2 - y^2)
\end{equation}

where $U_0$ is the amplitude of the AC voltage, $\omega$ is the angular frequency, and $r_0$ is the field radius. The stability of ion motion is determined by the dimensionless parameters:
\begin{equation}
	a = \frac{8 e U_{\text{DC}}}{M r_0^2 \omega^2}, \quad q = \frac{4 e U_0}{M r_0^2 \omega^2}
\end{equation}

with $U_{\text{DC}}$ as the superimposed DC voltage and $M$ the ion mass. For a given $M/Q$, stable transmission occurs only within specific $(a,q)$ regions, enabling mass separation \cite{dawson_quadrupole_1997, march_quadrupole_1989}. Only ions with trajectories stable in both radial and axial directions reach the detector.\\

To reduce interference from the arc's magnetic field, the ERMS was equipped with a grounded mu-metal shield \cite{RN5}. 

\begin{figure}[h]
	\centering
	\includegraphics[width=0.9\linewidth]{"Figures/experimental methods/hiden eqp"}
	\caption[ERMS schematic diagram]{ERMS, Hiden EQP HE 1000: (1) Sampling Orifice, (2) Electron Impact Ion Source, (3) Transfer Ion Optics, (4) Quadrupole Lens, (5) Energy Filter, (6) Decelerating Lens, (7) Quadrupole Mass Filter, (8) Detector, (9) Differential Pump Port \cite{hidenanalytical}}
	\label{fig:hiden-eqp}
\end{figure}

IEDFs were measured using a double-trigger acquisition scheme synchronized with the arc pulses (1~ms duration, 5~Hz repetition rate). For each $M/Q$ value, two 20~ms acquisition windows were recorded, activated 10~ms before the onset of the pulse. The combined 40~ms of data for each point were averaged to obtain the final IEDF. Measurements were performed for charge states $1^+$, $2^+$, and $3^+$ of aluminum ions, and for charge states $1^+$, $2^+$, $3^+$, and $4^+$ of titanium ions. For nitrogen ion species (N and N$_2$), only the $1^+$ ionization level was measured. This was achieved by scanning $V_{\text{ENERGY}}$ while fixing the quadrupole mass filter to the corresponding $M/Q$ values (Table~\ref{tab:mq_ratios}).


\begin{table}[h]
	\centering
	\caption[Mass-to-charge ratios for ion species]{Mass-to-charge ratios measured for each ion species and charge state.}
	\label{tab:mq_ratios}
	\begin{tabular}{c|cccc|}
		\cline{2-5}
		& \multicolumn{4}{c|}{Molar mass over charge ratio of:}                                     \\ \hline
		\multicolumn{1}{|c|}{Ionization} & \multicolumn{1}{c|}{Al}   & \multicolumn{1}{c|}{Ti}     & \multicolumn{1}{c|}{N}  & N$_2$ \\ \hline
		\multicolumn{1}{|c|}{1+}         & \multicolumn{1}{c|}{27}   & \multicolumn{1}{c|}{47.867} & \multicolumn{1}{c|}{14} & 28    \\ \hline
		\multicolumn{1}{|c|}{2+}         & \multicolumn{1}{c|}{13.5} & \multicolumn{1}{c|}{23.933} & \multicolumn{1}{c|}{-}  & -     \\ \hline
		\multicolumn{1}{|c|}{3+}         & \multicolumn{1}{c|}{9}    & \multicolumn{1}{c|}{15.955} & \multicolumn{1}{c|}{-}  & -     \\ \hline
		\multicolumn{1}{|c|}{4+}         & \multicolumn{1}{c|}{-}    & \multicolumn{1}{c|}{11.966} & \multicolumn{1}{c|}{-}  & -     \\ \hline
	\end{tabular}
\end{table}

To derive ion energy distributions for different charge states, the measured $E/Q$ distributions were multiplied by the corresponding charge state number $Q$. This approach accounts for the charge-dependent scaling of ion energies and enables calculation of mean charge states $\langle Q \rangle$ and the energies for each species, as detailed in Section~\ref{sec:error_propagation}.

%The \(m/q\) range was calibrated using N\(_2^+\) as reference ions, with typical operating parameters set to \(U_0 = \SI{100}{V}\), \(U_{\text{DC}} = \SI{5}{V}\), and \(\omega/2\pi = \SI{1.2}{MHz}\). Under these settings, the system provided a mass resolution of \(\Delta m/m \approx 0.5\) (FWHM) for ions in the \(\SI{0}{--}\SI{150}{eV}\) energy range \cite{hiden_eqp_manual}. The energy resolution was adjusted to \(\Delta E/E \approx 0.02\) by tuning \(V_{\text{PLATES}}\) and optimizing the alignment of the orifice and detector.
%%%


%The ratio $\Phi_{\text{total}} / \Gamma_{\text{ion}}$ provides insight into the fraction of deposited material arriving as ions versus neutrals, which is particularly important for understanding reactive mode deposition where activated nitrogen species contribute significantly to film growth. In metallic mode (vacuum deposition), this ratio approaches unity if all deposited material arrives as ions. In reactive mode, values greater than unity indicate significant neutral nitrogen incorporation.


%The integrated ion flux \(\Gamma_i\) was determined by normalizing the IEDFs to the plasma density, as measured by a Langmuir probe (\ref{sec:langmuir_probe}). The observed charge-state distributions of Ti\(^+\) and Al\(^+\) ions aligned with earlier findings for cathodic arc plasmas \cite{RN6,unutulmazsoy}.


\section{Thin Film Deposition and Characterization}\label{sec:film_characterization}

Thin films were deposited on silicon substrates mounted on a movable substrate holder positioned at controlled distances from the macroparticle filter exit. Prior to deposition, the 2x2 substrates were cut into four pieces and cleaned with compressed nitrogen gas. For thickness measurements, a masking technique was employed: a marker line was drawn near the edge of each substrate before deposition, creating a well-defined step edge. After deposition, this marker was removed by ultrasonic cleaning in isopropanol, leaving a sharp boundary between the coated and uncoated regions for profilometry analysis.\\

The deposition parameters (distance, magnetic field strength, nitrogen pressure) were selected to match a subset of the plasma diagnostic measurements, enabling direct correlation between plasma properties and film characteristics. All depositions were performed with 6000--8000 pulses at 5~Hz repetition rate (1~ms pulse duration) to ensure sufficient film thickness for ex situ characterization. The substrate holder was not heated, allowing the investigation of room-temperature film growth under energetic ion bombardment conditions characteristic of cathodic arc deposition.



\subsection{Profilometry}\label{sec:profil}

Stylus profilometry was used to measure film thickness by mechanically tracing the surface topography using a diamond-tipped stylus. The technique provides direct measurement of step heights between masked and deposited regions, making it particularly useful for verifying film thickness values obtained by QCM \cite{poon1995}.\\

In profilometry, a stylus with a small tip radius is dragged across the sample surface with a controlled force of 3~mg while its vertical displacement is monitored electromagnetically. The resulting trace provides a profile of the surface from which the step height (film thickness) with vertical resolution down to $\sim$ 1~nm can be extracted.\\

%The Dektak stylus profilometer used in this study has a measurement range from 6.5~nm to 800~$\mu$m with repeatability of 0.4~nm. For thickness measurements, silicon substrates were partially masked with a marker line, which created a well-defined step edge during deposition. This marker line was subsequently removed with Isopropanol in a ultrasonic bath. Multiple scans across each step were performed to ensure reproducibility, and the average thickness was calculated from at least three different positions on each sample.\\


%One limitation of contact profilometry is the potential for stylus-induced damage on very soft films, though this was not a concern for the hard TiAlN coatings investigated here. The technique is complementary to XRR, with profilometry providing rapid, direct thickness measurements while XRR offers higher accuracy for very thin films and additional information on density and roughness \cite{windover1999}.


\subsection{X-ray Diffraction (XRD)}\label{sec:XRD}

X-ray diffraction (XRD) was employed to analyze the crystallographic structure of the thin films deposited during the experiments. XRD is a non-destructive technique that provides detailed information about the crystalline phases present in the material, as well as their lattice parameters, crystallite size, and strain.\\

Due to the small film thickness, out-of-plane diffraction techniques were used to enhance the film signal relative to the substrate. Two measurement geometries are employed \cite{mitsunaga2009}:

\begin{itemize}
	\item \textbf{Symmetrical reflection ($2\theta/\theta$ scan):} Both incident and diffracted beams make equal angles with the sample surface. This geometry probes lattice planes parallel to the substrate and is suitable for textured films, but substrate peaks can obscure weak film signals.
	
	\item \textbf{Asymmetrical reflection (thin-film method):} The incident beam is fixed at a small grazing angle $\alpha$, while the detector scans in $2\theta$. This reduces the X-ray penetration depth from tens of micrometers to a few micrometers, greatly enhancing sensitivity to thin films \cite{mitsunaga2009}.
\end{itemize}

\begin{figure}
	\centering
	\includegraphics[width=0.75\linewidth]{"Figures/experimental methods/xrd_in+outofplae"}
	\caption[Schematic of diffraction geometries for XRD]{Schematic of out-of-plane diffraction geometries: symmetrical reflection (left) and asymmetrical reflection (right) for thin film analysis. Taken from \cite{mitsunaga2009}.}
	\label{fig:xrdinoutofplane}
\end{figure}

The XRD measurements were performed using a Rigaku Ultima IV system equipped with a non monochromatic $Cu_{\alpha}$ source. The crystallographic structure was determined by analyzing the diffraction patterns. The Bragg equation was used to identify the crystalline phases:
\begin{equation}
	2d \sin(\theta) = n\lambda,
\end{equation}
where $d$ is the spacing between atomic planes, $\theta$ is the diffraction angle, $n$ is an integer, and $\lambda$ is the X-ray wavelength.\\



\subsection{X-ray Reflectometry (XRR)}\label{sec:XRR}

X-ray reflectometry (XRR) is a non-destructive technique used to determine film thickness, density, and surface or interface roughness by measuring the intensity of X-rays reflected at grazing incidence angles (0.01--5°) \cite{yasaka2010}.

When X-rays strike a flat surface at shallow angles below the critical angle $\theta_c$, total external reflection occurs. The critical angle is related to the electron density of the material through:
\begin{equation}
	\theta_c = \sqrt{2\delta},
\end{equation}
where $\delta$ depends on the material density and composition \cite{yasaka2010}. Above $\theta_c$, X-rays penetrate the film and reflect from interfaces, creating interference patterns known as Kiessig fringes. The period of these oscillations is directly related to film thickness \cite{kiessig1931}:
\begin{equation}\label{eq:xrr_thickness}
	d \approx \frac{\lambda}{2\Delta\theta},
\end{equation}
where $\lambda$ is the X-ray wavelength and $\Delta\theta$ is the angular spacing between fringes.

\begin{figure}[h]
	\centering
	\includegraphics[width=\linewidth]{Figures/experimental methods/XRR SAMPLE.png}
	\caption[Example XRR measurement and fit]{Example XRR measurement and fit for a TiAlN thin film deposited on Si substrate. The oscillations (Kiessig fringes) arise from interference between reflections from the film surface and the film-substrate interface.}
	\label{fig:xrr_fit}
\end{figure}

The XRR technique provides three key parameters \cite{yasaka2010}:
\begin{itemize}
	\item \textbf{Film thickness:} Determined from the oscillation period, with accuracy of 0.1--0.2~nm for films in the 1--150~nm range.
	
	\item \textbf{Density:} Extracted from the critical angle position and oscillation amplitude. Larger density contrast between film and substrate produces higher amplitude oscillations.
	
	\item \textbf{Surface and interface roughness:} Surface roughness causes faster decay of reflectivity at higher angles, while interface roughness reduces oscillation amplitude.
\end{itemize}

%The XRR measurements were performed using a \textcolor{red}{model name} diffractometer equipped with a Cu $K_\alpha$ X-ray source ($\lambda = 1.5406$~Å). Sample alignment was performed using the total reflection phenomenon to ensure proper angular positioning. Scans were performed in $2\theta/\theta$ geometry from \textcolor{red}{add range, e.g., 0.2° to 6°} with a step size of \textcolor{red}{add step size, e.g., 0.01°} and a scan speed of \textcolor{red}{add speed, e.g., 0.5°/min}.

The measured reflectivity curves were analyzed using the Parratt recursive formalism \cite{parratt1954}, which accounts for multiple reflections at each interface. Film thickness, density, and roughness parameters were refined by fitting the calculated reflectivity curve to the experimental data. Initial thickness estimates from Profilometry measurements (Section~\ref{sec:profilometry}) were used as starting values for the fitting procedure. XRR is particularly suited for verifying thicknesses and characterizing film density and surface quality for the deposited TiAl/TiAlN films.



\subsection{Scanning Electron Microscopy (SEM)}\label{sec:SEM}

Scanning electron microscopy (SEM) was employed to characterize the surface morphology and microstructure of the deposited films. SEM uses a focused electron beam to scan the sample surface, generating secondary electrons (SE) and backscattered electrons (BSE) that provide information about surface topography and compositional contrast, respectively \cite{goldstein2017,reimer1998}.\\

SEM imaging was performed using a \textcolor{red}{[MODEL - Zeiss Gemini/Sigma]} field emission scanning electron microscope equipped with an in-lens detector for high-resolution secondary electron imaging. Samples were imaged at an accelerating voltage of 15~kV with magnifications ranging from 30,000$\times$ to 100,000$\times$. The conductive TiAlN films required no additional coating preparation. \\

The primary objectives of SEM characterization were to:
\begin{itemize}[noitemsep]
	\item Examine surface morphology and the presence of any macroparticles typical of cathodic arc deposition
	\item Assess film uniformity across the substrate
	\item Identify microstructural features resulting from different deposition conditions\\
\end{itemize}

Representative SEM micrographs and analysis of surface features are presented in Section~\ref{sec:SEM_results}.

\begin{figure}[h]
	\centering
	\includegraphics[width=0.6\linewidth]{Figures/experimental methods/SEM_chamber.jpg}
	\caption[Interior view of the SEM sample chamber]{Interior view of the SEM sample chamber showing the sample holder stage and detector configuration used for imaging and EDX analysis.}
	\label{fig:SEM_chamber}
\end{figure}


\subsection{Energy-Dispersive X-ray Spectroscopy (EDX)}\label{sec:EDX}

Energy-dispersive X-ray spectroscopy (EDX) relies on the ionization of inner-shell electrons by the incident electron beam. When an inner-shell electron is ejected, an electron from a higher energy level fills the vacancy, releasing energy in the form of a characteristic X-ray. The energy of this X-ray is unique to each element, allowing identification and quantification of the sample composition. The intensity of the characteristic X-ray peaks is proportional to the concentration of each element, enabling quantitative analysis through comparison with standards or standardless quantification algorithms \cite{goldstein2017}.\\

EDX measurements were performed using a silicon drift detector (SDD) integrated with the SEM system, operating at an accelerating voltage of 10~kV. To minimize substrate contribution from the thin films, samples were tilted to 55° relative to the electron beam increasing the effective path length by $\approx$ 1.74.

Prior to analysis, samples were cleaned ultrasonically in isopropanol to remove loose particles. However, organic solvent residues and hydrocarbon contamination from the vacuum system can lead to carbon deposition under electron beam irradiation \cite{goldstein2017,reimer1998}. A small carbon peak was therefore typically observed in EDX spectra and was excluded from compositional quantification of the TiAlN films. Multiple area measurements were taken on each sample to verify that all surface features, including any macroparticles or morphological structures, had the same metal-nitride composition as the underlying thin film and to identify potential impurities beyond the surface carbon layer.

The primary elements analyzed were titanium (Ti), aluminum (Al), and nitrogen (N). Oxygen (O) and silicon (Si) were also monitored to assess surface oxidation and to detect any contribution from the Si substrate. The measured film compositions are used to calculate the effective molar mass $M_{\text{eff}}$ required for determining total deposited flux from QCM measurements (Section~\ref{sec:QMS}). Compositional results and representative EDX spectra are presented in Section~\ref{sec:EDX_results}.

\section{Calculation of Ion Flux and Total Deposited Flux}
The plasma diagnostic measurements described in Section~\ref{sec:langmuir_probe}--\ref{sec:QMS} provide electrical currents, deposited masses, and ion energy distributions. To quantify the actual particle arrival rates and relate these to film growth, these raw measurements must be converted to particle fluxes. This section describes the calculation procedures that combine multiple diagnostic techniques to derive ion flux and total deposited flux.


\subsection{Ion Flux from Probe and ERMS Measurements}

The ion flux $\Gamma_{\text{ion}}$ (ions\,cm$^{-2}$\,s$^{-1}$) represents the number of charged metal ions arriving per unit area per unit time. It is calculated from the ion current measured by the Langmuir probe (Section~\ref{sec:langmuir_probe}) combined with the mean charge state determined from ERMS measurements (Section~\ref{sec:QMS}):
\begin{equation}
	\Gamma_{\text{ion}} = \frac{I_{\text{ion}}}{e \langle Q \rangle A_{\text{probe}}}
	\label{eq:ion_flux}
\end{equation}


where $\langle Q \rangle$ is the average charge state determined from ERMS measurements (Section~\ref{sec:QMS}), $e = 1.602 \times 10^{-19}$~C is the elementary charge, $A_{\text{probe}} = 0.196$~cm$^2$ is the probe collection area, and $R = 400$~$\Omega$ is the measurement resistor. This combination of probe and mass spectrometer data accounts for the multiply charged nature of the cathodic arc plasma and enables accurate determination of the particle flux from the electrical current measurement.

\subsection{Total Deposited Flux from QCM and Film Characterization}
The total deposited flux $\Phi_{\text{total}}$ (atoms\,cm$^{-2}$\,s$^{-1}$) quantifies the arrival rate of all species contributing to film growth, including both ions and neutrals. It is calculated from QCM mass measurements (Section~\ref{section:QCM}) combined with film composition from EDX (Section~\ref{sec:EDX}):
\begin{equation}
	\Phi_{\text{total}} = \frac{\Delta m \cdot N_A}{A_{\text{QCM}} \cdot \Delta t \cdot M_{\text{eff}}}
	\label{eq:total_flux}
\end{equation}

where $\Delta m$ is the mass change measured by QCM over time interval $\Delta t$, $A_{\text{QCM}} = 0.5027$~cm$^2$ is the active crystal area (8~mm diameter), $N_A = 6.022 \times 10^{23}$~mol$^{-1}$ is Avogadro's constant, and $M_{\text{eff}}$ is the effective molar mass calculated from film composition:

\begin{equation}
	M_{\text{eff}} = \sum_i x_i M_i = x_{\text{Ti}} \cdot 47.867 + x_{\text{Al}} \cdot 26.982 + x_{\text{N}} \cdot 14.007
	\label{eq:meff}
\end{equation}

where $x_i$ are the atomic fractions from EDX measurements (Table~\ref{tab:edx}) and $M_i$ are the atomic masses.

The effective molar mass varies significantly between metallic and reactive mode: $M_{\text{eff}} \approx 43.5$~g$\cdot$mol$^{-1}$ for Ti--Al films versus $M_{\text{eff}} \approx 31.8$--33.2~g$\cdot$mol$^{-1}$ for TiAlN films. This ~25\% reduction reflects the incorporation of light nitrogen atoms (14.0~g$\cdot$mol$^{-1}$) displacing heavier metal atoms.




%below is old data processing

\iffalse

\section{Data Processing}

Experimental data were processed using custom Python scripts to ensure consistency and reproducibility. A central Excel logbook served as the reference for all measurements, each identified by a unique suffix and linked to its corresponding data files. The logbook recorded experimental parameters such as date, distance, pressures, MFC flow rate, cryopump position, power supply settings, and pulse characteristics, as well as initial and final QCM frequencies for deposited mass determination. Associated oscilloscope waveforms were stored as CSV files for ion current analysis.\\


A Python script automated data handling by matching logbook entries to raw data, averaging ion current waveforms over multiple pulses to reduce noise when possible, and compiling all parameters into a unified dataset. This ensured uniform processing and efficient preparation for subsequent analysis.\\


QMS data were evaluated with a separate Python script that integrated raw spectra, applied mass transmission corrections to account for detection biases, and extracted parameters such as mean ion energy, charge state distribution, and potential and kinetic energy components for each species (different ionization levels of the four atoms measured). The processed results were visualized and exported into a structured CSV file for detailed examination of the ion energy distributions.



\section{Error Handling}
\subsection{Mass Spectrometry Measurements}

In mass spectrometry measurements, particularly in the context of cathodic arc processes, the standard deviation is crucial for characterizing data variability. The non-stationary nature of cathode spots in cathodic arcs leads to significant fluctuations in ion flux and charge composition from pulse to pulse \cite{cathodic_arcs}. Calibrating mass spectrometry measurements can be challenging due to these fluctuations, as random errors at each standard point used to determine the calibration curve can lead to a distribution of values for the same observed response for the unknown \cite{massspec_err}. As a result, achieving optimal calibration in such conditions may not be straightforward, and the accuracy of the calibration could be affected.

In this study, while efforts were made to calibrate the mass spectrometer accurately, the inherent difficulties associated with the calibration process in cathodic arc environments suggest that the calibration might not have been optimal. To account for this potential source of error, the impact of truncating the measurement artifact (e.g., signal noise or background interference) at different points was evaluated, and its effect on the data was analyzed in detail. This analysis will be further discussed in Section \ref{massspec_results}.

As detailed in Appendix \ref{appendix:code}, the standard deviation of energy measurements is calculated using the \verb|calculate_energy_stats| function, providing both average energy and standard deviation. This statistical approach offers a visual representation of measurement variability through error bars. The use of standard deviation in mass spectrometry is well-documented in scientific literature. For instance, standard deviation is commonly used to represent uncertainties in measurements and to describe the scatter among measured data points \cite{basic_massspec}. This statistical approach is crucial for characterizing data variability and ensuring the accuracy of the measurements.

\subsection{QCM and Ion Current Probe Measurements}\label{sec:ion/qcm_error}

For the ion current probe measurements, the standard deviation of the averaged measurements is calculated to characterize the variability in the data. This variability within a pulse can be attributed to several factors: the fluctuations in the plasma potential at the beginning of the pulse, charge exchange reactions between ions and neutrals and variations in the arc current and pulse parameters. These factors can cause fluctuations in the ion current during the pulse, leading to the observed variability \cite{evolution_pulse}. This kind of instability within the pulse will be henceforth shown with the help of error bars.

However, certain sources of error specific to the ion probe must be considered. The design of the ion probe can influence the measurements due to sheath effects, where the sheath around the probe can expand based on the probe's size and the biasing applied \cite[Appendix A.2]{cathodic_arcs}. Additionally, secondary electron emissions from the probe surface can affect the ion current measurements \cite[Chap. 8.2]{cathodic_arcs}. These emissions can be caused by interactions between the ions and the probe surface leading to inaccuracies in the measurements. As stated before, each pulse varies in terms of energies and amount of particles; these variations are investigated briefly in Sec.~\ref{sec:errorbar}, but cannot be considered for every data point.

The QCM measurements are subject to inaccuracies in terms of the error in the resonance frequency measurements, which is based on the manufacturer's specifications. The given measurement inaccuracy is $\Delta f = 0.03$~Hz, which will be taken for error propagation \cite{SQM160_manual}. Additionally, the relative position of the QCM sensor with respect to the ion flux affects the accuracy of the measurements especially at close distances.

\subsection{Error Propagation Analysis}\label{sec:error_propagation}

To properly characterize the uncertainties in the derived quantities, a comprehensive error propagation analysis was performed. The primary measured quantities with associated uncertainties are: the ion current $I_{\text{ion}}$ with standard deviation $\sigma_{I}$, the deposited mass $m$ measured by QCM with uncertainty $\sigma_m$, and the mean charge state $\bar{Q}$ with standard deviation $\sigma_Q$ determined from mass spectrometry measurements. These independent measurements are combined to calculate particle fluxes, necessitating careful propagation of uncertainties through the calculation chain.

\subsubsection{Ion Flux from Current Measurements}
The ion flux in mass units $\Phi_{\text{ion}}$ (in $\mu$g/cm$^2$/s) is calculated from the measured ion current according to:
\begin{equation}
	\Phi_{\text{ion}} = \frac{I_{\text{ion}}}{A \cdot \bar{Q} \cdot e}
	\label{eq:ion_flux_mass}
\end{equation}
where $e$ is the elementary charge, $A$ is the collection area of the ion probe and $\bar{Q}$ is the mean charge state.\\


For error propagation, considering the independent uncertainties in $I_{\text{ion}}$, $\bar{Q}$, the standard Gaussian error propagation formula gives \cite{taylor1997}:
\begin{equation}
	\left(\frac{\sigma_{\Phi_{\text{ion}}}}{\Phi_{\text{ion}}}\right)^2 = \left(\frac{\sigma_I}{I_{\text{ion}}}\right)^2 + \left(\frac{\sigma_Q}{\bar{Q}}\right)^2
\end{equation}

This shows that the relative uncertainties add in quadrature, which is typical for multiplicative error propagation.

\subsubsection{Atomic Flux from QCM Measurements}
The atomic flux $\Phi_{\text{atom}}$ (in atoms/cm$^2$/s) is determined from the mass flux measured by the QCM:
\begin{equation}
	\Phi_{\text{atom}} = \frac{\Phi_{\text{mass}} \cdot N_A}{M_{\text{eff}}}
	\label{eq:atomic_flux}
\end{equation}
where $\Phi_{\text{mass}} = m/(A_{\text{QCM}} \cdot t)$ is the mass flux and $N_A = 6.022 \times 10^{23}$~mol$^{-1}$ is Avogadro's constant. Following the same approach: \textcolor{red}{this depends what kind of data i get from EDX, else just standard error}
\begin{equation}
	\left(\frac{\sigma_{\Phi_{\text{atom}}}}{\Phi_{\text{atom}}}\right)^2 = \left(\frac{\sigma_m}{m}\right)^2 + \left(\frac{\sigma_{M_{\text{eff}}}}{M_{\text{eff}}}\right)^2
	\label{eq:rel_error_atomic_flux}
\end{equation}

The QCM mass uncertainty $\sigma_m$ is calculated from the frequency measurement uncertainty $\Delta f = 0.03$~Hz using the Sauerbrey relation.


\fi

