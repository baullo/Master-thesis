\newpage
\section*{\centering{Abstract}}

Pulsed filtered cathodic arc deposition generates highly ionized metal plasmas capable of producing dense, crystalline thin films without substrate heating. However, the process involves multiple coupled parameters: external magnetic fields simultaneously increase both ion charge states (potential energy) and ion flux, while reactive gases alter plasma composition through charge-exchange collisions. This thesis investigates the individual roles of ion flux and ion energy in Ti--Al--N thin film growth by systematically varying magnetic field strength (0--0.25~T), nitrogen pressure (0--0.3~Pa), and spatial position (10--20~cm from the plasma source).\\


Two diagnostics were operated simultaneously: a Langmuir probe measured ion current density, a quartz crystal microbalance tracked deposited mass. Additionally an energy-resolving mass spectrometer determined charge-state-resolved ion energy distributions. Measurements were performed in both metallic mode (vacuum) and reactive mode (nitrogen background) to characterize the transition between regimes.\\

The results show that increasing the magnetic field strength can amplify ion flux by up to a factor of eight. In reactive mode, the presence of nitrogen further boosts ion flux, as charge-exchange collisions reduce the number of highly ionized species and increase the proportion of lower-ionization ions. Energy-resolved mass spectrometry shows the charge-state distribution shifts toward lower ionization levels when operating in reactive mode. It also confirms that the total ion energy (kinetic and potential energy) remains within the 30–70 eV range, which is ideal for room-temperature crystallization. \\


This work extends the energy-flux decoupling framework established for (V,Al)N coatings to the industrially relevant TiAlN system and demonstrates that ion flux control via magnetic field strength provides a viable route to tailoring film microstructure independently of substrate temperature.


\chapter{Introduction and Literature Review}\label{chap:intro}

\section{Motivation}

Titanium aluminum nitride (Ti$_{1-x}$Al$_x$N) coatings are widely used in cutting tools and wear-resistant applications because of their high hardness (25--35~GPa), thermal stability, and oxidation resistance \cite{paldey2003}. These properties depend on maintaining the metastable cubic B1 crystal structure, which provides superior mechanical performance compared to the thermodynamically stable wurtzite phase \cite{mayrhofer2006}. Traditional deposition methods require substantial substrate heating to achieve dense, crystalline coatings. Cathodic arc deposition offers a key advantage: highly ionized metal plasmas with intrinsic high ion energies enable room-temperature crystallization through energetic condensation \cite{cathodic_arcs}.\\

The challenge is that, cathodic arc processes involve multiple coupled parameters. Ion energy has two components: kinetic energy from plasma expansion and potential energy released upon neutralization. The ion flux determines how rapidly this energy is delivered to the growing film. External magnetic fields can enhance both ion charge states and flux by up to an order of magnitude \cite{RN5}, while adding reactive gases such as nitrogen introduces charge-exchange collisions that alter plasma composition \cite{cathodic_arcs}. Understanding how these parameters individually influence film properties remains a fundamental challenge for predictive process control.

\section{State of the Art}

Recent systematic studies have made progress in understanding ion energy effects in cathodic arc deposition. Unutulmazsoy et al.\ showed that applying an external magnetic field to the cathode increases ion charge states (and thus potential energy), while applying a DC bias to the arc source adjusts kinetic energy \cite{unutulmazsoy}. Their work on (V,Al)N films demonstrated that crystalline coatings can be achieved at room temperature through energetic ion bombardment. However, a critical observation was that ``application of an external magnetic field also leads to an enhancement of the ion flux and hence the desired complete decoupling of the potential and kinetic energy effects will require further steps'' \cite{unutulmazsoy}. In their study, the magnetic field not only modified ion charge states but also increased ion flux by up to a factor of 10, making it impossible to isolate the effect of potential energy from that of ion flux intensity.\\


Kalanov et al.\ refined this approach using detailed energy-resolved mass spectrometry, taking a step toward decoupling these effects \cite{decoupling_kalanov_2025}. Their work on (V,Al)N films showed that the enhancement of room-temperature crystallinity correlates primarily with the increase in potential energy input. Notably, applying a DC bias to increase kinetic energy by approximately 30~eV did not produce the same crystallization effect, even though it increased the total ion energy. This demonstrated that potential energy plays a distinct role in promoting film crystallinity. However, even in this refined approach, the magnetic field enhancement of potential energy remained coupled to an increase in ion flux, leaving flux as an additional variable affecting film growth.\\

Both studies identified ion flux as another parameter requiring systematic investigation. The role of ion flux as an independent variable affecting energy delivery rate and adatom mobility has not been fully isolated. Additionally, these studies explored the effects of activated nitrogen in reactive deposition, which consists of both ionized and neutral (but excited) nitrogen species \cite{cathodic_arcs,bendikt2012}. The first study \cite{unutulmazsoy} focused entirely on reactive deposition with nitrogen, while the second \cite{decoupling_kalanov_2025} examined both metallic and reactive modes in equal measure. Both works demonstrated that activated nitrogen significantly affects film properties, but a key question remains unanswered: what fraction of the activated nitrogen consists of neutral species versus ionized species? This distinction is crucial for understanding the relative contributions of ions and neutrals to film growth, particularly in reactive mode where both species can participate in nitride formation.\\

For TiAlN specifically, while comparative studies between cathodic arc and magnetron sputtering have shown that ion energy flux significantly affects texture and mechanical properties \cite{karimi_aghda_2023}, the flux-energy decoupling framework established for (V,Al)N has not been applied to this industrially important system.

\section{Objectives}

This thesis addresses these gaps by systematically investigating the role of ion flux in Ti--Al--N thin film growth via pulsed filtered cathodic arc deposition. The specific objectives are:

\begin{enumerate}
	\item Characterize the ion flux and the total deposited flux as functions of external magnetic field strength and distance from the macroparticle filter. Furthermore, analyze the ion energy distributions of individual ion species.
	
	\item Investigate the transition from metallic to reactive mode by varying nitrogen pressure and quantify how this affects the ion-to-neutral arrival ratio and charge-state distributions.
	
	\item Establish quantitative correlations between ion flux and atomic flux across the experimental parameter space of magnetic field, nitrogen pressure, and distance from the macroparticle filter.
	
	\item Characterize the crystal structure, film thickness, film composition, and microstructure of deposited TiAlN films to correlate plasma parameters with film properties.
\end{enumerate}

The experimental approach combines multiple in situ plasma diagnostics with ex situ film characterization to separate ion flux effects from ion energy effects while extending the Unutulmazsoy/Kalanov framework to both reactive mode operation and the industrially relevant TiAlN system. This multi-diagnostic strategy, applied across a systematic parameter space, enables investigation of two key novel aspects: (1) the application to the TiAlN material system, and (2) the combined measurement of ion flux versus total deposited flux to distinguish the contribution of non-ionized activated nitrogen species and correlate this with film properties.

