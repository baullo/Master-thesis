\newpage
\section*{\centering{Abstract}}


Titanium aluminum nitride (Ti--Al--N) coatings are widely used in cutting tools and wear-resistant applications, but their industrial deposition typically requires use of external substrate, usually around 400~°C. Cathodic arc deposition offers room-temperature crystallization through energetic bombardment with multiply charged species. However, multiple process parameters are inherently coupled: external magnetic fields simultaneously increase both ion charge states (potential energy) and ion flux, while reactive gases alter plasma composition through charge-exchange collisions. This thesis investigates the TiAlN plasma to decouple the roles of ion flux, ion potential energy, and ion kinetic energy for room-temperature deposition  of TiAlN as a function of magnetic field strength (0--0.25~T), nitrogen pressure (0--0.3~Pa), and spatial position (10--20~cm from the plasma source).\\

Three main diagnostics tools were used to separate energy effects in the plasma process: a Langmuir probe for ion current density, a quartz crystal microbalance for deposited mass, and an energy-resolving mass spectrometer for charge-state-resolved ion energy distributions. Measurements in metallic mode (vacuum) and reactive mode (nitrogen background) characterized the transition from metallic to reactive deposition and quantified the contribution of ionized versus neutral nitrogen species to film growth.\\

The results show that increasing the magnetic field strength can amplify ion flux by up to a factor of eight for the same distance between the source and the measurement spot. In the reactive process, the presence of nitrogen further boosts ion flux, as charge-exchange collisions reduce the number of highly ionized species and increase the proportion of singly ionized species. Ion energy distribution functions measured by energy-resolved mass spectrometry, shift toward lower ionization levels, when operating in reactive mode.\\

This work extends the ion energy decoupling framework established for (V,Al)N coatings to the industrially relevant TiAlN system and demonstrates that ion flux control via magnetic field strength provides a viable route to tailoring film crystallinity without the need of external substrate temperature.


\chapter{Introduction and Literature Review}\label{chap:intro}

\section{Motivation}

Titanium aluminum nitride (Ti$_{1-x}$Al$_x$N) coatings are widely used, as protective coatings, particularly for cutting tools, because of their high hardness (25--35~GPa), thermal stability, and oxidation resistance \cite{paldey2003}. These properties depend on maintaining the metastable cubic B1 crystal structure, which provides superior mechanical performance compared to the thermodynamically wurtzite phase \cite{mayrhofer2006}. Conventional deposition methods, such as magnetron sputtering require substantial substrate heating to achieve dense, crystalline coatings. Cathodic arc deposition offers a key advantage: highly ionized metal plasmas with intrinsic high ion energies enable room-temperature crystallization through energetic condensation \cite{cathodic_arcs}.\\

The challenge is that cathodic arc processes depend on the interplay of several interdependent parameters. Ion energy has two components: kinetic energy from plasma expansion and potential energy from ionization. External magnetic fields can enhance ion charge states and ion flux, the latter increasing by up to an order of magnitude \cite{RN5}. Adding reactive gases such as nitrogen introduces charge-exchange collisions that alter plasma composition \cite{cathodic_arcs}. Understanding how these parameters individually influence film properties remains a fundamental challenge for predictive thin film deposition.

\section{State of the art}

Recent studies have made progress in understanding and decoupling ion energy effects in cathodic arc deposition. Unutulmazsoy et al.\ showed that applying an external magnetic field at the plasma source increases ion charge states (and thus potential energy), while applying a substrate bias adjusts kinetic energy \cite{unutulmazsoy}. Their work on (V,Al)N films demonstrated that crystalline films can be deposited at room temperature through energetic ion bombardment. Unutulmazsoy et al. stated that ``application of an external magnetic field also leads to an enhancement of the ion flux and hence the desired complete decoupling of the potential and kinetic energy effects requires further steps'' \cite{unutulmazsoy}. In their study, the magnetic field not only modified ion charge states but also increased ion flux by up to a factor of 10, making it impossible to isolate the effect of potential energy from that of ion flux intensity.\\


Kalanov et al.\ refined this approach using detailed energy-resolved mass spectrometry and systematic distance variation \cite{decoupling_kalanov_2025}. Their work on (V,Al)N films showed that the enhancement of room-temperature crystallinity correlates primarily with the increase in potential energy input. Notably, applying a DC bias to increase kinetic energy by approximately 30~eV did not produce the same crystallization effect, even though it increased the total ion energy. This demonstrated that potential energy plays a distinct role in promoting film crystallinity. Crucially, Kalanov et al. addressed the flux-energy coupling by varying the source-to-substrate distance from 8 to 20~cm: moving the substrate further away reduced the ion flux while maintaining similar ion charge states and energies, enabling direct comparison of potential energy effects at almost identical flux levels.\\

Despite these advances, key questions remain for extending this framework to other material systems. Both studies explored the effects of activated nitrogen in reactive deposition, which consists of both ionized and neutral (but excited) nitrogen species \cite{cathodic_arcs,bendikt2012}. While Unutulmazsoy et al. \cite{unutulmazsoy} focused on reactive deposition with nitrogen and Kalanov et al. \cite{decoupling_kalanov_2025} examined both metallic and reactive modes, a key question remains unanswered: what fraction of the activated nitrogen (nitrogen participating in the plasma process) consists of neutral species versus ionized species? This distinction is crucial for understanding the relative contributions of ions and neutrals to film growth, particularly in reactive mode where both species can participate in nitride formation.\\

For TiAlN specifically, which benefits from greater natural abundance and lower cost than vanadium-based nitrides, while comparative studies between cathodic arc and magnetron sputtering have shown that ion energy flux significantly affects texture and mechanical properties \cite{karimi_aghda_2023}, the flux-energy decoupling framework established for (V,Al)N has not been applied to this industrially important system.

\section{Objectives}

This thesis addresses above mentioned gaps in literature by systematically investigating the role of ion flux in Ti--Al--N thin film growth via pulsed filtered cathodic arc deposition. The specific objectives are:

\begin{enumerate}
	\item Characterize the ion flux and the total flux (flux of all particles arriving at the substrate) as a functions of external magnetic field strength and distance from the macroparticle filter. Furthermore, analyse the ion energy distributions of individual ionic species.
	
	\item Investigate the sweet spot from metallic to reactive mode by varying nitrogen pressure and quantify how this affects the ion-to-neutral ratio arriving to the substrate.
	
	\item Establish quantitative correlations between ion flux and total flux across the experimental parameter space of magnetic field, nitrogen pressure, and distance from the macroparticle filter.
	
	\item Characterize the crystallinity, film thickness, film composition of deposited TiAlN films at selected process parameters with film properties.
\end{enumerate}

The experimental approach combines multiple in situ plasma diagnostics with ex situ film characterization to extend the framework \cite{unutulmazsoy, decoupling_kalanov_2025} to both reactive mode operation and the industrially relevant TiAlN system. This multi-diagnostic strategy, applied across a systematic parameter space, enables investigation of two key aspects: (1) explore room temperature deposition of TiAlN films, and (2) the combined measurement of ion flux versus total deposited flux to distinguish the contribution of non-ionized activated nitrogen species and forming thin film structure.
