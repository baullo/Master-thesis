% !TeX root = main.tex
\chapter{Appendix: Experimental Methods supplementary}
\section{Longer QCM depositions}\label{long_qcm}
For larger mass loadings of the QCM, the linear approximation fails and the Z-match\texttrademark\ technique is used. This method, introduced by Lu and Lewis in 1972 on the basis of Miller and Bolef’s theoretical treatment \cite{Miller1968,Lu1972}, incorporates the acoustic properties of both the quartz and the deposited film via the acoustic impedance ratio  

\begin{equation}
	Z = \left(\frac{d_q \, \mu_q}{d_f \, \mu_f}\right)^{1/2},
\end{equation}

with $d$ and $\mu$ denoting the density and shear modulus of quartz ($q$) and film ($f$), respectively \cite{SQM160_manual}. In practice, the controller applies a correction function $f(Z)$ to the Sauerbrey relation,  

\begin{equation}
	m_f = \frac{N_{\text{AT}} \, d_q \, \pi r^2}{F_q^2} \cdot \Delta F \cdot f(Z),
\end{equation}

which compensates for the acoustic mismatch and extends the validity of thickness determination up to $\sim 0.4 F_q$.\\ 

\section{Holder assembly design}
\label{sec:holder_assembly_figure}

The integrated holder assembly (Figure~\ref{fig:holder_assembly}) enabled simultaneous QCM and ion probe measurements with minimal spatial separation. The aluminum mount positioned both diagnostics as close as practicable to bring the flux and plasma conditions as close as possible between measurement locations.


\begin{figure}[ht]
	\centering
	\includemedia[
	width=0.8\linewidth,
	activate=pageopen,
	passcontext,
	3Dmenu,
	3Dcoo=4.639472961425781 -21.05224609375 27.223785400390625,
	3Dc2c=92.426 9.840 -436.780,
	3Droo=200,
	]{
		\includegraphics[width=0.8\linewidth]{Figures/experimental methods/mounting assembly thingy.png}
	}{Figures/experimental methods/fun try out(1).u3d}
	\caption[Holder assembly for plasma diagnostics]{Holder assembly for In-Situ plasma Diagnostics: Integrated QCMs and Ion Collector Probe (interactive 3D model; static preview shown in non-Adobe viewers).}
	\label{fig:holder_assembly}
\end{figure}

\section{Data processing workflow}\label{appendix:data_workflow}\label{appendix:data_processing}

Experimental data were processed using custom Python scripts to ensure consistency and reproducibility across all measurements. This section describes the data handling procedures and processing workflows used throughout this work.


\subsection{Data organization and logbook system}

A central Excel logbook served as the reference for all measurements, with each measurement identified by a unique suffix and linked to its corresponding data files. The logbook recorded:
\begin{itemize}[noitemsep]
	\item Date and time of measurement
	\item Spatial parameters: distance from macroparticle filter (10--20~cm)
	\item Gas parameters: N$_2$ flow rate (MFC setting in sccm), working pressure (Pa)
	\item Vacuum system parameters: cryopump gate valve position, base pressure
	\item Power supply settings: arc voltage and current, EM-coil voltage and current
	\item Pulse characteristics: frequency (Hz), pulse width (ms), number of pulses
	\item QCM frequencies: initial ($f_0$) and final ($f_1$) values for mass determination
\end{itemize}

Associated oscilloscope waveforms were stored as CSV files for ion current analysis, with filenames linked to the logbook suffix for traceability.

\begin{figure}
	\centering
	\includegraphics[width=0.7\linewidth]{Figures/appendix/14cm-0.25t-n-0.1pa_Ti}
	\caption[Charge-state-resolved ion energy distribution functions for Ti]{Charge-state-resolved ion energy distribution functions for Ti$^+$ through Ti$^{4+}$ measured at 14~cm distance with 0.25~T magnetic field and 0.1~Pa N$_2$ pressure. Mean kinetic energies and standard deviations are shown in the legend.}
	\label{fig:ERMS-appendix}
\end{figure}
\begin{figure}
	\centering
	\includegraphics[width=0.7\linewidth]{Figures/appendix/14cm-0.25t-n-0.1pa_Al}
	\caption[Charge-state-resolved ion energy distribution functions for Al]{Charge-state-resolved ion energy distribution functions for Al$^+$ through Al$^{3+}$ measured at 14~cm distance with 0.25~T magnetic field and 0.1~Pa N$_2$ pressure. Mean kinetic energies and standard deviations are shown in the legend.}
\end{figure}


\subsection{Ion current data processing}

Ion current waveforms were recorded using a Tektronix MSO64 oscilloscope. For each measurement condition, multiple pulses were recorded and processed as follows:

\begin{enumerate}
	\item \textbf{Pulse averaging}: Individual pulse waveforms were averaged to obtain the mean ion current waveform.
	\item \textbf{Time-integrated current}: The mean ion current over the pulse duration (0--1~ms) was calculated by integrating the averaged waveform and dividing by the pulse width.
\end{enumerate}

The Python script automatically matched logbook entries to oscilloscope CSV files and compiled all parameters into CSV files for analysis.

\subsection{QCM data processing}

QCM frequency measurements were recorded before and after each set of pulses. The deposited mass was calculated using the Sauerbrey equation (Equation~\ref{eq:sauerbrey}). The mass deposition rate was obtained by dividing by the number of pulses and the crystal area.


\subsection{ERMS data processing}

The ERMS data processing was implemented in Python and performed the following operations on the raw spectral data. The underlying measurement principles and correction methodologies are described in Section~\ref{sec:QMS}, while this section focuses on the computational implementation.

\begin{enumerate}
	\item \textbf{Spectral integration}: Raw energy distribution functions (EDFs) for each $M/Q$ value were averaged over two measurement windows of 20 ms duration each. This averaging improves signal-to-noise ratio while maintaining synchronization with the pulsed arc operation (1 ms pulse duration, 5 Hz repetition rate).
	
	\item \textbf{Mass transmission correction}: Correction factors $T(M/Q)$ were applied to the raw ion signal intensities according to Equation~\ref{eq:mass_transmission}. The correction function accounts for the mass-dependent detection efficiency of the quadrupole mass filter, which systematically affects lighter ions (Al) differently than heavier ions (Ti). The transmission function was determined through calibration measurements and validated against the known cathode stoichiometry (Ti$_{0.75}$Al$_{0.25}$).
	
	\item \textbf{Energy conversion}: The ERMS measures energy-per-charge ($E/Q$) distributions. For multiply charged ions, the measured $E/Q$ values were multiplied by the charge state $Q$ to obtain the actual ion energy in eV (Equation~\ref{eq:energy_conversion}). This conversion was performed for charge states $Q = 1^+, 2^+, 3^+$ (Al) and $Q = 1^+, 2^+, 3^+, 4^+$ (Ti).
	
	\item \textbf{Statistical extraction}: For each charge-state-resolved energy distribution, the following statistical parameters were calculated:
	\begin{itemize}[noitemsep]
		\item Mean ion energy $\langle E \rangle$ - weighted average over the distribution
		\item Standard deviation $\sigma_E$ - characterizing the energy spread
		\item Peak energy $E_{\text{peak}}$ - the most probable energy (distribution maximum)
	\end{itemize}
	The mean ion energy represents the kinetic energy of ions in that charge state.
	
	\item \textbf{Mean charge state calculation}: The mean charge state $\langle Q \rangle$ for each element was calculated as a weighted average over all detected charge states using Equation~\ref{eq:mean_charge_state}. This parameter is essential for converting the ion current measurements to particle flux (Section~\ref{sec:flux_calculations}).
	
	\item \textbf{Potential energy calculation}: The potential energy $E_{\text{pot}}$ for each charge state was calculated separately from the cohesive energy and ionization potentials of the cathode material, following Kalanov et al. \cite{decoupling_kalanov_2025}:
	\begin{equation}
		E_{\text{pot}}(Q) = E_{\text{coh}} + C \sum_{i=1}^{Q} E_{\text{ion},i}
	\end{equation}
	where $E_{\text{coh}}$ is the cohesive energy (4.85 eV for Ti, 3.39 eV for Al), $E_{\text{ion},i}$ are the successive ionization energies, and $C = 0.6$ is a factor accounting for the fraction of ionization energy deposited in the solid. The mean potential energy was then calculated as a weighted average over charge states.
	
	\item \textbf{Energy analysis output}: The kinetic energy (measured) and potential energy (calculated) for each element were exported along with their standard deviations. These independent energy components enable investigation of their respective effects on film structure and properties.
\end{enumerate}

The processed results were exported to CSV files containing: pattern identification, mean charge states, mean kinetic energies, mean potential energies, and charge-state-resolved energies for each element. These data were correlated with ion probe and QCM measurements for comprehensive flux and energy analysis. Example charge-state-resolved ion energy distribution functions for titanium and aluminum ions are shown in Figures~\ref{fig:ERMS-appendix} and the following figure, measured at 14~cm distance with 0.25~T magnetic field and 0.1~Pa N$_2$ pressure.

\section{Uncertainty analysis}\label{appendix:uncertainty}

This section documents the uncertainty analysis for measurements reported in this work. Uncertainties were determined from three primary sources: instrument specifications, statistical variations in repeated measurements, and propagation through derived quantities.

\subsection{Primary measurements}

\subsubsection{QCM mass measurements}

The QCM frequency measurements have a manufacturer-specified resolution that translates to a mass uncertainty of approximately $\pm 5.64 \times 10^{-10}$ g per measurement. This corresponds to a relative uncertainty of approximately 0.1\% for typical deposited masses. This instrumental precision is negligible compared to other uncertainty sources in this work.\\

The number of pulses (64 in most cases) was controlled digitally and has negligible counting uncertainty. The crystal area (1.54 cm$^2$) is specified by the manufacturer.

\subsubsection{Ion current measurements}

Ion current measurements exhibited significant variability due to the non-stationary nature of cathodic arc plasmas. The standard deviation of time-averaged current within individual pulses ranged from 33\% to 68\% of the mean value (relative uncertainty), with a mean of 45\% across all conditions measured.\\

\begin{figure}[h]
	\centering
	\includegraphics[width=0.7\linewidth]{Figures/experimental methods/ion_probe_voltage}
	\caption[Example ion current waveform showing variability within a pulse]{Example ion current waveform showing variability within a pulse. The voltage signal measured across the 400~$\Omega$ resistor is proportional to the ion current. Rapid transients at pulse ignition (0~ms) and termination (1~ms) are excluded from the time-averaged current calculation.}
	\label{fig:ion_current_waveform_appendix}
\end{figure}

This large variability arises from cathode spot motion and splitting or extinction events, plasma potential fluctuations during pulse ignition and termination, and charge-exchange collisions with background gas that depend on pressure. The oscilloscope (Tektronix MSO64) has a specified voltage measurement accuracy of less than 1\%, which is negligible compared to the plasma variability. Error bars on ion current measurements represent the within-pulse standard deviation and characterize the inherent variability of the arc discharge under nominally constant operating conditions.

\subsubsection{ERMS measurements}

Mean charge states determined from energy-resolved mass spectrometry (Equation~\ref{eq:mean_charge_state}) have uncertainties ranging from 14\% to 24\% (relative), with a mean of 20\%. This value is calculated as the weighted standard deviation of the charge state distribution:
\begin{equation}
	\sigma_Q = \sqrt{\sum_Q w_Q (Q - \langle Q \rangle)^2}
\end{equation}
where $w_Q = I_Q / \sum I_Q$ are the normalized signal intensities after mass transmission correction. This represents the spread of the charge state distribution and is used in error propagation for flux calculations.


\subsection{Derived quantities}


Ion flux (calculated from ion current and mean charge state) has significantly larger uncertainties. For ion flux $\Phi_i = I_{\text{ion}}/(Q \cdot e \cdot A)$, the relative uncertainty is
%
\begin{equation}
	\frac{\sigma_{\Phi_i}}{\Phi_i} = \sqrt{\left(\frac{\sigma_I}{I}\right)^2 + \left(\frac{\sigma_Q}{Q}\right)^2},
\end{equation}
%
where uncertainties were propagated using standard error propagation for uncorrelated variables. This quantity is dominated by the ion current uncertainty (33--68\%) and charge state uncertainty (14--24\%), resulting in combined uncertainties of 36--72\% depending on conditions.\\

Atomic flux was calculated by converting mass flux to atomic flux using the effective molar mass (determined from film composition measured by ex-situ EDX analysis) and Avogadro's number. Since composition uncertainties were not formally characterized, the reported atomic flux values do not include propagated uncertainties.

\subsection{Uncertainties not quantified}

Several systematic uncertainty sources were not explicitly quantified in this work. Spatial plasma uniformity across the measurement area is estimated to contribute less than 10\% based on probe size relative to plasma diameter. Long-term calibration drift is expected to be less than 5\% as all measurements were completed within 3 months. Temperature effects on QCM sensitivity are minimal since chamber temperature remained stable within $\pm 2$~$^\circ$C and the QCM temperature coefficient is less than 0.1\% per degree. Geometric alignment uncertainties in probe positioning of $\pm 2$ cm correspond to less than 25\% flux variation at typical distances.\\

These effects are expected to be small compared to the dominant uncertainty sources identified above but were not independently characterized.

\subsection{Summary}

Table~\ref{tab:uncertainty_summary} summarizes the typical uncertainties for key measured and derived quantities. All error bars in figures represent these uncertainties unless otherwise noted. The large uncertainties in ion-based measurements reflect the intrinsic variability of cathodic arc plasmas rather than instrumental limitations.

\begin{table}[h]
	\centering
	\caption[Summary of measurement uncertainties]{Summary of measurement uncertainties}
	\label{tab:uncertainty_summary}
	\begin{tabular}{lcc}
		\toprule
		\textbf{Quantity} & \textbf{Relative Uncertainty} & \textbf{Dominant Source} \\
		\midrule
		QCM mass & 0.1\% & Instrument precision \\
		Ion current & 33--68\% (mean: 45\%) & Plasma variability \\
		Mean charge state & 14--24\% (mean: 20\%) & Charge state spread \\
		Mass flux & 0.1\% & QCM precision \\
		Ion flux & 36--72\% & Ion current + charge state \\
		Atomic flux  & Not reported & Composition error not characterized \\
		\bottomrule
	\end{tabular}
\end{table}

\newpage

\section{Langmuir probe bias voltage characterization}\label{sec:bias_characterization}

Before plasma investigations were performed, the functionality of the custom-built ion current probe was verified by characterizing its voltage-current relationship and determining the appropriate operating voltage within the ion saturation regime. This characterization ensures that the probe operates in a regime where the collected current is dominated by ions while effectively repelling electrons.\\


The bias voltage measurements were conducted without nitrogen in the chamber, at a fixed distance of 10 cm away from the filter, and with an EM-coil strength of 0.25 T.\\


The relationship between the collected current $I$ and the bias voltage $V$ was analyzed using the modified Langmuir equation \cite{chen1984introduction}:
\begin{align}\label{eq:langmuir_appendix}
	I = I_{\text{sat}} \left( 1 - e^{-V/V_0}\right) + m \cdot V
\end{align}
where $I_{\text{sat}}$ is the ion saturation current, $V_0$ is a characteristic voltage, and the linear term $m \cdot V$ accounts for sheath expansion and collisional effects at higher bias voltages.\\


Figure~\ref{fig:bias_current_appendix} displays the measured ion current as a function of bias voltage. The blue data points represent experimental measurements, while the green curve shows the fit to Equation~\ref{eq:langmuir_appendix} and the red dashed line shows the limit between the transition and the saturation regimes of the ion probe.

\begin{figure}[h]
	\centering
	\includegraphics[width=0.9\linewidth]{Figures/results 1/bias_current_voltage_Saturation_thesis_version.png}
	\caption[Ion current probe biasing tests]{Measured ion current vs. bias voltage, showing fit of eq: \ref{eq:langmuir_appendix}. The vertical dashed line at 70 V marks the transition to full ion saturation. Process conditions: no gas, 10 cm from the macroparticle filter, 0.25 T EM-coil strength.}
	\label{fig:bias_current_appendix}
\end{figure}

As the bias voltage increases, the sheath surrounding the probe expands. At higher voltages the sheath becomes non-planar, causing the probe to attract ions from a broader solid angle beyond the immediate frontal area. This geometric effect, combined with ion-neutral collisions within the extended sheath region, produces a residual linear increase in current even in the saturation regime. The linear correction term $m \cdot V$ in Equation~\ref{eq:langmuir_appendix} captures these effects, ensuring accurate modeling across the full voltage range \cite[Chap.~7]{chen1984introduction}.\\


The experimental data were fitted using a nonlinear least-squares method in Python, yielding:
\[ I_{\text{sat}} = 6.17 \, \text{mA}, \quad V_0 = 1.8 \, \text{V}, \quad m = 0.023 \, \text{mA/V}. \]

The voltage-current characteristic exhibits two distinct regimes:

\begin{enumerate}
	\item \textbf{Transition Regime (0--70 V):}
	
	At lower bias voltages, the probe collects both ions and electrons. As the negative bias increases, electrons are progressively repelled, reducing their contribution to the measured current. This produces a rapid rise in the net ion current as electron repulsion becomes increasingly effective. The exponential term in Equation~\ref{eq:langmuir_appendix} dominates this regime, reflecting the transition from mixed electron-ion collection to predominantly ion collection.
	
	\item \textbf{Ion Saturation Regime (70--130 V):}
	
	Beyond approximately 70 V, the current curve resembles more a plateau, indicating that the probe has entered the ion saturation regime. At these voltages, the negative bias effectively repels all electrons, and the measured current represents the ion flux to the probe surface. The slight residual increase in current with voltage is captured by the linear term $m = 0.023$ mA/V, which accounts for sheath expansion and collisional effects as described above.
\end{enumerate}

Based on this characterization, a bias voltage of $-80$ V was selected for all subsequent plasma measurements. This voltage lies well within the ion saturation regime, ensuring complete electron repulsion.

\section{Ion current variation over different pulses}\label{sec:errorbar}

To analyze the pulse-to-pulse variation, approximately 30 single pulses were recorded for each magnetic field strength. The average ion current of each pulse was calculated over the 0--1~ms pulse interval. The final mean and standard deviation were then determined by combining these averages, allowing for an assessment of the differences between pulses. 

\begin{figure}[h!]
	\centering
	\includegraphics[width=0.75\textwidth]{"Figures/results 1/pulse_variability_plot_mean"}
	\caption[Pulse to pulse variation of mean ion current]{Pulse to pulse variation in mean ion current at varying magnetic fields (0.1~Pa N$_2$, 10~cm distance)}
	\label{fig:pulsevariabilityplotmean}
\end{figure}

\begin{table}[h!]
	\centering
	\begin{tabular}{ccccc}
		\toprule
		\textbf{EM-coil field (T)} & \textbf{Mean $I_{\text{ion}}$ (mA)}  & \textbf{Std (\%)} & \textbf{Range (mA)} & \textbf{\# Pulses} \\ 
		\midrule
		0.00 & 1.931  & 13.8 & [1.13, 2.26] & 30 \\ 
		0.05 & 1.622  & 37.1 & [1.16, 4.85] & 32 \\ 
		0.10 & 4.279  & 12.7 & [2.72, 4.88] & 31 \\ 
		0.15 & 6.323  & 6.6 & [4.99, 6.92] & 27 \\ 
		0.20 & 7.996  & 5.0 & [7.26, 8.78] & 22 \\
		0.25 & 9.460  & 3.8 & [8.61, 10.02] & 30 \\ 
		\bottomrule
	\end{tabular}
	\caption[Summary of pulse measurement statistics]{Summary of pulse measurement statistics for varying magnetic field strengths. Note: The 0.05~T data includes variability from unstable arc behavior at this field strength with a large outlier of 4.85~mA.}
	\label{tab:pulse_measurements_appendix}
\end{table}

In all other measurements, the oscilloscope averaged the voltage signal over multiple pulses to improve signal-to-noise ratio. Therefore, the error reported in subsequent sections primarily reflects variations in ion current within individual pulses, rather than differences between distinct pulses.


\chapter{Appendix: supplementary material}\label{sec:additional plots}

\begin{figure}[h]
	\centering
	\includegraphics[width=\textwidth]{"Figures/appendix/Pulse waveform appendix (0.1T)"}
	\caption[Pulse waveform at 100 V input]{Additional Pulse waveform with the triggering timings (a) and (b) for the Arc-PSU and the EM-coil PSU marked with the orange dashed line for 100V input}
	\label{fig:pulse-waveform-appendix}
\end{figure}

\begin{figure}[h]
	\centering
	\begin{subfigure}{.5\textwidth}
		\centering
		\includegraphics[width=\linewidth]{Figures/appendix/Distance_vs_Pressure_B0.00T_Mass.png}
		\caption{}
	\end{subfigure}%
	\begin{subfigure}{.5\textwidth}
		\centering
		\includegraphics[width=\linewidth]{Figures/appendix/Distance_vs_Pressure_B0.00T_IonCurrent.png}
		\caption{}
	\end{subfigure}
	\caption[Additional distance measurements]{Additional plot for Section \ref{subsec:distance}. QCM and Ion Probe measurements showing (a) the deposited mass after 64 pulses and (b) the ion current averaged over a single pulse, each plotted as a function of distance. Data shown for representative pressures, with error bars representing variation within pulses.}
	\label{fig:distance_0T_mass/ion}
\end{figure}

\begin{figure}[h]
	\centering
	\includegraphics[width=\linewidth]{Figures/appendix/MagField_ChargeState_correlation.png}
	\caption[Magnetic field correlation with charge state]{Mean charge state as a function of magnetic field strength for Ti and Al ions under representative conditions: (a) 10 cm distance in metallic mode (0 Pa), (b) 10 cm distance in reactive mode (0.3 Pa), and (c) 14 cm distance in reactive mode (0.1 Pa). The systematic increase in charge states with magnetic field strength occurs across all conditions, demonstrating the inherent coupling between magnetic confinement and ion charge states. This coupling prevents independent control of ion flux and potential energy, as discussed in Section~\ref{chap:discussion}.}
	\label{fig:magfield_vs_chargestate}
\end{figure}

\newpage

\section{Data availability}\label{appendix:code}

All raw data files, processed datasets, and analysis scripts are archived and available upon request. The main Python scripts used for data processing include:
\begin{itemize}[noitemsep]
	\item \href{https://github.com/baullo/Junk_Paul-Masterthesis-Code/blob/main/ion%20current%20and%20mass%20csv%20generator.py}{Ion current and mass csv generator.py}: Ion current waveform analysis (mean and std) and qcm mass inclusion if provided.
	\item \href{https://github.com/baullo/Junk_Paul-Masterthesis-Code/blob/main/Massspec%20final.py}{Massspec final.py}: ERMS data integration, mass transmission correction and energy states, mean charge states and potential aswell as kinetic energy extraction.
	\item \href{https://github.com/baullo/Junk_Paul-Masterthesis-Code/blob/main/xrd_plot_stacked.py}{XRD plot stacked.py}: plotting of stacked XRD plots from .txt files
	\item {Data sets created and used for the Thesis}
\end{itemize}
\textcolor{red}{dunno if i will modify the csv merger script and add it since it is too specific maybe}
\begin{table}[htbp]
	\centering
	\caption[Complete overview of experimental measurements]{Complete overview of experimental measurements. All distance/field/pressure combinations not explicitly listed were measured with ion current probe and QCM only. In total: 252 Ion+QCM measurements, 17 ERMS measurements, 4 deposited films}
	\label{tab:comprehensive_measurements}
	\small
	\renewcommand{\arraystretch}{1.3}
	\resizebox{\textwidth}{!}{
	\begin{tabular}{@{}ccccccc@{}}
		\toprule
		Distance & Magnetic Field & Pressure & \multicolumn{3}{c}{Measurements Performed} & Film \\
		\cmidrule(lr){4-6}
		(cm) & (T) & (Pa) & Ion/QCM & Mass Spec & Ex-situ & ID \\
		\midrule
		\multicolumn{7}{l}{\textit{\textbf{Systematic parameter scan:}}} \\
		\makecell[l]{12, 16, 18} & \makecell[c]{0, 0.05, 0.1,\\0.15, 0.2, 0.25} & \makecell[c]{0, 0.025, 0.05, 0.075,\\0.1, 0.2, 0.3} & \checkmark & -- & -- & -- \\
		\midrule
		\multicolumn{7}{l}{\textit{\textbf{Distance 10 cm --- standard conditions:}}} \\
		10 & \makecell[c]{0.05, 0.1,\\0.2} & all pressures$^{\dagger}$ & \checkmark & -- & -- & -- \\
		10 & 0 & 0.025, 0.05, 0.075, 0.2 & \checkmark & -- & -- & -- \\
		10 & 0.15 & 0.025, 0.05, 0.075, 0.2 & \checkmark & -- & -- & -- \\
		10 & 0.25 & 0.025, 0.05, 0.075 & \checkmark & -- & -- & -- \\
		\midrule
		\multicolumn{7}{l}{\textit{\textbf{Distance 10 cm --- with mass spectrometry:}}} \\
		10 & 0, 0.15 & 0, 0.1, 0.3 & \checkmark & \checkmark & -- & --\\
		10 & 0.25 & 0 & \checkmark & \checkmark & \checkmark & 003 \\
		10 & 0.25 & 0.1 & \checkmark & \checkmark & \checkmark & 004 \\
		10 & 0.25 & 0.2 & \checkmark & \checkmark & -- & -- \\
		10 & 0.25 & 0.3 & \checkmark & \checkmark & \checkmark & 002 \\
		\midrule
		\multicolumn{7}{l}{\textit{\textbf{Distance 14 cm --- standard conditions:}}} \\
		14 & 0, 0.05, 0.1, 0.2 & all pressures$^{\dagger}$ & \checkmark & -- & -- & -- \\
		14 & 0.15 & \makecell[c]{0, 0.025, 0.05,\\0.075, 0.2, 0.3} & \checkmark & -- & -- & -- \\
		14 & 0.25 & 0.025, 0.05, 0.075, 0.2 & \checkmark & -- & -- & -- \\
		\midrule
		\multicolumn{7}{l}{\textit{\textbf{Distance 14 cm --- with mass spectrometry:}}} \\
		14 & 0.15 & 0.1 & \checkmark & \checkmark & -- & -- \\
		14 & 0.25 & 0 & \checkmark & \checkmark & \checkmark & 008/010 \\
		14 & 0.25 & 0.1 & \checkmark & \checkmark & \checkmark & 011/013 \\
		14 & 0.25 & 0.2 & \checkmark & \checkmark & \checkmark & 014 \\
		14 & 0.25 & 0.3 & \checkmark & \checkmark & \checkmark & 009/011 \\
		\midrule
		\multicolumn{7}{l}{\textit{\textbf{Distance 20 cm --- standard conditions:}}} \\
		20 & \makecell[c]{0, 0.05, 0.1,\\0.15, 0.2} & all pressures$^{\dagger}$ & \checkmark & -- & -- & -- \\
		20 & 0.25 & 0.025, 0.05, 0.075, 0.1, 0.2 & \checkmark & -- & -- & -- \\
		\midrule
		\multicolumn{7}{l}{\textit{\textbf{Distance 20 cm --- with mass spectrometry:}}} \\
		20 & 0.25 & 0, 0.3 & \checkmark & \checkmark & -- & -- \\
		\bottomrule
		\multicolumn{7}{l}{\footnotesize $^{\dagger}$all pressures = 0, 0.025, 0.05, 0.075, 0.1, 0.2, 0.3~Pa} \\
		\multicolumn{7}{l}{\footnotesize Ion/QCM: Ion current probe and quartz crystal microbalance measured at the same time} \\
		\multicolumn{7}{l}{\footnotesize Mass Spec: Energy-resolved mass spectrometry (ERMS)} \\
		\multicolumn{7}{l}{\footnotesize Ex-situ: XRD, SEM, EDX, profilometry characterization of deposited films} \\
	\end{tabular}
	}
\end{table}

