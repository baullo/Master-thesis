\chapter{Discussion of Results}\label{chap:discussion}

This chapter interprets the experimental findings presented in Chapter~\ref{chap:results} and connects them to the theoretical framework established in Chapter~\ref{chap:theory}. The discussion is organized around three central themes: plasma transport and composition, film growth mechanisms, and the correlation between plasma parameters and film properties. Measurement limitations and comparison with previous (V,Al)N work are addressed in the final sections.

\section{Plasma Transport and Magnetic Field Effects}

\subsection{Magnetic Mirror Effect at Low Fields}

The anomalous decrease in both mass deposition rate and ion current at 0.05~T magnetic field (Figure~\ref{fig:metallic_mass/ion}) reflects a magnetic mirror effect at the entrance of the EM-coil. When plasma transitions from the weak fringing field ($\sim$0.01~T) outside the coil into the stronger field (0.05~T) inside, conservation of the magnetic moment $\mu = mv_\perp^2 / 2B$ causes electrons with significant perpendicular velocity components to be reflected \cite{magnetic_mirror}. The reflection condition is met when:
\begin{equation}
	\frac{v_\perp^2}{v_\parallel^2} > \frac{B_{\text{max}} - B_{\text{min}}}{B_{\text{min}}}
\end{equation}

For the transition from 0.01~T to 0.05~T, electrons with pitch angles greater than approximately 60$^\circ$ are reflected, creating a localized space-charge layer that retards ion flow. This temporarily reduces both ion flux and deposition rate despite the nominally increased magnetic confinement. At higher fields (0.1~T and above), the beneficial effects of plasma compression and enhanced ionization \cite{decoupling_kalanov_2025} overcome this mirror loss, restoring the monotonic increase expected from magnetic confinement.

The persistence of this effect across all distances and pressures indicates it is a fundamental feature of the plasma-coil interaction rather than a transient or distance-dependent phenomenon. Future system optimization could minimize this loss by tapering the magnetic field profile at the coil entrance, reducing the field gradient that drives electron reflection.

\subsection{Spatial Expansion and 1/r$^2$ Scaling}

In metallic mode, mass deposition rate follows approximate $1/r^2$ dependence (Figure~\ref{fig:distance_0.25T_mass/ion}a), consistent with free expansion of a quasineutral plasma from a point source \cite[Chap.~4.3]{cathodic_arcs}. The 20~cm flux is approximately 25--30\% of the 10~cm value, close to the geometric factor $(10/20)^2 = 0.25$. Small deviations from perfect $1/r^2$ scaling arise from the finite source size (cathode diameter 6.35~mm) and magnetic field effects that partially collimate the plasma plume.

Ion current also decreases with distance but more steeply than $1/r^2$ in reactive mode, reflecting the additional effects of charge-exchange neutralization. The spatial profiles of mass deposition rate and ion current progressively diverge as nitrogen pressure increases, indicating that charge-exchanged neutral metal atoms follow ballistic trajectories while ions can be guided by residual magnetic fields from the filter coils.

\section{Reactive Mode Plasma Chemistry}

\subsection{Charge-Exchange Collisions and Neutral Formation}

The dramatic pressure-dependent reduction in ion current (Figure~\ref{fig:distance_0.25T_mass/ion}b) occurs through charge-exchange collisions, in which fast metal ions transfer electrons to slow nitrogen molecules:
\begin{align}
	\text{Ti}^{2+} + \text{N}_2 &\rightarrow \text{Ti}^{+} + \text{N}_2^{+} \\
	\text{Ti}^{+} + \text{N}_2 &\rightarrow \text{Ti}^{0} + \text{N}_2^{+}
\end{align}

The resulting fast neutral metal atoms are invisible to the biased probe, leading to an apparent reduction in measured current even though the total metal flux remains substantial, as evidenced by QCM measurements (Figure~\ref{fig:distance_0.25T_mass/ion}a). The charge-exchange cross section increases with collision energy and is largest for highly ionized species \cite[Chap.~9.4]{cathodic_arcs}, explaining why the effect is most pronounced at short distances where metal ion energies are highest.

ERMS measurements confirm this mechanism: mean charge states decrease from $\langle Q \rangle \approx 2.3$ (metallic) to $\langle Q \rangle \approx 1.4$ (0.3~Pa reactive), with the charge-state distribution shifting toward lower ionization levels (Figures~\ref{fig:ERMS_distance0Pa} and \ref{fig:ERMS_distance0.3Pa}). Potential energy decreases proportionally with charge state, while kinetic energy shows modest variations reflecting the competing effects of elastic scattering and cathode spot chemistry changes \cite{bendikt2012}.

Additionally, nitrogen ions (N$^+$, N$_2^+$) contribute less current per particle than multiply charged metal ions because their charge states are predominantly 1+. A probe measuring 10~mA of Ti$^{2+}$ current corresponds to $3.2 \times 10^{16}$~ions$\cdot$s$^{-1}$, while the same 10~mA of N$^+$ represents $6.4 \times 10^{16}$~ions$\cdot$s$^{-1}$. This explains part of the apparent current reduction when nitrogen ions displace metal ions in the plasma composition.

\subsection{Cathode Poisoning and Spot Transitions}

The non-monotonic pressure dependence of mass deposition rate at 10~cm (Figure~\ref{fig:pressure_0.25T_mass/ion}a), with a maximum near 0.05--0.1~Pa, reflects competing effects of cathode poisoning and reactive film growth. At low nitrogen pressures, the cathode surface remains predominantly metallic (type-2 spots), with nitrogen adsorbing on the growing film rather than the cathode \cite[Chap.~9.2--9.3]{cathodic_arcs}. This enhances deposition through compound formation without significantly reducing the metal emission rate.

As pressure increases beyond 0.1~Pa, a compound layer (likely TiN or mixed TiAlN) forms on the cathode surface, transitioning to type-1 (poisoned) spots. Type-1 spots have lower erosion rates and altered emission characteristics, reducing the total material flux. Simultaneously, collisional deflection becomes significant: nitrogen molecules scatter metal ions through momentum transfer, broadening the spatial distribution and reducing flux at the measurement position \cite{boxman1995}.

At 14~cm and 20~cm, the pressure independence of both mass deposition rate and ion current (within uncertainties) indicates that the plasma has already undergone extensive scattering and equilibration. Collisional mean free paths at 0.1--0.3~Pa are on the order of 1--3~cm, so multiple collisions occur during transit to these distances. Further pressure increases produce no additional deflection because the plasma is already thermalized and broadly distributed.

\subsection{Activated Nitrogen Species}

EDX measurements (Table~\ref{tab:edx}) show substantial nitrogen incorporation in reactive mode films (33--39~at.\%), while ERMS detects only modest N$^+$/N$_2^+$ signals compared to metal ions. This imbalance suggests that nitrogen arrives predominantly through neutral channels: either ground-state N$_2$, vibrationally/electronically excited N$_2$*, or atomic nitrogen radicals N* produced by dissociative electron impact in the cathode spot plasma \cite{bendikt2012}.

The efficient nitrogen incorporation at modest pressures (0.1~Pa achieving 33~at.\% N) indicates that activated nitrogen species, not thermal N$_2$ molecules, drive the reactive deposition. The chemical reactivity of N* radicals and excited N$_2$* molecules is several orders of magnitude higher than ground-state N$_2$, enabling nitride formation even with substoichiometric nitrogen arrival rates relative to metal flux \cite{bendikt2012}.

The quantitative relationship between ion flux measurements and atomic flux calculations (Figure~\ref{fig:flux_vs_pressure}) shows systematic differences that require further investigation. Potential contributions include spatial gradients between diagnostics, calibration uncertainties in the flux conversion factors, and time-averaging effects between pulsed and continuous measurements. These systematic uncertainties do not affect the qualitative conclusions about pressure dependence and nitrogen incorporation mechanisms, but they preclude quantitative statements about ion-to-neutral arrival ratios at this stage.

\section{Film Growth Mechanisms and Microstructure}

\subsection{Energetic Condensation and Room-Temperature Crystallization}

ERMS measurements confirm that total ion energies ($E_{\text{kin}} + E_{\text{pot}}$) remain above 30~eV across all experimental conditions (Figures~\ref{fig:ERMS_distance0Pa} and \ref{fig:ERMS_distance0.3Pa}), well above the threshold for subplantation-driven densification \cite[Chap.~8.1]{cathodic_arcs}. In metallic mode, total energies reach 60--80~eV, while in reactive mode they remain in the 30--50~eV range despite charge-exchange reduction of potential energy.

This energetic condensation enables room-temperature film growth through two mechanisms: (1) shallow implantation of ions below the surface, occupying interstitial sites and displacing near-surface atoms through knock-on collisions, reducing porosity; and (2) localized atomic-scale heating from kinetic and potential energy deposition, enhancing adatom mobility on nanosecond timescales without global substrate heating \cite[Chap.~8.2]{cathodic_arcs}.

XRD results validate this mechanism. In metallic mode without nitrogen, films exhibit weak and broad diffraction (Figure~\ref{fig:xrd_stacked}, 0~Pa), consistent with limited crystallinity despite energetic ion bombardment. The low intensity and broad peak width suggest small crystallite size ($<$10~nm estimated from Scherrer broadening) or incomplete crystallization, typical of vapor-deposited alloys without substrate heating \cite{thornton1977}. The absence of the expected (101) reflection indicates either strong (002) texture or predominantly amorphous structure with small hexagonal domains.

In reactive mode, nitrogen incorporation fundamentally alters film growth. The appearance of sharp cubic TiAlN peaks at 0.1~Pa and their continued sharpening at higher pressures reflects progressive crystallization driven by the chemical driving force for nitride formation (formation enthalpy approximately -200~kJ/mol for TiN and AlN \cite{paldey2003}) combined with energetic ion bombardment. The complete disappearance of the hexagonal TiAl peak indicates full conversion rather than phase coexistence.

\subsection{Structure-Zone Model Analysis}

The Anders structure-zone model \cite{anders2010szm} predicts film microstructure based on normalized temperature $T^*$ and ion energy $E^*$. For room-temperature deposition ($T/T_m \approx 0.2$ for TiAlN), the model places metallic films in Zone~1 (porous, poorly crystalline) and predicts that energetic ion bombardment can shift the microstructure to Zone~T or Zone~2 (dense, crystalline).

XRD results support this prediction. Metallic films occupy Zone~1, characterized by insufficient adatom mobility for long-range ordering. The weak (002) peak suggests some short-range ordering but not fully developed crystallinity. Upon nitrogen addition, films transition to Zone~T/Zone~2, showing well-defined cubic peaks and intensity ratios approaching powder patterns.

Three factors drive this transition: (1) Chemical driving force from nitride formation provides approximately 2~eV per atom of stabilization energy \cite{paldey2003}, equivalent to raising the effective deposition temperature by several hundred degrees. (2) Energetic ion bombardment (30--50~eV total energy) provides localized heating and defect annealing through collision cascades. (3) Enhanced ion flux from magnetic confinement (up to $5 \times 10^{15}$~ions$\cdot$cm$^{-2}$$\cdot$s$^{-1}$ at 0.25~T) increases the ion-to-neutral arrival ratio, maximizing the fraction of film growth occurring through energetic ion processes rather than thermal vapor condensation.

The progressive sharpening of XRD peaks from 0.1~Pa to 0.3~Pa correlates with nitrogen saturation in EDX (33~at.\% $\rightarrow$ 39~at.\%). At 0.1~Pa, incomplete nitridation leaves residual metallic or mixed-phase regions that broaden diffraction peaks. At 0.2--0.3~Pa, full nitridation produces chemically homogeneous films with fewer compositional fluctuations, enabling larger coherent diffraction domains.

\subsection{Composition and Stoichiometry}

EDX measurements show systematic deviations from cathode stoichiometry. In metallic mode, the Ti:Al ratio shifts from 62.8:37.2 (cathode) to 79:21 (film). This deviation likely arises from: (1) Different erosion rates: titanium's higher melting point (1668°C vs 660°C) and cohesive energy may lead to preferential aluminum emission from the explosive cathode spots. (2) Ionization efficiencies: aluminum's lower ionization potentials (5.99 eV for Al$^{1+}$ vs 6.83 eV for Ti$^{1+}$) could favor Al ionization, but charge-state distributions show Ti$^{2+}$ is more abundant than Al$^{2+}$, suggesting ionization is not the primary factor. (3) Transport through the macroparticle filter: multiply charged ions with higher m/Q ratios experience stronger magnetic deflection, potentially causing differential transmission.

In reactive mode, the Ti:Al ratio shifts to 3.4:1, closer to the cathode composition. This suggests that nitrogen incorporation alters the emission process, possibly through cathode poisoning that reduces overall erosion rates and equalizes Ti/Al emission. The nitrogen saturation at 39~at.\% (0.2--0.3~Pa) is close to stoichiometric (Ti,Al)N (50~at.\% N), with the deficiency attributable to either measurement uncertainties in EDX or actual substoichiometric nitride phases.

The sharp nitrogen uptake between 0~Pa and 0.1~Pa (0~at.\% $\rightarrow$ 33~at.\%) indicates efficient reactive deposition. The arrival rate of activated nitrogen species (ions + neutrals) exceeds the metal flux, enabling rapid nitride formation. Above 0.2~Pa, nitrogen availability exceeds stoichiometric requirements, leading to saturation.

\section{Measurement Limitations and Uncertainties}

\subsection{Spatial Gradients Between Diagnostics}

Direct comparison between ion current and mass deposition requires careful consideration of spatial gradients. The ion probe (diameter 5~mm, area 0.196~cm$^2$) and QCM crystal (diameter 8~mm, area 0.5027~cm$^2$) were mounted as close as possible (Section~\ref{sec:holder_assembly_figure}), but plasma density gradients create position-dependent variations. Radial profiles in expanding cathodic arc plasmas typically show Gaussian-like distributions with characteristic widths of 5--10~cm at distances of 10--20~cm \cite[Chap.~6.2]{cathodic_arcs}.

At 10~cm distance, a 5~mm lateral separation between probe and QCM centers could result in flux differences of 10--20\% based on typical gradient scales. This partially explains the ion-to-total flux ratios deviating from unity in metallic mode (Table~\ref{tab:flux_ratios}). The ratios systematically increase with distance (0.8 at 10~cm, 1.5 at 14~cm), suggesting that spatial separation effects become more significant as the plasma expands and density gradients sharpen.

Future experiments should employ co-located diagnostics (probe mounted directly on QCM crystal face) or spatially resolved measurements (scanning probe arrays) to eliminate this systematic error.

\subsection{Film Thickness and Deposition Rate Inconsistencies}

Profilometry measurements (Table~\ref{tab:profilometry_results}) show apparent inconsistencies when comparing films 9 (0.3~Pa, 6000 pulses, 35~nm) and 11 (0.3~Pa, 8000 pulses, 42~nm). The thickness increase of only 20\% for a 33\% increase in pulses suggests either: (1) Non-linear deposition behavior at high nitrogen pressures, possibly due to resputter or etching processes. (2) Position-dependent variations: films 9 and 11 were deposited in separate runs, potentially with slight differences in substrate positioning. (3) Pressure measurement uncertainties: the 0.2 and 0.3~Pa setpoints may have actual pressures varying by $\pm$0.05~Pa due to MFC drift or pumping speed fluctuations.

Additionally, Figure~\ref{fig:pressure_0.25T_mass/ion}a predicts approximately constant mass flux at 14~cm for 0.1--0.3~Pa, but film thicknesses show 59~nm (0.1~Pa, 8000 pulses) vs 42~nm (0.3~Pa, 8000 pulses) by profilometry. This 30\% difference exceeds the QCM uncertainties ($\pm$10\%) and suggests real variations in deposition rate that are not fully captured by the averaged QCM measurements. Pulse-to-pulse variations (Table~4.1, standard deviations 10--40\%) contribute to this scatter, with the highest variability at intermediate magnetic fields where arc stability is most sensitive to minor perturbations.

These inconsistencies highlight the challenges of absolute rate measurements in pulsed systems with multiple coupled parameters. The trends (magnetic field enhancement, pressure effects) are robust, but quantitative predictions for specific conditions require averaging over larger datasets or longer deposition times.

\subsection{ERMS Calibration and Mass Transmission}

Energy-resolved mass spectrometry in pulsed cathodic arc environments faces unique challenges. The non-stationary nature of cathode spots produces pulse-to-pulse variations in ion flux and energy distributions, complicating signal averaging and calibration \cite{cathodic_arcs}. The 40~ms acquisition windows (two 20~ms windows per pulse) average over multiple spot events but may not fully capture rare high-energy ions or low-abundance charge states.

Mass transmission corrections were applied based on manufacturer-provided functions, but these are calibrated for steady-state plasmas and may not account for the time-varying potential distributions in pulsed arcs. The systematic uncertainty in absolute ion flux from ERMS is estimated at $\pm$30\%, larger than the $\pm$15\% uncertainty in probe current measurements. However, relative comparisons (metallic vs reactive mode, distance scaling) remain valid because systematic errors cancel when taking ratios.

The charge-state distributions (Ti: 1+--4+, Al: 1+--3+, N: 1+ only) are consistent with previous cathodic arc measurements \cite{RN6,unutulmazsoy}, providing confidence in the ERMS data quality despite absolute calibration uncertainties.

\section{Comparison with (V,Al)N Literature}

This work extends the energy-flux decoupling framework established by Unutulmazsoy et al.\ \cite{unutulmazsoy} and Kalanov et al.\ \cite{decoupling_kalanov_2025} for (V,Al)N to the industrially relevant TiAlN system. Key similarities and differences are summarized:

\textbf{Similarities:}
\begin{itemize}
	\item Magnetic field enhancement of ion flux by factors of 5--10 is consistent across both material systems.
	\item Room-temperature crystallization enabled by total ion energies above 30--40~eV validates the energetic condensation model for transition-metal nitrides.
	\item Charge-state reductions in reactive mode (20--50\%) match the (V,Al)N observations.
\end{itemize}

\textbf{Differences:}
\begin{itemize}
	\item TiAlN shows more pronounced pressure dependence at 10~cm distance compared to (V,Al)N, possibly reflecting different charge-exchange cross sections for Ti vs V ions with N$_2$.
	\item The magnetic mirror effect at 0.05~T was not reported for (V,Al)N, likely because that study used different EM-coil geometries and field profiles.
	\item TiAlN crystallizes more readily at lower nitrogen pressures (0.1~Pa) than (V,Al)N, consistent with the higher thermodynamic stability of TiN vs VN (formation enthalpies -338 vs -217~kJ/mol \cite{paldey2003}).
\end{itemize}

The successful application of the decoupling approach to TiAlN demonstrates the generality of the framework for understanding ion-assisted film growth in cathodic arc systems. The combined probe + QCM + ERMS diagnostic strategy provides sufficient information to disentangle flux, energy, and composition effects, enabling predictive process control.

\section{Implications for Process Optimization}

The results demonstrate three independent control parameters for tailoring TiAlN film properties:

\textbf{Magnetic field strength (0--0.25~T):} Primary control of ion flux. Increasing field from 0~T to 0.25~T enhances flux by factors of 5--8 without dramatically changing ion energies or charge states. Optimal for maximizing deposition rate while maintaining energetic condensation conditions.

\textbf{Nitrogen pressure (0--0.3~Pa):} Primary control of stoichiometry and phase. Pressures above 0.1~Pa ensure complete nitridation and cubic phase formation. Pressures below 0.05~Pa produce mixed metallic/nitride films with reduced crystallinity.

\textbf{Substrate distance (10--20~cm):} Balances flux (decreases as $1/r^2$) against ion-to-neutral ratio (decreases with distance due to charge exchange). Short distances (10~cm) maximize flux but introduce stronger spatial gradients. Intermediate distances (14~cm) provide acceptable flux with improved uniformity.

For applications requiring maximum hardness and wear resistance, the optimal conditions are 0.25~T field, 0.2--0.3~Pa nitrogen, and 10--14~cm distance, producing fully crystalline cubic TiAlN with total ion energies of 30--50~eV and deposition rates of 0.5--1.0~nm/pulse.

