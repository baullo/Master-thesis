\chapter{Discussion of Results}\label{chap:discussion}

This chapter interprets the experimental findings presented in Chapter~\ref{chap:results} and connects them to the theoretical framework established in Chapter~\ref{chap:theory}. The discussion is organized around three central themes: plasma transport and composition, film growth mechanisms, and the correlation between plasma parameters and film properties. Measurement limitations and comparison with previous (V,Al)N work are addressed in the final sections.

\section{Plasma Transport and Magnetic Field Effects}

\subsection{Magnetic Mirror Effect at Low Fields}

The anomalous decrease in both mass deposition rate and ion current at 0.05~T magnetic field (Figure~\ref{fig:metallic_mass/ion}) reflects a magnetic mirror effect at the entrance of the EM-coil. When plasma transitions from the weak fringing field ($\sim$0.01~T) outside the coil into the stronger field (0.05~T) inside, conservation of the magnetic moment $\mu = mv_\perp^2 / 2B$ causes electrons with significant perpendicular velocity components to be reflected \cite{magnetic_mirror}. The reflection condition is met when:
\begin{equation}
	\frac{v_\perp^2}{v_\parallel^2} > \frac{B_{\text{max}} - B_{\text{min}}}{B_{\text{min}}}
\end{equation}

For the transition from 0.01~T to 0.05~T, electrons with pitch angles greater than approximately 60$^\circ$ are reflected, creating a localized space-charge layer that retards ion flow. This temporarily reduces both ion flux and deposition rate despite the nominally increased magnetic confinement. At higher fields (0.1~T and above), the beneficial effects of plasma compression and enhanced ionization \cite{decoupling_kalanov_2025} overcome this mirror loss, restoring the monotonic increase expected from magnetic confinement.

The persistence of this effect across all distances and pressures indicates it is a fundamental feature of the plasma-coil interaction rather than a transient or distance-dependent phenomenon. Future system optimization could minimize this loss by tapering the magnetic field profile at the coil entrance, reducing the field gradient that drives electron reflection.

\subsection{Spatial Expansion and 1/r$^2$ Scaling}

In metallic mode, mass deposition rate follows approximate $1/r^2$ dependence (Figure~\ref{fig:distance_0.25T_mass/ion}a), consistent with free expansion of a quasineutral plasma from a point source \cite[Chap.~4.3]{cathodic_arcs}. The 20~cm flux is approximately 25--30\% of the 10~cm value, close to the geometric factor $(10/20)^2 = 0.25$. Small deviations from perfect $1/r^2$ scaling arise from the finite source size (cathode diameter 6.35~mm) and magnetic field effects that partially collimate the plasma plume.

Ion current also decreases with distance but more steeply than $1/r^2$ in reactive mode, reflecting the additional effects of charge-exchange neutralization. The spatial profiles of mass deposition rate and ion current progressively diverge as nitrogen pressure increases, indicating that charge-exchanged neutral metal atoms follow ballistic trajectories while ions can be guided by residual magnetic fields from the filter coils.

\section{Reactive Mode Plasma Chemistry}

\subsection{Charge-Exchange Collisions and Neutral Formation}

The pressure dependence of ion current exhibits complex behavior at close distances (Figure~\ref{fig:pressure_0.25T_mass/ion}b). At 10~cm, the ion current initially increases from 0 to approximately 0.05~Pa before decreasing at higher pressures. This non-monotonic behavior reflects two competing mechanisms operating in different pressure regimes.

At low nitrogen pressures (0--0.1~Pa), the initial increase may result from enhanced plasma generation at the cathode. Nitrogen incorporation into the cathode surface can alter spot dynamics, potentially increasing the erosion rate and metal ion production. Simultaneously, ion molecule reactions begin to produce nitrogen ions (N$^+$, N$_2^+$) that contribute to the total measured probe current.

At higher pressures (above 0.1~Pa), charge exchange collisions dominate, reducing the metal ion contribution to the measured current:
\begin{align}
	\text{Ti}^{2+} + \text{N}_2 &\rightarrow \text{Ti}^{+} + \text{N}_2^{+} \\
	\text{Ti}^{+} + \text{N}_2 &\rightarrow \text{Ti}^{0} + \text{N}_2^{+}
\end{align}

The resulting fast neutral metal atoms are invisible to the biased probe, leading to apparent current reduction even though total metal flux (measured by QCM) remains substantial. The charge-exchange cross section increases with collision energy and is largest for highly ionized species \cite[Chap.~9.4]{cathodic_arcs}.\\

ERMS measurements confirm this mechanism: average mean charge states decrease from $\langle Q \rangle \approx 2.4$ (metallic) to $\langle Q \rangle \approx 1.6$ (0.3~Pa reactive), with the charge-state distribution shifting toward lower ionization levels (Figures~\ref{fig:ERMS_distance0Pa} and \ref{fig:ERMS_distance0.3Pa}). Potential energy decreases proportionally with charge state, while kinetic energy shows modest variations reflecting the competing effects of elastic scattering and cathode chemistry changes \cite{bendikt2012}.\\

Additionally, nitrogen ions (N$^+$, N$_2^+$) contribute less current per particle than multiply charged metal ions because their charge states are predominantly 1+. A probe measuring 10~mA of Ti$^{2+}$ current corresponds to $3 \times 10^{16}$~ions$\cdot$s$^{-1}$, while the same 10~mA of N$^+$ represents $6 \times 10^{16}$~ions$\cdot$s$^{-1}$. This explains part of the apparent current reduction when nitrogen ions displace metal ions in the plasma composition.


\subsection{Cathode Poisoning and Spot Transitions}

The non-monotonic pressure dependence of mass deposition rate at 10~cm (Figure~\ref{fig:pressure_0.25T_mass/ion}a), with a maximum near 0.05--0.1~Pa, reflects competing effects of cathode poisoning and reactive film growth. At low nitrogen pressures, the cathode surface remains predominantly metallic (type-2 spots), with nitrogen adsorbing on the growing film rather than the cathode \cite[Chap.~9.2--9.3]{cathodic_arcs}. This enhances deposition through compound formation without significantly reducing the metal emission rate.\\

As pressure increases beyond 0.1~Pa, a compound layer (TiN or mixed TiAlN) forms on the cathode surface, transitioning to type-1 (poisoned) spots. Type-1 spots have lower erosion rates and altered emission characteristics, reducing the total material flux. Simultaneously, collisional deflection becomes significant: nitrogen molecules scatter metal ions through momentum transfer, broadening the spatial distribution and reducing flux at the measurement position \cite{boxman1995}.\\

At 14~cm and 20~cm, the pressure independence of both mass deposition rate and ion current (within uncertainties) indicates that the plasma has already undergone extensive scattering and equilibration. Collisional mean free paths at 0.1--0.3~Pa are on the order of centimeters, so multiple collisions occur during transit to these distances. Further pressure increases produce no additional deflection because the plasma is already thermalized and broadly distributed.

\subsection{Activated Nitrogen Species}

EDX measurements (Table~\ref{tab:edx}) show substantial nitrogen incorporation in reactive mode films (33--39~at.\%), while ERMS detects only modest N$^+$/N$_2^+$ signals (10--28~at.\%) compared to metal ions. This imbalance suggests that nitrogen arrives predominantly through neutral channels: either ground-state N$_2$, vibrationally/electronically excited N$_2$*, or atomic nitrogen radicals N* produced by dissociative electron impact in the cathode spot plasma \cite{bendikt2012}.\\

The quantitative relationship between ion flux measurements and atomic flux calculations (Figure~\ref{fig:flux_vs_pressure}) shows systematic differences that require further investigation. Potential contributions include calibration uncertainties in the flux conversion factors and time averaging effects between pulsed and continuous measurements.

\section{Film Growth Mechanisms and Microstructure}

\subsection{Energetic Condensation and Room-Temperature Crystallization}

ERMS measurements confirm that total ion energies ($E_{\text{kin}} + E_{\text{pot}}$) remain above 30~eV across all experimental conditions (Figures~\ref{fig:ERMS_distance0Pa} and \ref{fig:ERMS_distance0.3Pa}), well above the threshold for subplantation-driven densification \cite[Chap.~8.1]{cathodic_arcs}. In metallic mode, total energies reach 60--80~eV, while in reactive mode they remain in the 30--50~eV range despite charge-exchange reduction of potential energy.\\

This energetic condensation enables room temperature film growth through two mechanisms: (1) shallow implantation of ions below the surface, occupying interstitial sites and displacing near-surface atoms through knock-on collisions and (2) localized atomic-scale heating from kinetic and potential energy deposition, enhancing adatom mobility on nanosecond timescales without global substrate heating \cite[Chap.~8.2]{cathodic_arcs}.\\

XRD results validate this mechanism. In metallic mode without nitrogen, films exhibit weak and broad diffraction (Figure~\ref{fig:xrd_stacked}, 0~Pa), consistent with limited crystallinity despite energetic ion bombardment. The low intensity and broad peak width suggest small crystallite size or incomplete crystallization, typical of vapor deposited alloys without substrate heating \cite{thornton1977}. The absence of the expected (101) reflection indicates either strong (002) texture or predominantly amorphous structure with small hexagonal domains.\\

In reactive mode, nitrogen incorporation fundamentally alters film growth. The appearance of sharp cubic TiAlN peaks at 0.1~Pa and their continued sharpening at higher pressures reflects progressive crystallization driven by the chemical driving force for nitride formation (formation enthalpy approximately -200~kJ/mol for TiN and AlN \cite{paldey2003}) combined with energetic ion bombardment. The complete disappearance of the hexagonal TiAl peak indicates full conversion rather than phase coexistence.

\subsection{Composition and Stoichiometry}

EDX measurements show systematic deviations from cathode stoichiometry. In metallic mode, the Ti:Al ratio shifts from 62.8:37.2 (cathode) to 79:21 (film). This deviation likely arises from: (1) Different erosion rates: titanium's higher melting point (1668°C vs 660°C) and cohesive energy may lead to preferential aluminum emission from the explosive cathode spots. (2) Ionization efficiencies: aluminum's lower ionization potentials (5.99 eV for Al$^{1+}$ vs 6.83 eV for Ti$^{1+}$) could favor Al ionization, but charge state distributions show Ti$^{2+}$ is more abundant than Al$^{2+}$, suggesting ionization is not the primary factor. (3) Transport through the macroparticle filter: multiply charged ions with higher m/Q ratios experience stronger magnetic deflection, potentially causing differential transmission.\\

In reactive mode, the Ti:Al ratio shifts to 3.4:1, closer to the cathode composition. This suggests that nitrogen incorporation alters the emission process, possibly through cathode poisoning that reduces overall erosion rates and equalizes Ti/Al emission. The nitrogen saturation at 39~at.\% (0.2--0.3~Pa) is close to stoichiometric (Ti$_x$,Al$_{1-x}$)N (50~at.\% N), with the deficiency attributable to a multitude of factors.


