\chapter{Discussion of Results}\label{chap:discussion}
This chapter interprets the experimental findings presented in Chapter~\ref{chap:results} and connects them to the theoretical framework established in Chapter~\ref{chap:theory}. The discussion is organized around two central themes: film growth mechanisms and plasma transport.

\section{Film Growth and Microstructure}\label{sec:film_analysis}

XRD results demonstrate that nitrogen incorporation enables room-temperature crystallization of TiAlN. The metallic TiAl film (0 Pa) exhibits only weak hexagonal reflections, while all reactive mode films (0.1--0.3 Pa nitrogen) develop sharp cubic TiAlN (111) and (200) reflections.\\

XRR measurements show the measured densities are in good agreement with reference values for their respective crystal structures (Table~\ref{tab:xrr_results}). The metallic TiAl film shows 4.00 g$\cdot$cm$^{-3}$, while reactive mode films reach 4.65--4.80 g$\cdot$cm$^{-3}$, similar to the cubic TiAlN reference value of 4.8 g$\cdot$cm$^{-3}$ \cite{ICDD_TiAlN}. \\

EDX measurements show systematic deviations from cathode stoichiometry (Table~\ref{tab:edx}). In metallic mode, the Ti:Al ratio shifts from 62.8:37.2 (cathode) to 79:21 (film), likely arising from differential erosion rates, ionization efficiencies, or transport through the magnetic filter. In reactive mode, the ratio shifts to approximately 3.4:1, closer to cathode composition. Nitrogen saturation at 39 at.\% (0.2--0.3 Pa) approaches stoichiometric (Ti,Al)N (50 at.\% N).\\

XRR and profilometry reveal 5--8.5 nm thickness gradients between measurement positions (Tables~\ref{tab:profilometry_results}--\ref{tab:xrr_results}), attributable to plasma transport through the curved magnetic filter. The 90 degree filter guides electrons along field lines while ions follow electrostatically. Different species follow slightly different trajectories, leading to potential spatial variations in flux and composition \cite{Bilek1996FilterTransport}.

\section{Plasma transport and magnetic field effects}\label{sec:magnetic_effect}

The anomalous decrease in both mass deposition rate and ion current at 0.05 T magnetic field (Figure~\ref{fig:metallic_mass/ion}) reflects a magnetic mirror effect at the entrance of the EM-coil, where plasma transitions from the weak fringing field (approximately 0.01 T) outside into the stronger field (0.05 T) inside. Electrons with sufficient perpendicular velocity components are reflected, creating a localized space-charge layer that temporarily retards ion flow \cite{cathodic_arcs}. At higher fields (0.1 T and above), the beneficial effects of plasma compression and enhanced ionization overcome this mirror loss, restoring monotonic increase. This effect persists across all distances and pressures, indicating it is a fundamental feature of the plasma-coil interaction.\\

In metallic mode, mass deposition rate follows approximate $1/r^2$ dependence (Figure~\ref{fig:distance_0.25T_mass/ion}a), consistent with free plasma expansion. Ion current decreases more steeply in reactive mode due to charge-exchange neutralization. The magnetic field enhancement increases both mass flux and ion current by factors of 5--8 from zero field to 0.25 T (Figure~\ref{fig:mag_field_14cm_mass/ion}), demonstrating effective plasma enhancement.


\section{Reactive mode plasma chemistry}

The pressure dependence of mass flux exhibits complex behavior at 10 cm (Figure~\ref{fig:pressure_0.25T_mass/ion}b), with an initial decrease from 0 to 0.05 Pa before decreasing at higher pressures. At low nitrogen pressures (0--0.1 Pa), enhanced plasma generation at the cathode from altered spot dynamics and nitrogen ion production contribute to increased measured mass flux. The same cannot be said for the ion current, which is gradually decreasing with increasing nitrogen pressure. This is due to the many reactions possible with the background gas, reduction of the metallic species can lead to ionization, dissociation and excitations.\\

ERMS measurements confirm this mechanism: mean charge states decrease from $\langle Q \rangle \approx 2.4$ (metallic) to $\approx 1.6$ (0.3 Pa reactive), with charge-state distributions shifting toward lower ionization levels (Figures~\ref{fig:ERMS_10cm_pressure}--\ref{fig:ERMS_14cm_pressure}). This represents a 35\% reduction in average charge state. Importantly, total ion energy (kinetic plus potential) remains above 30 eV across all conditions, sufficient for subplantation processes. \\

Table~\ref{tab:plasma_film_correlation} summarizes the measured plasma parameters and resulting film properties across the pressure range at 14 cm distance with 0.25 T magnetic field.

\begin{table}[h]
	\centering
	\caption{Plasma parameters and film properties correlation at 14 cm, 0.25 T}
	\label{tab:plasma_film_correlation}
	\begin{tabular}{c c c c c c}
		\toprule
		Pressure & Ion Flux & Atomic Flux & $\langle Q \rangle$  & Density & XRD Quality \\ 
		(Pa) & ($10^{17}$ cm$^{-2}$s$^{-1}$) & ($10^{17}$ cm$^{-2}$s$^{-1}$) &  & (g cm$^{-3}$) & \\ 
		\midrule
		0.0 & 0.7 & 0.75 & 2.4  & 4.00 & weak hex \\ 
		0.1 & 0.95 & 1.0 & 2.0  & 4.80 & strong cubic \\ 
		0.2 & 1.05 & 1.15 & 1.8  & 4.80 & strong cubic \\ 
		0.3 & 1.0 & 0.95 & 1.6  & 4.65 & strong cubic \\ 
		\bottomrule
	\end{tabular}
\end{table}

The ionization fraction $f_{\text{ion}} = \Gamma_{\text{ion}}/\Phi_{\text{total}}$ remains between 0.9 and 1.1 across all pressures. This high ionization persists even as mean charge states decrease by 35\% from metallic to reactive mode, confirming that charge-exchange reduces individual ion charges without substantial neutralization of the metal plasma.\\

Atomic flux exhibits non-monotonic pressure dependence with maximum at 0.2 Pa (54\% increase over metallic mode), while ion flux remains approximately constant. This indicates that reactive film formation enhances mass deposition through incorporation of nitrogen atoms from the gas phase, while the metal ion flux remains determined primarily by arc current and magnetic field. The flux minimum at 0.3 Pa may relate to pressure-dependent sticking coefficients, as observed by Mahieu et al. for sputtered atoms \cite{Mahieu2008Sticking}. Further systematic pressure variation studies are needed to clarify this effect.

\subsection{Activated nitrogen species}

EDX measurements (Table~\ref{tab:edx}) show substantial nitrogen incorporation in reactive mode films (33--39 at.\%), while ERMS detects only modest N$^+$/N$_2^+$ signals (10--28 at.\%) compared to metal ions. The imbalance suggests that nitrogen has some excited species but since ERMS estimates the nitrogen contents via the overall species counts, therefore the imbalance can only be used as a hint towards activated nitrogen.

\newpage
\section{Summary}

This work established a comprehensive methodology for characterizing magnetically filtered cathodic arc plasmas using combined QCM and ion probe measurements, supplemented by ERMS charge-state analysis and film characterization via XRD, XRR, and EDX. The approach was validated for the TiAl cathode system across metallic and reactive nitrogen modes, demonstrating near-unity ionization fractions ($f_{\text{ion}} = 0.92$--1.05) and revealing the dominant role of ions in mass transport.\\

The methodology enabled quantification of ion and atomic fluxes under varying magnetic field (0--0.25 T), distance (10--14 cm), and nitrogen pressure (0--0.3 Pa) conditions. Magnetic field enhancement factors of 5--8 were demonstrated, while charge-exchange reactions in reactive mode reduced mean charge states by 35\% without equally decreasing the ion flux. Film analysis confirmed that high ionization fractions and ion energies exceeding 30 eV enable room-temperature crystallization of dense cubic TiAlN films.

\section{Outlook}


Further work should focus on characterizing neutral species contributions to identify potential gaps in the flux balance. Improved measurement stability with more consistent ion currents would help resolve subtle trends obscured by current error bars. Systematic investigation of pressure-dependent sticking coefficients is needed to explain the flux behavior at higher pressures. These studies should begin with TiN as a simpler binary system before extending to multi-component materials like TiAlN.



