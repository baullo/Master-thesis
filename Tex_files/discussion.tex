\chapter{Discussion of Results and Outlook}\label{chap:discussion}
This chapter interprets the experimental findings presented in Chapter~\ref{chap:results} and connects them to the theoretical framework established in Chapter~\ref{chap:theory}. The discussion is organized around three central themes: plasma transport and composition, film growth mechanisms, and the unresolved flux measurement discrepancy.\\


\section{Plasma Transport and Magnetic Field Effects}\label{sec:magnetic_effect}
The anomalous decrease in both mass deposition rate and ion current at 0.05~T magnetic field (Figure~\ref{fig:metallic_mass/ion}) reflects a magnetic mirror effect at the entrance of the EM-coil, where plasma transitions from the weak fringing field ($\sim$0.01~T) outside into the stronger field (0.05~T) inside. Electrons with sufficient perpendicular velocity components are reflected, creating a localized space-charge layer that temporarily retards ion flow \cite{cathodic_arcs}. At higher fields (0.1~T and above), the beneficial effects of plasma compression and enhanced ionization overcome this mirror loss, restoring monotonic increase. This effect persists across all distances and pressures, indicating it is a fundamental feature of the plasma-coil interaction.\\

In metallic mode, mass deposition rate follows approximate $1/r^2$ dependence (Figure~\ref{fig:distance_0.25T_mass/ion}a), consistent with free plasma expansion. Ion current decreases more steeply in reactive mode due to charge-exchange neutralization. The magnetic field enhancement increases both mass flux and ion current by factors of 5--8 from zero field to 0.25~T (Figure~\ref{fig:mag_field_14cm_mass/ion}), demonstrating effective plasma confinement despite the inherent coupling between flux and ion charge states.


\section{Reactive Mode Plasma Chemistry}

The pressure dependence of ion current exhibits complex behavior at 10~cm (Figure~\ref{fig:pressure_0.25T_mass/ion}b), with an initial increase from 0 to $\sim$0.05~Pa before decreasing at higher pressures. At low nitrogen pressures (0--0.1~Pa), enhanced plasma generation at the cathode from altered spot dynamics and nitrogen ion production contribute to increased measured current. At higher pressures (above 0.1~Pa), charge-exchange collisions dominate:
\begin{equation}
	\text{Ti}^{3+} + \text{N}_2 \rightarrow \text{Ti}^{2+} + \text{N}_2^{+}
\end{equation}
\begin{equation}
	\text{Al}^{2+} + \text{N}_2 \rightarrow \text{Al}^{+} + \text{N}_2^{+}
\end{equation}

ERMS measurements confirm this mechanism: mean charge states decrease from $\langle Q \rangle \approx 2.4$ (metallic) to $\approx 1.6$ (0.3~Pa reactive), with charge-state distributions shifting toward lower ionization levels (Figures~\ref{fig:ERMS_10cm_pressure}--\ref{fig:ERMS_14cm_pressure}). Potential energy decreases proportionally, while kinetic energy shows modest variations reflecting competing effects of elastic scattering and cathode chemistry changes.\\


The non-monotonic pressure dependence of mass deposition rate at 10~cm (Figure~\ref{fig:pressure_0.25T_mass/ion}a), with maximum near 0.05--0.1~Pa, reflects competing effects of reactive film growth versus cathode poisoning and collisional deflection. At 14~cm and 20~cm, pressure independence indicates the plasma has undergone extensive equilibration, with collisional mean free paths ($\sim$1--10~cm at 0.1--0.3~Pa) comparable to transit distances.


\subsection{Activated Nitrogen Species}

EDX measurements (Table~\ref{tab:edx}) show substantial nitrogen incorporation in reactive mode films (33--39~at.\%), while ERMS detects only modest N$^+$/N$_2^+$ signals (10--28~at.\%) compared to metal ions. This imbalance suggests that nitrogen arrives predominantly through neutral channels: either ground-state N$_2$, vibrationally/electronically excited N$_2$*, or atomic nitrogen radicals N* produced by dissociative electron impact in the cathode spot plasma \cite{bendikt2012}.\\

The quantitative relationship between ion flux measurements and atomic flux calculations (Figure~\ref{fig:flux_vs_pressure}) shows systematic differences that require further investigation. Potential contributions include calibration uncertainties in the flux conversion factors and time averaging effects between pulsed and continuous measurements.\\


\section{Film Growth Mechanisms and Microstructure}

ERMS measurements confirm total ion energies (E$_{\text{kin}}$ + E$_{\text{pot}}$) remain above 30~eV across all conditions (Figures~\ref{fig:ERMS_10cm_pressure}--\ref{fig:ERMS_14cm_pressure}), exceeding the threshold for subplantation-driven densification \cite{lifshitz1990subplantation}. In metallic mode, total energies reach 60--80~eV; in reactive mode, 30--50~eV despite charge-exchange reduction. This energetic condensation enables room-temperature crystallization through shallow ion implantation and localized atomic-scale heating.\\


XRD results demonstrate that nitrogen incorporation is aiding for room-temperature crystallization. The metallic TiAl film (0~Pa) exhibits only weak hexagonal reflections despite receiving higher ion energies (50--60~eV from ERMS), indicating that energetic bombardment alone cannot produce well-crystallized films in this system. In contrast, all reactive mode films (0.1--0.3~Pa nitrogen) develop sharp cubic TiAlN (111) and (200) reflections even with lower ion energies (30--50~eV). This demonstrates that nitride formation provides the thermodynamic driving force ($\Delta H_f \approx -200$~kJ$\cdot$mol$^{-1}$ for TiN/AlN) necessary for crystallization, which ion bombardment alone cannot supply. Both chemical driving force and sufficient ion energy (>30~eV) are required for dense, crystalline growth at room temperature.

XRR measurements show nitrogen incorporation raises film density from 4.00~g$\cdot$cm$^{-3}$ (metallic) to 4.65--4.80~g$\cdot$cm$^{-3}$ (reactive), approaching the cubic TiAlN reference value of 4.8~g$\cdot$cm$^{-3}$ (Table~\ref{tab:xrr_results}). This 16--20\% increase reflects both the transition to the dense cubic phase and energetic ion bombardment promoting atomic rearrangement. The slight density reduction at 0.3~Pa (4.65~g$\cdot$cm$^{-3}$) compared to 0.1--0.2~Pa (4.80~g$\cdot$cm$^{-3}$) correlates with reduced XRD intensities, suggesting over-nitriding or secondary phase formation.\\


EDX measurements show systematic deviations from cathode stoichiometry (Table~\ref{tab:edx}). In metallic mode, the Ti:Al ratio shifts from 62.8:37.2 (cathode) to 79:21 (film), likely arising from differential erosion rates, ionization efficiencies, or transport through the magnetic filter. In reactive mode, the ratio shifts to $\sim$3.4:1, closer to cathode composition. Nitrogen saturation at 39~at.\% (0.2--0.3~Pa) approaches stoichiometric (Ti,Al)N (50~at.\% N), with the deficiency attributable to incomplete nitridation or measurement uncertainty.\\


XRR and profilometry reveal 5--8.5~nm thickness gradients between measurement positions (Tables~\ref{tab:profilometry_results}--\ref{tab:xrr_results}), attributable to plasma transport through the curved magnetic filter. The 90$^{\circ}$ filter acts as a plasma optical system, guiding electrons along field lines while ions follow electrostatically. Different species and neutrals follow slightly different trajectories, creating spatial variations in flux and composition. The pressure independence of this gradient confirms filter geometry, not gas-phase collisions, dominates the spatial distribution.


\subsection{Quantitative Plasma-Film Correlations}

The comprehensive characterization across plasma diagnostics and film properties enables establishing quantitative correlations that validate the energetic condensation model for TiAlN. Table~\ref{tab:plasma_film_correlation} summarizes the measured plasma parameters and resulting film properties across the pressure range at 14~cm distance with 0.25~T magnetic field.

\begin{table}[h]
	\centering
	\caption{Plasma parameters and film properties correlation at 14~cm, 0.25~T}
	\label{tab:plasma_film_correlation}
	\begin{tabular}{c c c c c c c}
		\toprule
		Pressure & Ion Flux & Atomic Flux & $\langle Q \rangle$ & E$_{\text{tot}}$ & Density & XRD Quality \\ 
		(Pa) & ($10^{17}$ cm$^{-2}$s$^{-1}$) & ($10^{17}$ cm$^{-2}$s$^{-1}$) & & (eV) & (g cm$^{-3}$) & \\ 
		\midrule
		0.0 & 0.7 & 0.75 & 2.4 & 50--70 & 4.00 & weak hex \\ 
		0.1 & 0.95 & 1.0 & 2.0 & 30--60 & 4.80 & strong cubic \\ 
		0.2 & 1.05 & 1.15 & 1.8 & 30--40 & 4.80 & strong cubic \\ 
		0.3 & 1.0 & 0.95 & 1.6 & 20--35 & 4.65 & strong cubic \\ 
		\bottomrule
	\end{tabular}
\end{table}

Several quantitative relationships emerge from this data. First, the ionization fraction $f_{\text{ion}} = \Gamma_{\text{ion}}/\Phi_{\text{total}}$ remains between 0.9 and 1.1 across all pressures, validating that deposition occurs through predominantly ionic pathways regardless of nitrogen pressure. This high ionization persists even as mean charge states decrease by 35\% from metallic to reactive mode, confirming that charge-exchange reduces individual ion charges without substantial neutralization.\\

Second, film density correlates with the product of ion flux and total energy, $\Gamma_{\text{ion}} \times E_{\text{tot}}$, which represents the rate of energy delivery to the growing film. At 0~Pa, this product is approximately $0.7 \times 10^{17} \times 60 \approxeq 4 \times 10^{19}$~eV$\cdot$cm$^{-2}$$\cdot$s$^{-1}$, yet density remains only 4.00~g$\cdot$cm$^{-3}$ due to lack of chemical driving force. At 0.1~Pa, the product stays similar $0.95 \times 10^{17} \times 45 \approxeq 4 \times 10^{19}$~eV$\cdot$cm$^{-2}$$\cdot$s$^{-1}$, but density increases to 4.80~g$\cdot$cm$^{-3}$ due to nitride formation. This demonstrates that absolute energy delivery rate does not uniquely determine density. Rather the combination of sufficient energy ($>30$~eV) plus thermodynamic driving force is required.\\

Third, XRD peak intensities increase systematically with nitrogen pressure, with 0.3~Pa showing the sharpest (111), (200), and (220) reflections despite having slightly lower density (4.65~g$\cdot$cm$^{-3}$) compared to 0.1--0.2~Pa (4.80~g$\cdot$cm$^{-3}$). This indicates that XRD crystalline quality and XRR mass density optimize at different pressures within the reactive window.\\

Fourth, atomic flux exhibits non-monotonic pressure dependence with maximum at 0.2~Pa (54\% increase over metallic mode), while ion flux remains approximately constant. This indicates that reactive film formation enhances mass deposition through incorporation of nitrogen atoms from the gas phase, while the metal ion flux (the plasma component) remains determined primarily by arc current and magnetic field. The atomic flux decrease at 0.3~Pa suggests collisional deflection begins to dominate over reactive enhancement at higher pressures.\\

These correlations provide practical guidelines for process optimization: to maximize density, maintain ion energies above 30~eV (achievable with nitrogen pressure $\leq 0.2$~Pa at this magnetic field) and ensure ion flux reach $10^{17}$~ions$\cdot$cm$^{-2}$$\cdot$s$^{-1}$ (requiring magnetic field $\geq 0.2$~T at 14~cm distance).





\section{Outlook}



This work successfully demonstrated room-temperature TiAlN crystallization through energetic condensation, validated near-unity ionization fractions ($f_{\text{ion}} = 0.92$--$1.05$) across reactive and metallic regimes, and quantified magnetic field enhancement factors of 5--8$\times$ for ion flux. The key finding that nitrogen pressure reduces charge states by 35--40\% while maintaining approximately constant ion flux provides a control pathway for flux-energy decoupling that magnetic field strength alone cannot achieve. While the inherent coupling between magnetic confinement and plasma temperature prevents complete independence of flux and charge states through magnetic field control, the combination of magnetic field (controlling absolute flux magnitude) and nitrogen pressure (controlling charge states and hence potential energy) enables optimization within the identified process window of 0.1--0.2~Pa nitrogen, 0.25~T magnetic field, yielding ion energies of 30--50~eV and fluxes of $1.0$--$1.15 \times 10^{17}$~ions/atoms$\cdot$cm$^{-2}$$\cdot$s$^{-1}$.\\

An important unresolved aspect of plasma characterization is the contribution of neutral species to film growth. While the validated near-unity ionization fractions confirm that metal atoms arrive predominantly as ions, nitrogen incorporation in films (33--39~at.\% from EDX) significantly exceeds nitrogen ion signals in ERMS (10--28~at.\% relative to metal ions), indicating that some nitrogen arrives through neutral channels. Optical emission spectroscopy (OES) synchronized with arc pulses would enable identification and quantification of excited neutral species including N$_2^*$, N$^*$, Ti$^*$, and Al$^*$ through characteristic emission lines. Such diagnostics would provide independent validation of ionization degrees and quantify neutral contributions that remain unmeasured in this work, enabling complete mass balance accounting across all depositing species.\\

Computational modeling offers a path from empirical parameter optimization toward predictive process design. Particle-in-cell (PIC) simulations of plasma transport through the curved magnetic filter could predict spatial flux distributions and explain observed thickness gradients (25--30\%), while molecular dynamics simulations of energetic ion impacts would reveal atomic-scale mechanisms of subplantation, defect generation and crystallization. Training models on the comprehensive datasets collected here (flux versus distance/pressure/magnetic field and ERMS energy distributions) would enable interpolation between measured conditions and extrapolation to unexplored parameter spaces.\\

The broader scientific challenge is that cathodic arc deposition of multi-element plasmas involves coupled processes beyond simple superposition of single-element behaviors. The Ti-Al system studied here exhibits compositional shifts (79:21 films versus 62.8:37.2 cathode), pressure-dependent variations and spatial gradients arising from species-dependent ionization and charge-state distributions, mass-dependent magnetic transport and potentially species-dependent charge-exchange cross sections. Changing a single parameter produces cascading consequences throughout plasma chemistry and film growth. Future systematic investigation should map these coupled interactions across multiple cathode compositions, reactive gases pressures, and geometric configurations to identify universal principles versus system-specific parameters. Building this knowledge base would transform cathodic arc deposition from a mature but empirically optimized technology into a predictable process, where coating properties can be tailored to specific applications based on fundamental understanding of the plasma-surface interactions governing film growth.
