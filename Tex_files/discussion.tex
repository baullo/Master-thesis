\chapter{Discussion and Summary}\label{chap:discussion}
This chapter interprets the experimental findings presented in Chapter~\ref{chap:results} and connects them to the theoretical framework established in Chapter~\ref{chap:theory}. The discussion is organized around two central themes, first the film growth mechanisms and then to conclude the  plasma transport.

\section{Film Growth and Microstructure}\label{sec:film_analysis}

XRD results demonstrate room-temperature crystallization of TiAlN with nitrogen incorporation. The metallic TiAl film (0 Pa) exhibits a broad weak hexagonal reflection, while all reactive mode films (0.1--0.3 Pa nitrogen) develop clearly visible peaks, representing polycrystalline TiAlN (111), (200) and (220) reflections.\\

XRR measurements show the measured densities are in good agreement with reference values for their respective crystal structures (Table~\ref{tab:xrr_results}). The metallic TiAl film shows 4.00 g$\cdot$cm$^{-3}$, while TiAlN films reach densities around 4.65--4.80 g$\cdot$cm$^{-3}$ in good agreement with the reported theoretical value for cubic TiAlN of 4.8 g$\cdot$cm$^{-3}$ \cite{ICDD_TiAlN}. \\

EDX measurements show systematic deviations from cathode chemical composition (Table~\ref{tab:edx}). In metallic mode, the Ti:Al ratio shifts from 62:37 (cathode) to 79:21 (film), likely arising from differential erosion rates, ionization efficiencies, or transport through the magnetic filter. In reactive mode, the Ti:Al ratio remains similar (77:23) to the metallic films with nitrogen at 39 at.\% (0.2--0.3 Pa) not reaching the theoretical concentration of 50 at.\%.\\

XRR and profilometry reveal 5--8.5 nm thickness gradients between measurement positions (Tables~\ref{tab:profilometry_results}--\ref{tab:xrr_results}), attributable to plasma transport through the curved magnetic filter. The 90 degree filter guides electrons along field lines while ions follow electrostatically. Different species follow slightly different trajectories, leading to potential spatial variations in flux and composition \cite{Bilek1996FilterTransport}.

\section{Plasma transport and magnetic field effects}\label{sec:magnetic_effect}

The anomalous decrease in both mass deposition rate and ion current at 0.05 T magnetic field (Figure~\ref{fig:metallic_mass/ion}) reflects a magnetic mirror effect at the entrance of the EM-coil, where plasma transitions from the weak fringing field (approximately 0.01 T) outside into the stronger field (0.05 T) inside. Electrons with sufficient perpendicular velocity components are reflected, creating a localized space-charge layer that temporarily retards ion flow \cite{cathodic_arcs}. At higher fields (0.1 T and above), the beneficial effects of plasma compression and enhanced ionization overcome this mirror loss, restoring monotonic increase. This effect persists across all distances and pressures, indicating it is a fundamental feature of the plasma-coil interaction.\\

In metallic mode, mass deposition rate follows approximate $1/r^2$ dependence (Figure~\ref{fig:distance_0.25T_mass/ion}a), consistent with free plasma expansion. Ion current decreases more steeply in reactive mode due to charge-exchange collision and therefore neutralization. The magnetic field enhancement increases both mass flux and ion current by factors of 5 to 8 from zero field to 0.25 T (Figure~\ref{fig:mag_field_14cm_mass/ion}), demonstrating effective plasma enhancement.


\section{Reactive mode plasma chemistry}

The pressure dependence of mass flux exhibits complex behavior at 10 cm (Figure~\ref{fig:pressure_0.25T_mass/ion}b), with an initial decrease from 0 to 0.05 Pa before decreasing at higher pressures. At low nitrogen pressures (0--0.1 Pa), enhanced plasma generation at the cathode from altered spot dynamics and nitrogen ion production contribute to increased measured mass flux. However the ion current show different characteristics, which is gradually decreasing with increasing nitrogen pressure. This is due to the multiple reactions possible with the background gas, for example reduction of the metallic species can lead to ionization, dissociation and excitations.\\

ERMS measurements confirm this mechanism: mean charge states decrease from $\langle Q \rangle \approx 2.4$ (metallic) to $\approx 1.6$ (0.3 Pa reactive), with charge-state distributions shifting toward lower ionization levels (Figures~\ref{fig:ERMS_10cm_pressure}--\ref{fig:ERMS_14cm_pressure}). This represents a 35\% reduction in average charge state. Importantly, the total ion energy (kinetic ion energy + potential ion energy) remains above 30 eV across all conditions, sufficient for subplantation processes. \\

Table~\ref{tab:plasma_film_correlation} summarizes the measured plasma parameters and resulting film properties across the pressure range at 14 cm distance with 0.25 T magnetic field.

\begin{table}[h]
	\centering
	\caption{Plasma parameters and film properties correlation at 14 cm, 0.25 T}
	\label{tab:plasma_film_correlation}
	\begin{tabular}{c c c c c c}
		\toprule
		Pressure & Ion Flux & total Flux & $\langle Q \rangle$  & Density & XRD Quality \\ 
		(Pa) & ($10^{17}$ cm$^{-2}$s$^{-1}$) & ($10^{17}$ cm$^{-2}$s$^{-1}$) &  & (g cm$^{-3}$) & \\ 
		\midrule
		0.0 & 0.6 & 0.75 & 2.4  & 4.00 & broad hexagonal peak \\ 
		0.1 & 0.8 & 1.0 & 2.0  & 4.80 & sharp phase cubic peak \\ 
		0.2 & 0.95 & 1.15 & 1.8  & 4.80 & sharp phase cubic peak \\ 
		0.3 & 0.95 & 0.95 & 1.6  & 4.65 & sharp phase cubic peak \\ 
		\bottomrule
	\end{tabular}
\end{table}
The ionization fraction $f_{\text{ion}} = \Gamma_{\text{ion}}/\Phi_{\text{total}}$ remains between 0.8 and 1 across all pressures. This high ionization persists even as mean charge states decrease by 35\% from metallic to reactive mode, confirming that charge-exchange reduces individual ion charges without substantial neutralization of the metal plasma. In reactive mode, the ionization fraction additionally benefits from activated (ionized and otherwise excited) nitrogen species generated through plasma-gas interactions, contributing to the measured ion flux while the metal plasma itself remains highly ionized.\\


Total flux exhibits non-monotonic pressure dependence with maximum at 0.2 Pa (54\% increase over metallic mode), while ion flux remains approximately constant. This indicates that reactive film formation enhances mass deposition through incorporation of nitrogen atoms from the gas phase, while the metal ion flux remains determined primarily by arc current and magnetic field. The total flux decrease at 0.3 Pa is unknown and needs further investigation to clarify.

\subsection{Activated nitrogen species}

EDX measurements (Table~\ref{tab:edx}) show substantial nitrogen incorporation in reactively deposited films (33--39 at.\%), while ERMS measurements detect weaker N$^+$/N$_2^+$ signals (10--28 at.\%) compared to the total nitrogen in films. The imbalance suggests that nitrogen has some excited species but since ERMS estimates the nitrogen contents via the overall species counts, the imbalance can only be used as a hint towards activated nitrogen.

\newpage
\section{Summary and Conclusions}

This thesis addressed the challenge of understanding ion flux and energy effects in room-temperature TiAlN film growth via filtered cathodic arc deposition. While previous work on (V,Al)N demonstrated that magnetic fields enhance both ion charge states and ion flux simultaneously \cite{unutulmazsoy,decoupling_kalanov_2025}, the individual role of ion flux as an independent variable remained unresolved. This work extended that framework to the industrially relevant TiAlN system by combining QCM and ion probe measurements with ERMS charge-state analysis and film characterization (XRD, XRR, EDX, SEM and Profilometry). Measurements across varying magnetic field strength (0--0.25~T), nitrogen pressure (0--0.3~Pa), and spatial position (10--20~cm) enabled quantification of ion and atomic fluxes while distinguishing contributions from ionized versus neutral nitrogen species.\\



The methodology revealed near-unity ionization fractions across all conditions, demonstrating the dominant role of ions in thin film deposition. High magnetic fields increased both mass flux and ion current, while charge-exchange collisions in reactive mode reduced mean charge states by 35\% without proportionally decreasing ion flux. Film analysis confirmed that ion energies exceeding 30 eV enable room-temperature crystallization of dense cubic TiAlN films.


\section{Outlook}


Further work should focus on characterizing neutral species contributions to identify potential gaps in the flux balance. Improved measurement stability with more consistent ion currents would help to resolve subtle trends obscured by current error bars. Systematic investigation of pressure-dependent sticking coefficients is needed to explain the flux behavior at higher pressures. A potential material for those investigations can be TiN as a simpler binary system before extending to multi-component materials like TiAlN.



